\documentclass[crop, tikz, multi=false]{standalone}

\usepackage{xcolor}
\usepackage{tikzstyle}

\usetikzlibrary{
  arrows.meta, calc, decorations.pathreplacing, ext.paths.ortho, 
  positioning, quotes, shapes, backgrounds, calligraphy, math,
  fadings, decorations.text
}

% Fading styles
\tikzfading[name=fade right, 
    left color=transparent!0,
    right color=transparent!100,
    middle color=transparent!90,
]
\tikzfading[name=fade left,
    left color=transparent!100,
    right color=transparent!0,
    middle color=transparent!90,
]

\begin{document}

\begin{tikzpicture}

    \node[neuron, fill=green!30] (in0) {};
    \foreach\i in {1,...,3} {
        \tikzmath{int \ii; \ii = \i - 1;}
        \node[below = 4ex of in\ii, neuron, fill=green!30] (in\i) {};
    }

    \node[above right = 1em and 6em of in0, neuron, fill=orange!50!yellow!10] (hid1_0) {};
    \foreach\i in {1,...,4} {
        \tikzmath{int \ii; \ii = \i - 1;}
        \node[below = 2em of hid1_\ii, neuron, fill=orange!50!yellow!10] (hid1_\i) {};
    }
    \foreach\i in {0,...,3} {
        \foreach\j in {0,...,4}
            \draw[line, semithick, color=black!30!green] (in\i) -- (hid1_\j);
    }

    \node[
        above left = 1em and 1em of hid1_0, 
        neuron, fill=gray!30, minimum width=0.1em, minimum height=0.1em,
    ] (bias1) {\scriptsize $1$};
    \foreach\i in {0,...,4} {
        \draw[line, semithick, color=gray!80] (bias1) -- (hid1_\i.north west);
    }

    \node[right = 6em of hid1_0, neuron, fill=orange!50!yellow!10] (hid2_0) {};
    \foreach\i in {1,...,4} {
        \tikzmath{int \ii; \ii = \i - 1;}
        \node[below = 2em of hid2_\ii, neuron, fill=orange!50!yellow!10] (hid2_\i) {};
    }
    \foreach\i in {0,...,4} {
        \foreach\j in {0,...,4}
            \draw[line, semithick, color=black!80] (hid1_\i) -- (hid2_\j);
    }

    \node[
        above left = 1em and 1em of hid2_0, 
        neuron, fill=gray!30, minimum width=0.1em, minimum height=0.1em,
    ] (bias2) {\scriptsize $1$};
    \foreach\i in {0,...,4} {
        \draw[line, semithick, color=gray!80] (bias2) -- (hid2_\i.north west);
    }

    \node[right = 6em of hid2_0, neuron, draw=none] (hid3_0) {};
    \foreach\i in {1,...,4} {
        \tikzmath{int \ii; \ii = \i - 1;}
        \node[below = 2em of hid3_\ii, neuron, draw=none] (hid3_\i) {};
    }
    \foreach\i in {0,...,4} {
        \foreach\j in {0,...,4}
            \filldraw[line, semithick, color=black!80, path fading=fade right] (hid2_\i) -- (hid3_\j);
    }

    \node[right = 6em of hid3_0, neuron, draw=none] (hid4_0) {};
    \foreach\i in {1,...,4} {
        \tikzmath{int \ii; \ii = \i - 1;}
        \node[below = 2em of hid4_\ii, neuron, draw=none] (hid4_\i) {};
    }

    \node[right = 9em of hid2_2, neuron, draw=none] (dots) {\huge $\cdots$};

    \node[right = 6em of hid4_0, neuron, fill=orange!50!yellow!10] (hid5_0) {};
    \foreach\i in {1,...,4} {
        \tikzmath{int \ii; \ii = \i - 1;}
        \node[below = 2em of hid5_\ii, neuron, fill=orange!50!yellow!10] (hid5_\i) {};
    }
    \foreach\i in {0,...,4} {
        \foreach\j in {0,...,4}
            \filldraw[line, semithick, color=black!80, path fading=fade left] (hid4_\i) -- (hid5_\j);
    }

    \node[
        above left = 1em and 1em of hid5_0, 
        neuron, fill=gray!30, minimum width=0.1em, minimum height=0.1em,
    ] (bias5) {\scriptsize $1$};
    \foreach\i in {0,...,4} {
        \draw[line, semithick, color=gray!80] (bias5) -- (hid5_\i.north west);
    }

    \node[below right = 1em and 6em of hid5_0, neuron, fill=cyan!30] (out0) {};
    \foreach\i in {1,...,2} {
        \tikzmath{int \ii; \ii = \i - 1;}
        \node[below = 3.6em of out\ii, neuron, fill=cyan!30] (out\i) {};
    }
    \foreach\i in {0,...,4} {
        \foreach\j in {0,...,2}
            \draw[line, semithick, black!25!cyan] (hid5_\i) -- (out\j);
    }

    \node[above = 12em of out0, draw=none](a){};
    \draw [
        arrow, line width = 0.2em, black!20!red,
        postaction={
            decorate,
            decoration={
                raise=1em, text along path, text align=center, reverse path,
                text color=black!20!red,
                text={|\Large\fontfamily{phv}\selectfont|Backwards Pass (Backpropagation)}
            }
        }
    ] ([yshift=5em]out0.north) to[out=90, in=90, bend right=20] ([yshift=5em]in0.north);

    \node[below = 6em of out2, draw=none](b){};
    \draw [
        arrow, line width = 0.2em, black!10!orange,
        postaction={
            decorate,
            decoration={
                raise=1em, text along path, text align=center,
                text color=black!10!orange,
                text={|\Large\fontfamily{phv}\selectfont|Forward Pass}
            }
        }
    ] ([yshift=-6em]in3.south) -- ([yshift=-6em]out2.south);

    \node[above left = 1em and 0em of in0, helv, text=black!35!green, align=center] () {\large Input \tikznewl{\large Layer}};
    \node[above right = 1em and 0em of out0, helv, text=black!30!cyan, align=center] () {\large Output \tikznewl{\large Layer}};
    \node[above = 2em of hid3_0, helv, align=center] () {\large Hidden Layers};

    \node[
        below right = 0em and 1.5em of in1,
        archblock, minimum width = 2em,
        fill=green!30,
    ](w1){\Large $W_1$};
    \node[
        right = 1.5em of hid1_2,
        archblock, minimum width = 2em,
        fill=orange!50!yellow!10,
    ](w2){\Large $W_2$};
    \node[
        right = 1.5em of hid5_2,
        archblock, minimum width = 2em,
        fill=cyan!30,
    ](wd){\Large $W_d$};
\end{tikzpicture}

\end{document}