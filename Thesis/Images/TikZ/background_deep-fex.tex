\documentclass[crop, tikz, multi=false]{standalone}

\usepackage{xcolor}
\usepackage{tikzstyle}

\usetikzlibrary{
  arrows.meta, calc, decorations.pathreplacing, ext.paths.ortho, 
  positioning, quotes, shapes, backgrounds, calligraphy, math
}

\begin{document}

\begin{tikzpicture}

    \node[neuron, fill=green!30] (in0) {};
    \foreach\i in {1,...,3} {
        \tikzmath{int \ii; \ii = \i - 1;}
        \node[below = 4ex of in\ii, neuron, fill=green!30] (in\i) {};
    }

    \node[above right = 1em and 5em of in0, neuron, fill=orange!30] (enc0) {};
    \foreach\i in {1,...,4} {
        \tikzmath{int \ii; \ii = \i - 1;}
        \node[below = 2em of enc\ii, neuron, fill=orange!30] (enc\i) {};
    }
    \foreach\i in {0,...,3} {
        \foreach\j in {0,...,4}
            \draw[line, semithick, color=black!30!green] (in\i) -- (enc\j);
    }

    \node[right = 4em of enc1, neuron, fill=orange!30] (encc0) {};
    \foreach\i in {1,...,2} {
        \tikzmath{int \ii; \ii = \i - 1;}
        \node[below = 2em of encc\ii, neuron, fill=orange!30] (encc\i) {};
    }
    \foreach\i in {0,...,4} {
        \foreach\j in {0,...,2}
            \draw[line, semithick, black!20!orange] (enc\i) -- (encc\j);
    }

    \node[right = 5em of encc0, neuron, fill=yellow!50] (decc0) {};
    \foreach\i in {1,...,2} {
        \tikzmath{int \ii; \ii = \i - 1;}
        \node[below = 2em of decc\ii, neuron, fill=yellow!50] (decc\i) {};
    }
    \foreach\i in {0,...,2} {
        \foreach\j in {0,...,2}
            \draw[line, semithick] (encc\i) -- (decc\j);
    }

    \node[right = 5em+4*2em+2*2em of enc0, neuron, fill=yellow!50] (dec0) {};
    \foreach\i in {1,...,4} {
        \tikzmath{int \ii; \ii = \i - 1;}
        \node[below = 2em of dec\ii, neuron, fill=yellow!50] (dec\i) {};
    }
    \foreach\i in {0,...,2} {
        \foreach\j in {0,...,4}
            \draw[line, semithick, black!20!yellow] (decc\i) -- (dec\j);
    }

    \node[below right = 1em and 5em of dec0, neuron, fill=cyan!30] (out0) {};
    \foreach\i in {1,...,2} {
        \tikzmath{int \ii; \ii = \i - 1;}
        \node[below = 3.6em of out\ii, neuron, fill=cyan!30] (out\i) {};
    }
    \foreach\i in {0,...,4} {
        \foreach\j in {0,...,2}
            \draw[line, semithick, black!25!cyan] (dec\i) -- (out\j);
    }

    \node[above = 1em of in0, figb, text=black!35!green, align=center] () {\Large Raw \tikznewl{\Large Input} \tikznewl{\Large Features}};
    \node[right = 1em of enc0, figb, text=black!30!orange, align=center] () {\Large Encoding};
    \node[left = 1em of dec0, figb, text=black!30!yellow, align=center] () {\Large Decoding};
    \node[above = 1.5em of out0, figb, text=black!30!cyan, align=center] () {\Large Classification};

    \draw[line, black!50!red] ([yshift=1.5em]enc0.west) -- ++(0,1.5em) -| node[figb, above,align=center, pos=0.25]{\Large Feature Extraction} ([yshift=1.5em]dec0.east);

    \draw[arrow, dashed, violet] ([yshift=-0.5em]encc2.south) -- ++(0,-6em) coordinate(latent);
    
    \node[below left = 1em and 4em of latent] (latent_start) {};
    \hboxvec(latent_start)({3,7,9,2,1}:{}:{5})(
        draw = violet,
        fill = violet!30,
    )(1:0:3em)

    \node[right = 2em of hboxvec_last, figb, text=black!10!violet, align=center] () {\Large Latent \tikznewl{\Large Representation}};

\end{tikzpicture}

\end{document}