In this work, we have explored various aspect of transformer interpretability, focusing on the creation of a specialized tool that enables the visualization of internal states of transformer-based language models: \textbf{InTraVisTo}.
InTraVisTo was developed with research in mind; its main purpose is to assist researchers and practitioners in the NLP field in observing the internal reasoning steps carried out by LLMs in an immediate and interactive way.
Moreover, its easily configurable, highly optimized and multi-user oriented implementation allows for an overall user-friendly experience centered around the discovery of LLM architectures through the lens of their own embeddings.
InTraVisTo answers the first identified research question in~\cref{sec:rq_intravisto} by providing an expanded and general view on the topic of vocabulary space decoding for LLMs.
The inclusion of the information flow visualization and the possibility of interacting with the generative process of the model from the inside offer a unique experience tailored towards the preliminary exploration and investigation process performed by NLP researchers.

The development of InTraVisTo introduced a series of additional research questions that required separate investigation in order to address significant issues and validate methodologies employed in the final application.
The first of these secondary inquiries is represented by the second research question formulated in~\cref{sec:rq_embeddings} and explored in~\cref{sec:exp_emb}.
We examined the behavior of embeddings belonging to autoregressive LLMs, with the primary goal of understanding whether these recent model retain linear properties in their embedding spaces.
Our analysis revealed that, while these properties are still present in LLMs to some extent, their prominence varies across different architectures, model sizes and is particularly influenced by the tokenization process and overall training objectives.
Additionally, our comparison of embedding and unembedding spaces highlighted crucial differences in their functional roles; in models where these two spaces are distinct, we observed that input embeddings contain more semantic information than the unembedding spaces.
As already mentioned in~\cref{ssec:exp_emb_discussion}, the proposed study on embedding linearity significantly drove the implementation of the vocabulary decoding techniques for hidden representations in InTraVisTo.
In particular, both the concept of interpolation and the idea of performing interpolation in a nonlinear manner stem from results obtained while pursuing the second research question.

The third research question (\cref{sec:exp_fom}) addressed the possibility of obtaining first-order predictions by concatenating the input and output embedding spaces of LLMs.
Building upon previous findings, we investigated whether such concatenation effectively replicates the behavior of a Markov model following the theoretical groundwork laid by~\citet{elhage2021}.
Our results indicate that the first order models (FOMs) extracted from most LLMs are generally faithful approximations of the corresponding Markov models and exhibit a solid bias towards the next-token prediction task, although the degree of similarity to the Markov model varies greatly across model architectures and weights.
Furthermore, our analysis investigates the compatibility of our results with the practice of weight tying, concluding that the benefit from setting the unembedding weights identical to the embedding ones is limited to a reduction in the number of overall model parameters.
The resulting findings tie back into InTraVisTo's view of the model by identifying the residual flow as the main communication channel between embeddings, with each component progressively adding information as the prediction progresses through the layers.

Future work could extend InTraVisTo by adding support for new decoding techniques, model architectures and injection approaches.
Additional developments addressing the secondary aspects treated in this work could include further experimenting and investigation on the impact of multilingual training on the embedding space LLMs, as well as further considerations on the feasibility of interpolating embeddings.
As already noted in~\cref{ssec:exp_fom_discussion}, we also acknowledge the possibility of utilizing a Markov model to initialize the embedding and unembedding weights pair for model training, which could be an interesting future development.

In conclusion, this thesis confirms previous findings and brings novel perspectives in the NLP field by providing a valuable interactive visualization tool designed to enhance the study of transformer interpretability: InTraVisTo.
By providing insight and innovative tools to the explainability field, we aim to contribute to the broader goal of making LLMs more transparent and interpretable.
