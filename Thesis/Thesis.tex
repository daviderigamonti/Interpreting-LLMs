% A LaTeX template for MSc Thesis submissions to 
% Politecnico di Milano (PoliMi) - School of Industrial and Information Engineering
%
% S. Bonetti, A. Gruttadauria, G. Mescolini, A. Zingaro
% e-mail: template-tesi-ingind@polimi.it
%
% Last Revision: October 2021
%
% Copyright 2021 Politecnico di Milano, Italy. NC-BY

\documentclass{Configuration_Files/PoliMi3i_thesis}

%------------------------------------------------------------------------------
%	REQUIRED PACKAGES AND  CONFIGURATIONS
%------------------------------------------------------------------------------

% CONFIGURATIONS
\usepackage{parskip} % For paragraph layout
\usepackage{setspace} % For using single or double spacing
\usepackage{emptypage} % To insert empty pages
\usepackage{multicol} % To write in multiple columns (executive summary)
\setlength\columnsep{15pt} % Column separation in executive summary
\setlength\parindent{0pt} % Indentation
\raggedbottom  

% PACKAGES FOR TITLES
\usepackage{titlesec}
% \titlespacing{\section}{left spacing}{before spacing}{after spacing}
\titlespacing{\section}{0pt}{3.3ex}{2ex}
\titlespacing{\subsection}{0pt}{3.3ex}{1.65ex}
\titlespacing{\subsubsection}{0pt}{3.3ex}{1ex}
\usepackage{color}

% PACKAGES FOR LANGUAGE AND FONT
\usepackage[english]{babel} % The document is in English  
\usepackage[utf8]{inputenc} % UTF8 encoding
\usepackage[T1]{fontenc} % Font encoding
\usepackage[11pt]{moresize} % Big fonts

% PACKAGES FOR IMAGES
\usepackage{graphicx}
\usepackage{transparent} % Enables transparent images
\usepackage{eso-pic} % For the background picture on the title page
\usepackage{subfig} % Numbered and caption subfigures using \subfloat.
\usepackage{tikz} % A package for high-quality hand-made figures.
\usetikzlibrary{}
\graphicspath{{./Images/}} % Directory of the images
\usepackage{caption} % Coloured captions
\usepackage{xcolor} % Coloured captions
\usepackage{amsthm,thmtools,xcolor} % Coloured "Theorem"
\usepackage{float}

% STANDARD MATH PACKAGES
\usepackage{amsmath}
\usepackage{amsthm}
\usepackage{amssymb}
\usepackage{amsfonts}
\usepackage{bm}
\usepackage[overload]{empheq} % For braced-style systems of equations.
\usepackage{fix-cm} % To override original LaTeX restrictions on sizes

% PACKAGES FOR TABLES
\usepackage{tabularx}
\usepackage{longtable} % Tables that can span several pages
\usepackage{colortbl}

% PACKAGES FOR ALGORITHMS (PSEUDO-CODE)
\usepackage{algorithm}
\usepackage{algorithmic}

% PACKAGES FOR REFERENCES & BIBLIOGRAPHY
\usepackage[colorlinks=true,linkcolor=black,anchorcolor=black,citecolor=black,filecolor=black,menucolor=black,runcolor=black,urlcolor=black]{hyperref} % Adds clickable links at references
\usepackage{cleveref}
\usepackage[square, numbers, sort&compress]{natbib} % Square brackets, citing references with numbers, citations sorted by appearance in the text and compressed
\bibliographystyle{abbrvnat} % You may use a different style adapted to your field

% OTHER PACKAGES
\usepackage{pdfpages} % To include a pdf file
\usepackage{afterpage}
\usepackage{lipsum} % DUMMY PACKAGE
\usepackage{fancyhdr} % For the headers
\fancyhf{}

% Input of configuration file. Do not change config.tex file unless you really know what you are doing. 
% Define blue color typical of polimi
\definecolor{bluepoli}{cmyk}{0.4,0.1,0,0.4}

% Custom theorem environments
\declaretheoremstyle[
  headfont=\color{bluepoli}\normalfont\bfseries,
  bodyfont=\color{black}\normalfont\itshape,
]{colored}

% Set-up caption colors
\captionsetup[figure]{labelfont={color=bluepoli}} % Set colour of the captions
\captionsetup[table]{labelfont={color=bluepoli}} % Set colour of the captions
\captionsetup[algorithm]{labelfont={color=bluepoli}} % Set colour of the captions

\theoremstyle{colored}
\newtheorem{theorem}{Theorem}[chapter]
\newtheorem{proposition}{Proposition}[chapter]

% Enhances the features of the standard "table" and "tabular" environments.
\newcommand\T{\rule{0pt}{2.6ex}}
\newcommand\B{\rule[-1.2ex]{0pt}{0pt}}

% Pseudo-code algorithm descriptions.
\newcounter{algsubstate}
\renewcommand{\thealgsubstate}{\alph{algsubstate}}
\newenvironment{algsubstates}
  {\setcounter{algsubstate}{0}%
   \renewcommand{\STATE}{%
     \stepcounter{algsubstate}%
     \Statex {\small\thealgsubstate:}\space}}
  {}

% New font size
\newcommand\numfontsize{\@setfontsize\Huge{200}{60}}

% Title format: chapter
\titleformat{\chapter}[hang]{
\fontsize{50}{20}\selectfont\bfseries\filright}{\textcolor{bluepoli} \thechapter\hsp\hspace{2mm}\textcolor{bluepoli}{|   }\hsp}{0pt}{\huge\bfseries \textcolor{bluepoli}
}

% Title format: section
\titleformat{\section}
{\color{bluepoli}\normalfont\LARGE\bfseries}
{\color{bluepoli}\thesection.}{1em}{}

% Title format: subsection
\titleformat{\subsection}
{\color{bluepoli}\normalfont\Large\bfseries}
{\color{bluepoli}\thesubsection.}{1em}{}

% Title format: subsubsection
\titleformat{\subsubsection}
{\color{bluepoli}\normalfont\large\bfseries}
{\color{bluepoli}\thesubsubsection.}{1em}{}

% Shortening for setting no horizontal-spacing
\newcommand{\hsp}{\hspace{0pt}}

\makeatletter
% Renewcommand: cleardoublepage including the background pic
\renewcommand*\cleardoublepage{%
  \clearpage\if@twoside\ifodd\c@page\else
  \null
  \AddToShipoutPicture*{\BackgroundPic}
  \thispagestyle{empty}%
  \newpage
  \if@twocolumn\hbox{}\newpage\fi\fi\fi}
\makeatother

%For correctly numbering algorithms
\numberwithin{algorithm}{chapter}

%----------------------------------------------------------------------------
%	NEW COMMANDS DEFINED
%----------------------------------------------------------------------------

% EXAMPLES OF NEW COMMANDS
\newcommand{\bea}{\begin{eqnarray}} % Shortcut for equation arrays
\newcommand{\eea}{\end{eqnarray}}
\newcommand{\e}[1]{\times 10^{#1}}  % Powers of 10 notation

%----------------------------------------------------------------------------
%	ADD YOUR PACKAGES (be careful of package interaction)
%----------------------------------------------------------------------------

\usepackage[threshold=1, thresholdtype=words]{csquotes}

% TODO: REMOVE after draft
\usepackage{soul}

%----------------------------------------------------------------------------
%	ADD YOUR DEFINITIONS AND COMMANDS (be careful of existing commands)
%----------------------------------------------------------------------------

\renewcommand\mkblockquote[4]{\leavevmode\llap{``}\textit{#1#2#3}''#4} % Blockquote quotation marks

% TODO: REMOVE after draft
\newcommand{\todo}[2][yellow]{\sethlcolor{#1}\texthl{#2}} % TODO highlight

%----------------------------------------------------------------------------
%	BEGIN OF YOUR DOCUMENT
%----------------------------------------------------------------------------

\begin{document}

\fancypagestyle{plain}{%
\fancyhf{} % Clear all header and footer fields
\fancyhead[RO,RE]{\thepage} %RO=right odd, RE=right even
\renewcommand{\headrulewidth}{0pt}
\renewcommand{\footrulewidth}{0pt}}

%----------------------------------------------------------------------------
%	TITLE PAGE
%----------------------------------------------------------------------------

\pagestyle{empty} % No page numbers
\frontmatter % Use roman page numbering style (i, ii, iii, iv...) for the preamble pages

\puttitle{
	title=Title, % Title of the thesis
	name=Name Surname, % Author Name and Surname
	course=Xxxxxxx Engineering - Ingegneria Xxxxxxx, % Study Programme (in Italian)
	ID  = 000000,  % Student ID number (numero di matricola)
	advisor= Prof. Name Surname, % Supervisor name
	coadvisor={Name Surname, Name Surname}, % Co-Supervisor name, remove this line if there is none
	academicyear={20XX-XX},  % Academic Year
} % These info will be put into your Title page 

%----------------------------------------------------------------------------
%	PREAMBLE PAGES: ABSTRACT (inglese e italiano), EXECUTIVE SUMMARY
%----------------------------------------------------------------------------
\startpreamble
\setcounter{page}{1} % Set page counter to 1

% ABSTRACT IN ENGLISH
\chapter*{Abstract} 
Here goes the Abstract in English of your thesis followed by a list of keywords.
The Abstract is a concise summary of the content of the thesis (single page of text)
and a guide to the most important contributions included in your thesis.
The Abstract is the very last thing you write.
It should be a self-contained text and should be clear to someone who hasn't (yet) read the whole manuscript.
The Abstract should contain the answers to the main scientific questions that have been addressed in your thesis.
It needs to summarize the adopted motivations and the adopted methodological approach as well as the findings of your work and their relevance and impact.
The Abstract is the part appearing in the record of your thesis inside POLITesi,
the Digital Archive of PhD and Master Theses (Laurea Magistrale) of Politecnico di Milano.
The Abstract will be followed by a list of four to six keywords.
Keywords are a tool to help indexers and search engines to find relevant documents.
To be relevant and effective, keywords must be chosen carefully.
They should represent the content of your work and be specific to your field or sub-field.
Keywords may be a single word or two to four words. 
\\
\\
\textbf{Keywords:} here, the keywords, of your thesis % Keywords

% ABSTRACT IN ITALIAN
\chapter*{Abstract in lingua italiana}
Qui va l'Abstract in lingua italiana della tesi seguito dalla lista di parole chiave.
\\
\\
\textbf{Parole chiave:} qui, vanno, le parole chiave, della tesi % Keywords (italian)

%----------------------------------------------------------------------------
%	LIST OF CONTENTS/FIGURES/TABLES/SYMBOLS
%----------------------------------------------------------------------------

% TABLE OF CONTENTS
\thispagestyle{empty}
\tableofcontents % Table of contents 
\thispagestyle{empty}
\cleardoublepage

%-------------------------------------------------------------------------
%	THESIS MAIN TEXT
%-------------------------------------------------------------------------
% In the main text of your thesis you can write the chapters in two different ways:
%
%(1) As presented in this template you can write:
%    \chapter{Title of the chapter}
%    *body of the chapter*
%
%(2) You can write your chapter in a separated .tex file and then include it in the main file with the following command:
%    \chapter{Title of the chapter}
%    \input{chapter_file.tex}
%
% Especially for long thesis, we recommend you the second option.

\addtocontents{toc}{\vspace{2em}} % Add a gap in the Contents, for aesthetics
\mainmatter % Begin numeric (1,2,3...) page numbering

\chapter{Background and motivations}
\label{ch:background}
With the current escalation in popularity and pervasiveness of machine learning solutions applied to the realm of natural languages, there exists an increasingly relevant topic that is still being the subject of extensive studies by experts in this field: interpretabilty.
Interpretabilty in itself is far from being a novel concept, as from the early advent of machine learning there has always been the necessity of understanding the inner process by which an opaque model performs its inner computations, possibly even going as far as giving an algorithmic interpretation to it. 

With the introduction of deep learning, the need for interpretabilty spiked as models got increasingly more complex.
The entire paradigm of deep learning is centered around the removal of human control over the feature space, letting models figure out `what needs to be learned' to solve the problem at hand.
In particular, we are interested in a specific class of machine learning models that are currently considered state-of-the art in the Natural Language Processing (NLP): Large Language Models (or LLMs).

This works intends to provide novel perspectives over the internal structure of Large Language Models, this goal is achieved by performing exploratory analyses, through the creation of tools and conducting of experiments aimed at confirming and possibly improving existing hypotheses and observations in the interpretability field.
Particular focus will be placed on the observation and interpretation of the models' hidden states, which are their internal representations, conveying condensed information between various internal components and acting as the `internal language' of the Language Model.

In this introductory section we will analyze the background and context of both NLP and machine learning techniques that will be relevant for this work.
This serves the purpose of providing a common knowledge base as a starting point, briefly covering all the pertinent subjects.

\section{Machine Learning}

Machine Learning is commonly considered a subfield of Artificial Intelligence, and is characterized by the idea of training a machine to learn from past experience, without providing an explicit algorithm to execute.
The term `past experience' is often used loosely in this scenario: it is not necessarily tied with information directly perceived by an agent within a specific environment.
Instead, it commonly denotes the entirety of accumulated data, often referred to as the dataset, which may have been collected heterogeneously and asynchronously, and that is fed to the model in order to learn.

\begin{figure}[H]
    \centering
    \includegraphics[width=\textwidth]{background_ml-para_placeholder.pdf}
    \caption{\todo[red]{placeholder image: Machine Learning Paradigms.}}
    \label{fig:background_ml-paradigm}
\end{figure}

There exist three main paradigms of learning for machine learning:
\begin{itemize}
    \item \textbf{Supervised learning}: in supervised learning, the data that we feed to the model is labeled, meaning that each data point is associated to a specific label representing the ground truth.
The model receives a sample from the dataset as input, eliciting an output that fulfills the role of the model's prediction.
This prediction is then compared against the label corresponding to the input data point.
If there is a discrepancy between the two, the model's internals are algorithmically adjusted to minimize future suboptimal predictions.
Most machine learning techniques belong to this category, and it also will be our main focus for this work.
    \item \textbf{Unsupervised learning}: unsupervised learning detaches the labeled component from the dataset, and removes direct feedback from the model's training process.
In this scenario we let the model autonomously find patterns inside data.
This process usually implies some kind of summarization or condensation of information, however the common factor is given by the search of hidden structure inside data.
Clustering techniques are a fairly common and well-known instance of unsupervised learning. 
    \item \textbf{Reinforcement learning}: reinforcement learning mostly concerns the decision-making process of an agent placed inside a dynamic environment.
The inputs of a reinforcement learning algorithm are represented by the environment state and rewards, while the agent's actions are treated as output.
In this paradigm we are trying to discover the optimal policy to navigate the environment, balancing both the exploration of new states, and the exploitation of the current maximally rewarding actions.
\end{itemize}

\subsection{Neural Networks}

Neural networks are one of the most popular machine learning techniques.
They are heavily inspired by the structure of the human brain, which is generally considered to be constituted by multiple neurons connected by synapses.
Classical neural networks usually have a layer-based architecture, where a layer consists of an array of neurons connected to neurons belonging to both the previous and next layers.
A neural network may have any number of intermediate layers, an input layer (which collects information at its inputs) and an output layer (which aggregates the outputs of its neurons into the final network prediction).

\begin{figure}[H]
    \centering
    \subfloat[Neural Network Architecture.\label{fig:background_nn-arch}]{
        \includegraphics[width=0.46\textwidth]{background_nn-arch_placeholder.pdf}
    }
    \quad
    \subfloat[Neuron.\label{fig:background_neuron}]{
        \includegraphics[width=0.46\textwidth]{background_neuron_placeholder.pdf}
    }
    \caption{\todo[red]{placeholder image: Neural Network Architecture + Neuron.}}
    \label{fig:background_nn-arch_neuron}
\end{figure}

Neurons are generally mono-directional and perform a linear combination of their inputs, each one scaled by the weight associated to its synapse.
Before transmitting their output, they also apply an activation function which in most cases is a `S-shaped' nonlinear function.
This is mainly chosen in order to prevent the network from just degenerating into a single linear transformation due to the linearity properties.
Thus, nonlinearities directly affect the expressive power of neural networks and are a vital component to guarantee their success, even on complex problems that require elaborate decision boundaries.

The central process used by neural networks to learn new information is backpropagation.
It works by following the inverse flow of information and updating the network's parameters (represented by the synapses' weights) according to the gradient value accumulated inside each neuron.
The gradient is said to be `accumulated inside neurons' for computational reasons, however it is always computed starting from the loss function at the output of the network and with respect to the parameters of the network.

\subsection{Deep Learning}

Deep Learning is a further specialization of machine learning.
The `deep' adjective refers to the high number of hidden layers that compose the intermediate section of deep neural networks.

The most important property obtained through the dimensional scale-up is the hierarchical feature extraction capability, which consists in obtaining rich condensed representations that follow a semantic hierarchy from the latent spaces defined by the network's layer projections.

\begin{figure}[H]
    \centering
    \subfloat[Classic Machine Learning.\label{fig:background_fex}]{
        \includegraphics[width=0.46\textwidth]{background_fex_placeholder.pdf}
    }
    \quad
    \subfloat[Deep Learning.\label{fig:background_deep-fex}]{
        \includegraphics[width=0.46\textwidth]{background_deep-fex_placeholder.pdf}
    }
    \caption{\todo[red]{placeholder image: Deep Learning Hierarchical Feature Extraction.}}
    \label{fig:background_fex_deep-fex}
\end{figure}

This is not a property that is exclusive to deep architectures, even classic neural networks present the same hierarchical feature patterns.
However, the sheer scale of deep neural networks makes the exploitation of these condensed representations a viable option to build an artificial feature base that has many upsides with respect to human-defined features.
For instance, we may require expert knowledge in order to craft an appropriate dataset containing meaningful features, while this would be done automatically in a deep learning scenario.

The main drawback of this approach comes from the fact that learned features are not directly human-understandable, as they result from mathematical optimization.
Consequently, the process of explaining a model's prediction is not as straightforward as with classical machine learning techniques.

The latent feature space concept and its properties are of particular relevance for this work, as most of the experiments that will be performed are aimed at attempting to understand them in the context of state-of-the-art models in the Natural Language Processing field of study.

\section{Natural Language Processing}

Natural Language Processing (NLP) is a subfield of computer science that stems off the branches of artificial intelligence, information engineering and computational linguistics.
Its main purpose is to interface computers to languages that are commonly used between humans for the purpose of communication.
This entails a great variety of many different subtasks, from text classification and machine translation to text generation itself.
The main focus of most of the state-of-the art models is text generation, even though it is possible to repurpose them for many other tasks due to their modular nature.

One of the first promising approaches to text generation was the Recurrent Neural Network (RNN) architecture, which featured a structure much more similar to classic neural networks, but with the addition of recurrent connections.
Recurrent connections enable RNN models to establish a distributed hidden state that acts as a memory, much like a flip-flop gate array in electronics.
In practice, each RNN module is fed the inputs at the current time instant and its own outputs at the previous time instant, making it suitable to process sequences and other time-based data.

\begin{figure}[H]
    \centering
    \subfloat[Recurrent Neural Networks Architecture.\label{fig:background_rnn}]{
        \includegraphics[width=0.46\textwidth]{background_rnn_placeholder.pdf}
    }
    \quad
    \subfloat[LSTMs.\label{fig:background_lstm}]{
        \includegraphics[width=0.46\textwidth]{background_lstm_placeholder.pdf}
    }
    \caption{\todo[red]{placeholder image: Recurrent Neural Networks Architectures + LSTMs.}}
    \label{fig:background_rnn_lstm}
\end{figure}

The most popular architecture for text generation before the current state-of-the-art models was the Long Short-Term Memory (LSTM).
LSTMs are a direct improvement on the RNN, solving the vanishing gradient problem, which was the main failure point of classic RNNs.
The \todo{overwhelming} vanishing gradient in RNNs is a direct consequence of the co-occurrence between activation functions without a prominent zero-valued region and the Backpropagation Through Time process (being the backpropagation equivalent of a network using recurrent connections), resulting in drastically small gradient values which can limit the model's context window and make the training process slow and unreliable.
LSTMs solve this by applying specialized gating functions that let the model decide what information to remember, forget and pass along into what is called the Constant Error Carousel.

\subsection{Word Embeddings}

One of the most relevant and groundbreaking developments in the field of NLP was the introduction of word embeddings.
Word embeddings came as a straightforward application of deep learning concepts and approaches to the world of NLP.
Their introduction not only determined an improvement in most text-based applications, but also motivated a shift in text representation techniques, incentivizing a word-centric approach rather than focusing on documents and defining words as frequencies inside them.

\begin{figure}[H]
    \centering
    \includegraphics[width=0.6\textwidth]{background_embeddings_placeholder.pdf}
    \caption{\todo[red]{placeholder image: Word Embedding Space Visualization.}}
    \label{fig:background_embeddings}
\end{figure}

The main appeal of word embeddings is the dimensionality reduction effect, since the original vocabulary representation (often referred as Bag of Words) suffers from the curse of dimensionality to a great extent.
The `curse of dimensionality' is a common way to refer to feature spaces that present numerous dimensions in conjunction with high data sparsity.
In a feature space affected by the curse of dimensionality, as the vocabulary space is, we can observe the fact that some common similarity measures and distances between data points (such as the Euclidean distance) almost lose meaning due to the majority of dimensions suppressing comparisons along the few dimensions that actually matter.
Word embeddings are not able to completely solve the curse of dimensionality as they still are relatively high-dimensional.
However, the improvement of adopting a dense feature space, with the choice of a normalized similarity function (such as cosine similarity) was still substantial enough to revolutionize the NLP field.

In fact, another enticing property is their ability to capture the semantic depth and context information inside the representation of each word.
Word embeddings are generally created by extrapolating the intermediate latent representations of a neural network solving a certain task, obtaining a continuous vector space for words to reside.
In practice, word embeddings are mappings from the vocabulary space to the embedding space which features a reduced number of dense dimensions devoid of any human-understandable meaning, characteristic of deep learning methodologies.

The concept of using distributed word representations was anticipated by \citet{bengio2020} with the Neural Network Language Model (NNLM) in the early 2000s.
The first real successful implementation of word embeddings was Word2Vec by \citet{mikolov2013}, greatly inspired by Bengio's previous work.
The main purpose of Word2Vec is to create word embeddings, it is composed by a single hidden layer and is trained discriminatively by using both positive and negative examples.

\blockquote[J.R. Firth]{You Shall Know a Word by the Company It Keeps}

\begin{figure}[H]
    \centering
    \subfloat[CBoW.\label{fig:background_word2vec-cbow}]{
        \includegraphics[width=0.46\textwidth]{background_word2vec-cbow_placeholder.pdf}
    }
    \quad
    \subfloat[Skip-gram.\label{fig:background_word2vec-skipg}]{
        \includegraphics[width=0.46\textwidth]{background_word2vec-skipg_placeholder.pdf}
    }
    \caption{\todo[red]{placeholder image: Skip-gram and CBoW Word2Vec.}}
    \label{fig:background_word2vec-cbow_word2vec-skipg}
\end{figure}

Mikolov's original work introduces two primary training variants for Word2Vec:
\begin{itemize}
    \item \textbf{Continuous Bag of Words (CBoW)}: predict a target word given its context.
The context is given as a symmetric window containing all terms occurring around the target term, modeling a many-to-1 prediction.
    \item \textbf{Skip-gram}: predicts surrounding words given a target word.
Single context words are predicted one at a time, modeling a 1-to-1 prediction.
\end{itemize}
The results are generally comparable between the two techniques.
The key aspect is that the full context of each word is integrated into the word representation itself.

\subsubsection*{Word Embeddings Semantic Properties}

One particular property that emerges from the majority of latent representations is their flexibility to arithmetic manipulations.
It is not uncommon to observe these properties showcased in many other deep learning applications, such as computer vision.
However, in the word embeddings scenario, these arithmetic properties have been reported to be much more prominent and \todo{reflecting} of the actual word semantics.

In particular, the semantic meaningfulness of certain arithmetic and geometric transformations applied to word embeddings was noticed by Mikolov himself in the Word2Vec paper \cite{mikolov2013}.
This went from finding similarities in embeddings of words that manifested syntactic regularities, to complete double-word analogies located in common semantic fields.
One of the most famous examples presented in \cite{mikolov2013} is: \texttt{emb(`King') - emb(`Man') + emb(`Woman') $\simeq$ emb(`Queen')}, which models both the relationships of `gender' and `royalty'.
Other \todo{relevant} analogies include country-capital, family relation, and verb tenses.
However, any kind of semantic relationship could be potentially modeled by word embeddings, given enough support in the training dataset.

\begin{figure}[H]
    \centering
    \includegraphics[width=\textwidth]{background_emb-arithmetic_placeholder.pdf}
    \caption{\todo[red]{placeholder image: Embedding Arithmetic https://developers.google.com/machine-learning/crash-course/embeddings/embedding-space?hl=it.}}
    \label{fig:background_emb-arithmetic}
\end{figure}

While these properties have some direct use cases principally involving clustering and similarity comparisons, their study has been proven to be insightful from the explainability standpoint.
Possibly understanding the semantic associations that exist \todo{at the base} of text models can provide an educated perspective on how they interpret their inputs, but also underline human and language biases.
Additionally, this topic is of particular relevance due to the focus on interpretability of this work.

\section{Transformer Architectures}

Transformers are the current state-of-the art model architecture for most text applications and not only.
Although they were initially conceived for textual sequential inputs, in recent years saw discrete success in a wide variety of other applications (in particular image processing), albeit occasionally with some slight variations.
This architecture has proven to be flexible, efficient and parallelizable, in contrast with previous popular choices for sequence learning problems such as RNNs and LSTMs.

The first transformer architecture was introduced by \citet{vaswani2017} in the seminal 2017 paper ``Attention Is All You Need''.
This architecture is of the encoder-decoder type, following the typical paradigm of sequence-to-sequence (seq2seq) learning.
This structure implies the existence of two separate modules (an encoder and a decoder), which work in tandem to elaborate inputs and provide an output.
Both input and output consist of sequential data.

\begin{figure}[H]
    \centering
    \includegraphics[width=0.6\textwidth]{background_trans-enc-dec_placeholder.pdf}
    \caption{\todo[red]{placeholder image: Attention is All You Need Transformer Architecture https://arxiv.org/abs/1706.03762.}}
    \label{fig:background_trans-enc-dec}
\end{figure}

The purpose of the encoder module is to process the inputs and create an intermediate representation that is read and processed by the decoder, which outputs the next token.
The encoder and the decoder have a fairly similar structure, both featuring multiple basic blocks that can be sequentially connected to each other (operation commonly referred to as `stacking').

\subsection{Structure}

\subsubsection*{Preprocessing}

The first step to process inputs is tokenization, as it is fundamental to reduce the possibly infinite vocabulary to a finite set of tokens that can be embedded and understood by the model.
This is normally done by a tokenizer, which performs some preprocessing on the input sentence to obtain a sequence of tokens that will be fed to the model.
Most transformer architectures use different variants of sub-word tokenization algorithms, meaning that each word may correspond to a combination of multiple tokens.
This type of tokenization is usually obtained as the result of a training process where common character sequences get progressively aggregated until they reach a certain frequency threshold, at this point a compact and efficient representation of the vocabulary that is balanced between granularity and size is achieved.
Sub-word tokenization improves the ability of the model to deal with rare and unknown words, greatly limiting the use of unknown tokens (special tokens used in place of certain words that cannot be correctly tokenized and encoded) and potentially facilitating the creation of multilingual models.

The tokenization and embedding steps are perhaps the most critical processes due to the fact that they shape both inputs and outputs of the transformer, framing how it interfaces itself with the environment.
Additionally, the transformer architecture is word position agnostic, meaning that embedded tokens do not directly contain any additional information that \todo{tells} the model their position inside the input sentence.
This is commonly solved by implementing \textbf{positional encoding}: an additive mask that gets applied to the input tokens right after the embedding step.
The original ``Attention Is All You Need'' \cite{vaswani2017} paper suggests the use of fixed positional encodings based on the \emph{sine} and \emph{cosine} functions, although alternatives exist.

\subsubsection*{Basic Block Structure}

Transformer blocks \todo{compose} large part of the transformer architecture and contain \textbf{attention blocks}, \textbf{feed-forward blocks}, \textbf{residual connections} and \textbf{normalization blocks}.

\textbf{Attention blocks} are \todo{powered} by the attention mechanism, the \todo{principal mechanic} responsible for the outstanding results achieved by transformer models.
The concept of attention is not novel and did already see some applications in seq2seq models \todo[orange]{cite https://arxiv.org/pdf/1409.0473, https://arxiv.org/pdf/1702.00887, https://arxiv.org/pdf/1507.01053}, determining substantial improvements in most LSTM architectures.
Attention in encoder-decoder models is implemented by allowing the decoder to \todo{look} at the internal states generated by the encoder, thus eliminating the information bottleneck represented by the single point of connection between the two.
The most important part of attention is its scoring function, which allows the decoder to `focus' on certain tokens by modeling a weight distribution on the attended hidden states using an attention function.
There exist different attention function implementations such as \emph{simple dot product}, \emph{Luong (multiplicative)} \todo[orange]{cite} and \emph{Bahdanau (additive)} \todo[orange]{cite}.
The original transformer architecture mentioned in ``Attention Is All You Need'' \cite{vaswani2017} uses \emph{scaled dot product attention}, which is implemented with the following formula:
\begin{equation}
    \label{eq:background_attention}
    Attention(Q,K,V) = softmax\left(\frac{QK^\T}{\sqrt{d_K}}\right)V
\end{equation}
Where $Q$, $K$ and $V$ represent the \emph{query}, \emph{key} and \emph{value} components of attention which are obtained by multiplying the hidden state vectors to their respective projection matrices $W_Q$, $W_K$ and $W_V$, which are composed of learnable model parameters.

\begin{figure}[H]
    \centering
    \subfloat[Encoder-Decoder.\label{fig:background_enc-dec-attention}]{
        \includegraphics[width=0.8\textwidth]{background_enc-dec-attention_placeholder.pdf}
    }
    \quad
    \subfloat[Self.\label{fig:background_self-attention}]{
        \includegraphics[width=0.8\textwidth]{background_self-attention_placeholder.pdf}
    }
    \caption{\todo[red]{placeholder image: Attention Blocks (Encoder-Decoder / Self).}}
    \label{fig:background_enc-dec-attention_self-attention}
\end{figure}

The transformer architecture implements three different types of attention:
\begin{itemize}
    \item An \textbf{encoder-decoder attention} (or \textbf{cross-attention}), which is found inside the decoder block, and attends to the encoder blocks' hidden states using its current state as the \emph{query}.
    \item The \textbf{self-attention}, which is used by both the encoder block to process the hidden states at its input.
It works in following the same principles as cross-attention however, it attends to representations of the same nature, meaning that \emph{query}, \emph{key} and \emph{value} are extrapolated by the same set of hidden states.
In practice, the model is able to manage long-range dependencies and possibly capture patterns or associations between representations in a completely parallelizable way.
    \item Lastly, \textbf{masked self-attention} is a specific implementation of self-attention that can be found in the decoder block, due to its need of autoregressive modeling.
In fact, the decoder is not allowed to use information about future words, since it is capable of returning a single output token per stack execution.
Masked self-attention differs from self-attention in scenarios where the full output is available (e.g.\ training), and it works by adding a mask with the effect of nullifying the contribution of future tokens in the attention computation.
The masked self-attention formula can be summarized as follows:
\begin{equation}
    \label{eq:background_masked-attention}
    \begin{gathered}
        MaskedAttention(Q,K,V) = softmax\left(\frac{QK^\T + M}{\sqrt{d_K}}\right)V \\
        M_{ij} = \begin{cases}
            0 & \text{if}\ i \ge j, \\
            -\infty & \text{otherwise}.
        \end{cases}
    \end{gathered}
\end{equation}
\todo[cyan]{check M definition equation}
\end{itemize}

In reality, transformers use a more refined attention implementation, called \textbf{multi-head attention} which can be applied in all the previously identified scenarios.
Multi-head attention is a novelty introduced in the ``Attention Is All You Need'' \cite{vaswani2017} paper, and it consists in replicating the attention structure multiple times, thus obtaining multiple representation spaces for each attention layer.
This operation influences the dimension of attention vectors, which needs to be divided by $h$, where $h$ is the chosen number of separate attention heads for multi-head attention.
The attention output for multi-head attention is obtained as the concatenation between the attention outputs of all $h$ heads as follows:
\begin{equation}
    \label{eq:background_multihead-attention}
    \begin{aligned}
    MultiHead(Q,K,V)    &= head_1\ \oplus\ \ldots\ \oplus\ head_h \\
                        &= Attention_1(Q,K,V)\ \oplus\ \ldots\ \oplus\ Attention_h(Q,K,V)
    \end{aligned}
\end{equation}
In practice, multi-head attention enables the model to simultaneously focus on information present at different positions, avoiding excessive attention on single tokens.

\todo[green]{attention dimensional analysis}

The second main \todo{block} contained in the transformer block is the \textbf{feed-forward block}.
As the name suggests, it consists of a simple feed-forward network and is normally situated as the last step of the transformer block, using the outputs of the previous attention layers as inputs.
The intermediate representations are processed by the feed-forward layer flow in a completely independent manner, therefore the execution of this block is parallelizable by nature.
Feed-forward blocks' principal contribution to the transformer architecture consists in the introduction of nonlinearities, since attention is still a predominantly linear process.
The feed-forward network acts as a `grouping' mechanism, where information gathered in previous steps through attention is aggregated, processed and reformulated, adding depth to the computation.

\textbf{Layer normalization} and \textbf{residual connections} are techniques that are commonly found in most deep learning models and are generally used to improve gradient flow, training stability and overall generalization.
In particular:
\begin{itemize}
    \item \textbf{Layer normalization} \todo[orange]{cite https://arxiv.org/pdf/1607.06450} is applied after each attention and feed-forward block.
It consists of two steps: a batch-independent normalization step performed with parameters ($\mu$, $\sigma$) which model a normal distribution over the feature space of the vector representations, and a linear transformation using a bias and a scale factor.
The bias and scale factor are usually referred to as $\beta$ and $\gamma$ respectively, and are trainable layer-level parameters which perform a shift and rescale operation on the normalized vector.
If $v_i$ is a vector representation, after layer normalization at layer $\ell$ we obtain:
\begin{subequations}
    \begin{gather}
        \bar v_{i,\ell} = \gamma_\ell \cdot \left( \frac{v_i - \mu}{\sigma} \right) + \beta_\ell, \label{eq:background_layernorm} \\
        \text{where}\ \mu = \frac{1}{n}\sum_{i=1}^{n}{v_i},\ \sigma = \sqrt{\frac{1}{n}\sum_{i=1}^{n}{{(v_i - \mu)}^2}} \label{eq:background_layernorm_extra}
    \end{gather}
\end{subequations}
\todo[cyan]{Check equation pedices}
    \item \textbf{Residual connections} \todo[orange]{cite https://arxiv.org/pdf/1512.03385} are a strategy employed in deep architectures to streamline gradient propagation and avoid vanishing gradients.
Additionally, it helps the transformer to preserve local information as attention and feed-forward blocks are added to the base residual flow, limiting their freedom in the effective output space.
In practice, residual connections sum the identity function of the input of a block to the output of that block, in the transformer's case the residual summation happens before layer normalization.
\end{itemize}

\subsubsection*{Postprocessing}

Of particular importance is the last step in the transformer generative process.
After the inputs have been embedded and have been passed through all the decoder's transformer blocks, iteratively refining them into new vector representations, the decoder has to output a new token.
Given the fact that a transformer model generates a single token at a time, we would presumably obtain a single hidden representation at the end of the decoder stack with a size compatible to the embedding space.
However, in order to output a vocabulary token we need the vector to lay in the vocabulary space.
To achieve this goal, the transformer architecture features a last linear layer outside of the stack with the purpose of bringing the last hidden representation from the embedding space to the vocabulary space, modeling an `unembedding' operation.
In some specific model architectures, this last weight matrix is forced to be equal to the transpose of the initial embedding matrix for the sake of retaining a consistent representation between the first and last layers; other architectures opt to learn a different embedding representation space for the outputs of the model.
\todo[green]{this is known as "weight tying"...}
After the `unembedding' operation, the resulting vector of floating point numbers (commonly referred to as logits) will be laying in the vocabulary space.
However, in order to get an actual probability distribution over the vocabulary for the next token of the input sequence, a last softmax application is needed.

\subsection{Current Decoder Architectures}

What was previously described is generally considered the first transformer architecture that was originally mentioned in the ``Attention Is All You Need'' \cite{vaswani2017} paper in 2017.
However, with time, a large number of variations and improvements on the base architecture has struck the NLP field.
We are going to be mainly concerned with the Llama 2 \todo[orange]{cite} and Llama 3 \todo[orange]{cite} (from here just Llama) models, developed by Meta AI \todo{reference} in 2023 and 2024 respectively; other minor models may also be taken into consideration, although without particular focus on their architecture.

\begin{figure}[H]
    \centering
    \includegraphics[width=0.6\textwidth]{background_llama-arch_placeholder.pdf}
    \caption{\todo[red]{placeholder image: Decoder-only Architecture Llama https://github.com/hkproj/pytorch-llama-notes.}}
    \label{fig:background_llama-arch}
\end{figure}

The first main novelty in the new state-of-the-art models is the complete removal of the encoder from the architecture.
These models are often called \textbf{`decoder-only'}, \textbf{`autoregressive'}, \textbf{`causal'} \todo[green]{differentiate between causal/autoregressive?}or \textbf{`GPT-like'} from the Generative Pre-trained Transformer (GPT) architecture \todo[orange]{cite https://openai.com/index/language-unsupervised/} that popularized the use of transformer models without the decoder, although the first documented use of a `decoder-only' architecture can be ascribed to this 2018 paper by Liu et al.\todo[orange]{cite https://arxiv.org/pdf/1801.10198} 
The major improvement of autoregressive architectures consists in a substantial performance enhancement for longer sequences of text \todo[orange]{cite}.
In addition, removing the encoder also removes possible information redundancy between the two components.
The decoder remains mostly unchanged from this modification, with the exception of the cross-attention block since it can no longer fetch the representations generated by the encoder.
Cross-attention is consequently removed, \todo{leaving} a single masked self-attention block \todo{inside} the decoder architecture.

Another important change in the transformer architecture is the \textbf{Rotary Position Embeddings (RoPE)} mechanism, proposed in 2021 by Su et al. \todo[orange]{https://arxiv.org/pdf/2104.09864}
This technique completely replaces the positional encoding strategy that was previously employed in transformer models (absolute positional encoding) by guaranteeing better flexibility for long sequences and introducing the decay of inter-token dependency with increasing relative token distance.
One of the main flaws of absolute positional encoding is its inability to model dependence between tokens, as each token positional encoding is independent of the others and, in particular, from the distance between tokens.
RoPE is able to model both absolute and relative token positional information by encoding them as a geometric rotation inside the embedding space.
This rotation operation, which has a multiplicative nature (opposed to the additive one of absolute positional encoding), has the additional advantages of avoiding any change in vector norm for the subjected hidden representation.
This property is vital in determining RoPE's capability of being directly applied inside the query and key matrices in linear self-attention, improving overall efficiency.

The original transformer implementation establishes the presence of layer normalization right after each attention and feed-forward blocks, after residual addition.
Extensive research has been performed on possible alternatives and a notable shift from models implementing classic normalization (post-norm) to models implementing pre-normalization emerged.
The \textbf{pre-normalization (pre-norm)} \todo[orange]{cite https://arxiv.org/pdf/1809.10853} \todo[orange]{cite https://aclanthology.org/P18-1167.pdf} \todo[orange]{cite https://github.com/tensorflow/tensor2tensor?tab=readme-ov-file} approach consists in applying layer normalization directly inside the residual block, right before the attention or feed-forward blocks.
Additionally, a further layer normalization operation is performed before prediction, right after the last transformer block.
The main benefits of this technique are tied to training efficiency and convergence speed \todo[orange]{cite https://arxiv.org/pdf/2002.04745 (pre-norm advantages)} as it was shown to improve overall gradient stability, even at initialization time.
This improvement allowed the removal of the `learning rate warm-up stage': a technique that was commonly used in post-norm transformer architectures to avoid diverging in training \todo[orange]{cite https://arxiv.org/pdf/1804.00247} and involved a gradual increase of the learning rate in the first training epochs \todo[orange]{cite https://github.com/tensorflow/tensor2tensor?tab=readme-ov-file}.

Another variation on the transformer normalization mechanism was achieved with the introduction of \textbf{root mean square layer normalization (RMSNorm)} \todo[orange]{cite https://arxiv.org/pdf/1910.07467}.
The fundamental improvement introduced by RMSNorm is a substantial reduction in the amount of computation performed with respect to classical layer normalization and an overall gain in efficiency, this is especially true for deep architectures where the computational overhead of layer normalization is meaningful.
Implementation wise, RMSNorm only focuses on the rescaling aspect rather than including both centering and rescaling as layer normalization does.
The central quantity used to perform the rescaling computation is the Root Mean Square (RMS) statistic and only the $\gamma$ parameter is retained from the set of learnable parameters due to the fact that the re-centering operation performed by $\beta$ was deemed to be mostly irrelevant for both layer normalization and RMSNorm.
If $v_i$ is a vector representation, after RMSNorm at layer $\ell$ we obtain:
\begin{equation}
    \label{eq:background_rmsnorm}
    \begin{aligned}
        \bar v_{i,\ell} = \gamma_\ell \cdot \frac{v_i}{RMS(v)}, &&
        \text{where}\ RMS(v) = \sqrt{\frac{1}{n}\sum_{i=1}^{n}{v_i^2}}
    \end{aligned}
\end{equation}
One peculiar side effect of RMSNorm is the fact it forces the summed inputs into a $\sqrt{n}$-scaled unit sphere which benefits the stability of layer activations and output distribution, positively influencing the interpretability of the hidden space.

The last main change featured in most recent autoregressive transformer architectures concerns the structure of the feed-forward block.
The base transformer implementation establishes the feed-forward layer as the combination of two linear transformations separated by a non-linear activation function, typically a choice between \emph{ReLU} and \emph{GELU}.
However, newer architectures have begun to incorporate Gated Linear Units (GLUs) \todo[orange]{cite https://arxiv.org/pdf/1612.08083} implemented using a \emph{Swish} (also called \emph{SiLU}) activation function, this particular architectural combination for the feed-forward block is commonly referred to as \textbf{SwiGLU} \todo[orange]{cite https://arxiv.org/pdf/2002.05202v1}.
\emph{Swish} \todo[orange]{cite https://arxiv.org/pdf/2002.05202v1} is an activation function similar to \emph{ReLU}, characterized by a non-monotonic depression in its zero-region, and is implemented in the following way: $Swish(x) = x \cdot \sigma(x)$ where $\sigma(x)$ represents the logistic sigmoid function.
\emph{Swish} was shown to offer marginal improvements over \emph{ReLU} and other similar variations, based on its smoothness and possibility of returning small negative values for inputs close to zero, implying the efficient convergence of non-zero gradients and consequently minimizing the problem of dead neurons (caused by the nullification of gradient).

\begin{figure}[H]
    \centering
    \includegraphics[width=0.6\textwidth]{background_ffnn-glu_placeholder.pdf}
    \caption{\todo[red]{placeholder image: GLU Feedforward Architecture.}}
    \label{fig:background_ffnn-glu}
\end{figure}

On the other hand, GLUs determine an entire revision of the feed-forward architecture, as they introduce a gating mechanism where one linear transformation is modulated by the output of another, which allows the model to dynamically control the flow of information.
In the case of Llama models, the gating mechanism happens through an element-wise multiplication where $W_{in} \in \mathbb{R}^{d_{model} \times d_{hid}}$ represents the input transformation matrix with its respective bias $b_{in}$, $V \in \mathbb{R}^{d_{model} \times d_{hid}}$ is the gate transformation matrix with its respective bias $b_{V}$, $W_{out} \in \mathbb{R}^{d_{hid} \times d_{model}}$ performs the out-projection to the model's hidden representation dimensionality with its respective bias $b_{out}$ and $f$ represents the activation/gating function (in the SwiGLU case $f(x) = Swish(x)$):
\begin{equation}
    \label{eq:background_ffnn}
    FFN(x) = GLU(x) \cdot W_{out} + b_{out} 
    = \Bigl( f(xV + b_V) \odot xW_{in} + b_{in} \Bigr) \cdot W_{out} + b_{out}
\end{equation}
\todo[cyan]{check dimensionality}
For Llama models, all biases of the transformation matrices referring to the SwiGLU implementation are set to zero ($b_{in} = b_V = b_{out} = \vec 0$).
The benefits of SwiGLU are largely tied to providing a better downstream performance on fine-tuning tasks and pre-training objectives, despite proof of SwiGLU actually having an impactful effect on transformer models being purely empirical.
However, the reasons for its effectiveness may potentially be associated with its capability to model complex functions through the gating meschanism. \todo[orange]{cite https://arxiv.org/pdf/2002.05202v1 https://kikaben.com/swiglu-2020/}

\subsection{The Large Language Model Paradigm}

With the advent of Large Language Models (LLMs), a complete shift in training paradigms was observed throughout the NLP field since, due to their massive scale, the need for applying ad-hoc Deep Learning training techniques \todo{arose}.
In particular, we can point out two main approaches commonly using for streamlining, improving and optimizing the training process: \textbf{pre-training} and \textbf{fine-tuning}.

\textbf{pre-training} is often performed in an unsupervised or \emph{`self-supervised'} manner, meaning that either the training dataset is used without labels or labels can be directly inferred from it.
Pre-training fulfills the function of providing the model with general language patterns, without specific orientation or direction.
For this reason, large unbiased corpuses of text are needed to perform a successful pre-training, often resulting in lengthy and expensive training runs.
There are two principal methods to perform pre-training: \emph{Masked Language Modeling (MLM)} and \emph{Next Sentence Prediction (NSP)}.
\emph{MLM} is popular between encoder architectures such as the BERT family, and consists of trying to infer certain `masked' tokens given the surrounding ones by leveraging the provided context.
Whereas, \emph{NSP} sees more use in the pre-training of GPT-like and decoder-based models by giving the model a pair of sentences and asking it to determine if the second sentence can logically follow the first one.

On the other hand, \textbf{fine-tuning} is usually faster and performed after pre-training, having the goal of specializing the model to carry out a specific task.
By freezing part of the model architecture (usually the first layers), fine-tuning can be performed on a subset of the model's parameters resulting in a faster and less disruptive training, as one of the main concerns correlated with fine-tuning is the model's possibility of forgetting what was learned during pre-training.

\chapter{Related works}
\label{ch:related_works}
The field of transformer interpretability has garnered significant attention over the past decade, resulting in a substantial and continually expanding body of literature.
Particular attention must be given to the fact that this is a relatively novel field of research, continuously evolving due to the numerous ongoing contributions at the present time.
Consequently, it is very possible that some results provided in this work may be obsoleted or invalidated by more recent works.
This section reviews the key contributions and developments in this area, highlighting the foundational studies and recent advancements that are pertinent to the present research.

In such a fast-paced and prolific field, it is nearly impossible to consider every relevant contribution, making it inevitable that some material may be overlooked.
To address this, we propose a cutoff date that standardizes a fixed knowledge base for our research.
However, this cutoff may be disregarded in cases of recent, exceptional contributions that have the potential to significantly impact or reshape the current research landscape, as such works merit acknowledgment.
Additionally, if a work was initially published before the cutoff date but subsequently re-published in a different journal, it will also be exempted from the cutoff.
The chosen cutoff date is July 2024, reflecting the start of the writing period.

\subsubsection*{Mechanistic interpretability}

\citet{rai2024} propose a taxonomy for interpretability techniques centered around the concept of Mechanistic interpretability (MI).
It is possible to identify two main fundamental objects of study in this context: \textbf{features} and \textbf{circuits}.
Features can be considered to be properties that are represented by the model and are mainly characterized by being human-interpretable, whereas circuits can be thought as the connections between features or, more generally, model components.
These objects of study serve as starting points for interpretability inquiries, while specific techniques act as tools to explore and verify those inquiries.
By leveraging MI tools to pursue interpretability questions, possibly through the use of evaluation techniques, we obtain findings: true generalizable statements about the model's inner workings.

Mechanistic interpretability offers a novel perspective over the interpretability research field, its primary aim being the reverse-engineering of language models (LMs) from an in-depth perspective~\cite{olah2022}.
Previously identified model-agnostic techniques have been proven to offer limited insight for the transformer architecture~\cite{neely2022,pruthi2022,bibal2022,krishna2024}.
In contrast, MI embraces the opposite philosophy by eliminating model abstractions and analyzing LMs in terms of their specific components and interactions.

Mechanistic interpretability was initially mentioned as being the main driving ideology behind the `transformer circuits thread'~\cite{elhage2021}.
Nonetheless, by following~\citet{rai2024} approach, we can observe that the concept of MI is not limited to the application of circuits.
Envisioning MI as being characterized by a general bottom-up approach for interpreting LMs, allows its scope to extend to include earlier techniques such as the logit lens~\cite{nostalgebraist2020} and other probing methods.

\subsubsection*{Overview on transformer interpretability}

Another possible taxonomy for interpretability techniques, more focused on their nature rather than their use, is presented by~\citet{ferrando2024}.
They identify two main classes of interpretability approaches: \textbf{behavior localization} and \textbf{information decoding}.
In the next sections we will follow their insightful classification to provide a synthetic analysis of the state of the art, with a specific focus on a restricted number of techniques that are especially relevant for the purpose of this work.

\begin{figure}[t!]
    \centering
    \includegraphics[width=0.9\textwidth]{related_ferrando-tax.pdf}
    \caption{Taxonomy for transformer interpretability methods proposed by~\citet{ferrando2024}.}
    \label{fig:related_ferrando-tax}
\end{figure}

\section{Behavior localization}

Behavior localization techniques consist in the localization elements inside language models that are responsible for specific predictions or certain prediction patterns.
It is a generally broad task, but an important distinction can be made between the localization of behaviors towards input features (\textbf{input attribution}) and towards model components (\textbf{model component attribution})~\cite{ferrando2024}. 

\subsection{Input attribution}

In the \textbf{input attribution} case, the model's predictions are directly traced back to the inputs via some kind of attribution mechanism.
The two main input attribution strategies are either based on gradients~\cite{denil2014, ding2021, sanyal2021} or on perturbations~\cite{li2016, amara2024, mohebbi2023}.
In both cases the great majority of techniques was directly influenced by model-agnostic approaches~\cite{sundararajan2017, smilkov2017, ribeiro2016, lundberg2017} that were initially studied and applied in the context of deep learning.

More recent input attribution techniques experimented with the aggregation of intermediate information, especially attention, to provide token-wise attributions exploiting context mixing properties of transformers~\cite{ferrando2022, modarressi2022, mohebbi2023}.
Whereas, other approaches focused on providing counterfactual explanations based on contrastive gradient attributions~\cite{yin2022} or studying specific training examples to understand and generalize their influence on model predictions~\cite{grosse2023}.
It is important to note that, through the years, some critiques have been moved towards input attribution methods, mainly concerning their limited reliability~\cite{sixt2019, adebayo2018, atanasova2020}.

\subsection{Model component attribution}

In \textbf{model component attribution}, the main research focus shifts towards analyzing the effects of individual or groups of transformer components, such as attention heads, feedforward layers, and neurons.
This shift is principally motivated by the inherent sparsity of LMs, where only a subset of the model's parameters significantly contributes to its predictions~\cite{zhao2021}.
By isolating and understanding the effects of these key components, it is possible to shed light on their contribution to the actual model's prediction.
\citet{ferrando2024} identifies three main distinct approaches for model component attribution: \textbf{logit attribution}, \textbf{causal interventions} and \textbf{circuit analysis}.

\subsubsection*{Logit attribution}

\textbf{Logit attribution} is based upon the concept of direct logit attribution (DLA), a metric specifically devised to measure the contribution of a certain component $c$ to the logit of the output token $w$ exploiting the inherent linearity of the transformer model's components.
Some variation on this idea enabled the computation of the logit attribution metric in more specialized cases.
For example:~\citet{geva2022} managed to measure the DLA of each FFN neuron,~\citet{ferrando2023} identified an alternative to measure the DLA of each path involving a certain attention head, and~\citet{wang2023} proposed the direct logit difference attribution (DLDA) using the logit difference (LD) as a comparative mean to measure contrastive attribution.

\subsubsection*{Causal interventions}

\textbf{Causal interventions} approaches are centered around the interpretation of the LM as a causal model~\cite{geiger2021,mcgrath2023}, which takes the form of a directed acyclic graph (DAG) having model computations as nodes and activations as edges.
The primary purpose of this representation is to enable specific interventions (known as activation patching or causal tracing) directly on the model's components, allowing for comparisons of different computational outcomes.

Two key choices which influence the result of causal intervention, besides the choice of which component to patch, are the choice of patching function and evaluation metric.
Different authors have suggested a variety of possible patched activation functions that accomplish different goals and have different uses.
There have been cases of null vectors being used as patched activations (zero intervention)~\cite{olsson2022, mohebbi2023}, noise being added to the input of the component (noise intervention)~\cite{meng2022} and counterfactual data being fed to the component either by sampling (resample intervention)~\cite{hanna2023, conmy2023} or averaging (mean intervention)~\cite{wang2023}.
\citet{zhang2024} provide an insightful overview for common practices of activation patching in language models, identifying KL divergence, probability and logit difference as common evaluation metrics.
Additionally, it is possible to identify an alternate `denoising' setup, which subverts the classic activation patching operation by applying a patched activation from a clean run to a corrupted one~\cite{lieberum2023, meng2022}.

In particular, \citet{meng2022} were able to trace causal effects of hidden state activations within GPT architectures using causal mediation analysis to identify modules that perform the recall of a fact about a certain subject.
Causal mediation analysis quantifies the contribution of intermediate variables in causal graphs, in this scenario the grid of hidden states affected by attention and FFNNs forms a causal graph.
The main findings of \citet{meng2022} include the localization of factual associations inside the parameters of FFNN modules at an early site of last subject token and at a late site for the last sequence token.
Additionally, by exploiting the previous findings, they also introduce Rank-One Model Editing (ROME) with the purpose of altering the parameters that determine a feedforward layer's behavior at the decisive token.
ROME makes it possible to modify factual associations by inserting a new knowledge tuple in place of a current tuple with both generalization and specificity.

\subsubsection{Circuit analysis}

\textbf{Circuit analysis} is closely related to the mechanistic interpretability (MI) subfield analyzed previously as its main goal is tied to the discovery of circuits inside LMs.
Circuits are subsets of model components that can be seen as acting independently while carrying out a specific task, and can possibly be synthesized into an algorithm.
Despite their successful application on LMs, circuits were not originally identified with the transformer architecture in mind; in fact, their first application was on vision models~\cite{cammarata2020}.
Most of the initial work regarding transformer circuits was performed on publications belonging to the `transformer circuits thread'~\cite{elhage2021,olsson2022,elhage2022,bricken2023}, heavily inspired by the preceding vision counterpart~\cite{cammarata2020}.

In particular, `A Mathematical Framework for Transformer Circuits' \cite{elhage2021} can be considered the seminal work that popularized circuit identification in transformer models.
In this work, \citet{elhage2021} perform a reverse-engineering analysis on simplified versions of transformer models, leading to the discovery of relevant properties.
For once, they entirely deconstruct a small attention-only model by keeping the embedding and unembedding layers, while progressively incorporating attention layers.
The results obtained from this process highlight the tendency of the zero layer transformer to model bigram statistics, while models with added attention layers can interpret more expressive patterns.
Another important contribution of this work is the proposal of an alternate deconstruction of the multi-head attention formula, which highlights the operations carried out by single heads.
On this topic, \citet{elhage2021} find that even the composition between attention heads holds meaningful expressiveness inside the transformer architecture and by generalizing this concept, it is possible to represent an attention-only transformer as a sum of end-to-end functions by exploiting the inherent linear structure of this simplified architecture.

By applying the circuit concept to the previously causal intervention techniques, we can extend further the idea of activation patching to edge patching and path patching: novel circuits-based techniques that take into account the interactions between model components.
Edge patching~\cite{li2023} considers edges that directly connect pairs of model components due to the fact that each component input can be modeled as the sum of the outputs of the previous model components inside the residual stream, while path patching~\cite{wang2023} is a generalization of edge patching to multiple edges.

\section{Information decoding}

\textbf{Information decoding} takes a step back from behavior localization techniques by focusing on the extraction of single pieces of information from model components, rather than trying to explain entire predictions by attributing them to various internal mechanisms.

These pieces of information take the name of features (or concepts) and are commonly characterized by being human interpretable properties of the input~\cite{kim2018}.
The three main categories that can be identified in this approach consist of \textbf{probing} which can be seen as the LM adaptation of a popular technique in deep learning, a broader categorization named sparse autoencoders that includes the application of sparse autoencoders following the \textbf{linear representation hypothesis}, and \textbf{vocabulary space decoding} which tackles the representation of models' representations using vocabulary tokens.

\subsection{Probing}

\textbf{Probing} techniques are used to analyze the inner workings of LMs and, more generally, any kind of deep neural network.
Probing usually implies the supervised training of ad-hoc models (often classifiers) to interpret the features present inside intermediate representations of the main model.
The probing classifier is designed to evaluate how much information about a particular property is encoded within an intermediate representation.
While the probe should seek out information about the chosen property directly from hidden representations, concerns have been raised regarding the limitations of probing classifiers~\cite{belinkov2022} due to the probes' tendency to collapse toward modeling the task itself, rather than extrapolating information.

Particular attention has been put towards probing transformer models~\cite{chwang2024, macdiarmid2024, burns2023}, especially the family of encoder-only models related to BERT~\cite{devlin2019}.
Some exceptional results include the discovery of syntactic information inside the hidden representations of BERT models~\cite{tenney2019a, lin2019}, even to the extent of uncovering entire syntax trees~\cite{hewitt2019} and hierarchical computation structures along the residual stream, reminiscent of classical NLP pipelines~\cite{tenney2019b}.

\subsection{Linear representation hypothesis}

The \textbf{linear representation hypothesis}~\cite{park2023} posits that high-level concepts are represented linearly within the representation space of a model.
The central idea for this hypothesis builds upon early findings of linearity inside the embedding space by~\citet{mikolov2013}, leading to the resolution of analogies and the presence of geometric properties as the direct consequences of a linear embedding space.
Recent studies have uncovered numerous instances of FFN neurons that consistently fire with  patterns linked to specific input features~\cite{voita2024}, suggesting that this behavior is an effect of the next token prediction training paradigm~\cite{jiang2024}.

Moreover, numerous attempts aimed at modifying the internal representations of models have been made by leveraging their linear properties.
These linear interventions have been proven successful in erasing specific concepts and features from intermediate model representations~\cite{ravfogel2020, ravfogel2022, belrose2023b}, as well as in meaningfully altering the model's behavior~\cite{nanda2023, belrose2023b}, opening up new avenues for model steering and alignment. 

Another important aspect of the linear representation hypothesis is the presence of polysemanticity and superposition in the identified features.
The effects of dimensionality reduction algorithms causing information compression and resulting in distributed representations has widely been observed and studied in many fields, however \citet{olah2023} makes an important distinction between the separate phenomena of composition and superposition.
Many extend these observations to actual experiments, successfully proving the existence of superposition both in simplified scenarios~\cite{elhage2022} and in the early layers of transformer-based LMs~\cite{gurnee2023}.

\citet{timkey2021} find that, when relying on cosine distance as a metric, representation spaces are often dominated by only 1 to 5 dimensions which drive anisotropy, low self-similarity, and the apparent drop in representational quality that happens later layers.
Consequently, they advise the use of simple post-processing techniques in order to standardize the representation space and make linear relationship more evident, enhancing similarity-based techniques.

Interestingly, \citet{wendler2024} apply linear properties of representation spaces to examine the phenomenon of `English pivoting', wherein an LLM translates tokens into English to conduct computations, even when prompted in a non-English language.
While the authors find no evidence of English fulfilling the role of an internal pivot directly, by analyzing latent embeddings as high-dimensional euclidean points, they observe that the middle layers of the transformer operate in an abstract concept space which is partially orthogonal to the language-specific concept space reached in the final layers.
Consequently, the illusion of English pivoting can be pointed to an English bias in the abstract concept space, rather than a direct translation.

\subsection{Vocabulary space decoding}\label{ssec:related_vocab}

One of the most direct methods to comprehend a model's hidden representations is by employing its own vocabulary to derive plausible interpretations. \textbf{Vocabulary space decoding} techniques are founded on this principle, by utilizing the model's existing vocabulary they can generate outputs that are immediately understandable and may unveil hidden patterns inside the model's generation process.

The first real implementation of vocabulary space decoding is logit lens~\cite{nostalgebraist2020}, which proposed the decoding of interlayer hidden representation using the model's own unembedding matrix following the intuition of an iterative refining of the model's prediction throughout the forward pass~\cite{jastrzebski2018}.
The contribution of logit lens was groundbreaking and, despite some acknowledged shortcomings by the author, inspired numerous similar techniques aimed at improving its design or offering alternative functionalities.
Some significant advancements include the introduction of translators, which act as probing classifiers to enhance logit lens' predictions by applying either linear mappings~\cite{din2024} or affine transformations~\cite{belrose2023a}.
Additionally, attention lens~\cite{sakarvadia2023} applies the concepts of logit lens and translators to the outputs of attention heads, while future lens~\cite{pal2023} extends logit lens predictions to also include the next most probable tokens by exploiting causal intervention methods.

Another crucial contribution, inspired by future lens, is the \emph{patchscopes} framework~\cite{ghandeharioun2024}, which aims to generalize all prior interpretability methods based on vocabulary space decoding and causal interventions.
Other significant approaches include the direct decoding of model weights~\cite{dar2023}, potentially using singular value decomposition techniques to factorize the weight matrices~\cite{millidge2022}, and logit spectroscopy~\cite{cancedda2024}, which employs a spectral analysis of the residual stream and parameter matrices interacting with it.
This last method aims to identify and analyze specific parts of the hidden representation spectrum that are most likely to be overlooked by the classic logit lens.

\begin{figure}[t!]
    \centering
    \includegraphics[width=0.8\textwidth]{related_patchscopes.pdf}
    \caption{\todo[red]{placeholder caption: Patchscopes Visualization.}}
    \label{fig:related_patchscopes}
\end{figure}

Other unrelated approaches based on vocabulary space decoding involve using maximally-activating inputs to explain the behavior of units and neurons that exhibit significant responses to specific features~\cite{dalvi2019}.
Additionally, other LMs have been used as zero-shot explainers to provide insights into possible shared features between input sequences that cause substantial activations of specific neurons in the target model~\cite{bills2023}.
Unfortunately, the maximally-activating input analysis has been criticized for generating false positives~\cite{bolukbasi2021}, while the elicitation of natural language explanations from LMs approach has faced criticism for its general lack of causal influence between the identified concept-neuron pairs~\cite{huang2023}.

\emph{LM Transparency Tool} (LM-TT)~\cite{tufanov2024} is an exceptional toolkit that offers interactive tools for analyzing the internal workings of transformer models.
LM-TT builds on top of the circuital interpretation of the transformer to visualize the information flow.
As it is possible to observe from \cref{fig:related_lm-tt}, LM-TT tool focuses on the visualization of the most relevant attention paths leading to the production of an embedding in the internal states of the transformer.
Additionally, it also provides useful information that can aid the interpretation of intermediate representations at varying degrees of granularity.

\begin{figure}[t!]
    \centering
    \includegraphics[width=0.9\textwidth]{related_lm-tt.pdf}
    \caption{\todo[red]{placeholder caption: Visualization of That One Paper (LM-TT).}}
    \label{fig:related_lm-tt}
\end{figure}

Surprisingly, LM-TT bares a lot of similarities with our proposed tool, InTraVisTo (\cref{sec:rq_intravisto}), as their main interpretative goal can be considered the same.
However, there are also some fundamental differences in how these tools carry out their intended purpose, and the approaches they take to visualize information.
For example, InTraVisTo includes a unique visualization of the residual flow using a Sankey diagram, which comes with the advantage of providing the user with a complete overview of the influence of all tokens in a single screen.
Additionally, the evolution of intermediate representations is quantified using the KL divergence measure between intermediate steps, and by applying vocabulary decoding techniques on their difference.
Lastly, InTraVisTo offers an injection and ablation framework to enable interactive interventions in real time, allowing users to directly affect and observe the model computation.
These distinctive features offered by InTraVisTo will be examined more in greater detail in \cref{sec:rq_intravisto,sec:method_intravisto,sec:exp_intravisto}.

\chapter{Research questions}
\label{ch:reseatch_questions}
In this section, we delineate the specific research questions regarding transformer interpretability that will be addressed in the present work.
The purpose of these research questions is to provide an overall informative analysis on the hidden representations and embedding space of recent transformer-based models.
The novelty of this work is not necessarily \todo{inherent in the} chosen approach, but resides in the combination of multiple known techniques to provide novel perspectives and interpretability tools with the ultimate purpose of confirming and possibly expanding upon previous findings.

\section{Interactive Visualization Tool for LLMs}

The first research question is centered around the development of an interactive visualization tool for LLMs.
The goal is to create a tool that allows users to visually explore and interpret the whole computation performed by an LLM when generating a sentence.
The chosen approach will be centered around the concept of vocabulary space decoding and will take the direction of the already mentioned logit lens \todo[orange]{cite}.
Additionally, new ideas will be introduced in the context of hidden representation decoding and visualization.
One of \todo{the central purposes} of the proposed tool is the visualization of the residual stream, making it possible to observe the path of influence composed of residuals and attention, that cumulates into each token prediction.

The creation of this transformer visualization framework is justified by an overall revisitation of the logit lens analysis from a new, \todo{multifaceted} perspective with the ultimate goal of confirming established results from a new angle and possibly observing new phenomena.
This framework serves as a starting point for many of the inquiries that will be tackled in this work, as it provided the means of preliminary analysis needed to \todo{kickstart} other, more \todo{targeted}, experiments.

\section{Do autoregressive LLMs retain linear properties in their embedding spaces?}

With the aid of the previously developed tool we noticed some peculiarities in the behavior of LLMs with respect to smaller and less \todo{sophisticated} language models, in particular, the embedding representations of LLMs seemed to encode less information about the actual word semantics.
From this starting point, the second research question investigates whether autoregressive LLMs retain linear properties in their embedding spaces.

As mentioned in the previous chapters, the linearity of the embedding space is an established property both in the original word2vec embedding space \cite{mikolov2013} and in transformer-based encoder models such as BERT \todo[orange]{cite}.
A linear embedding space enables particular geometric properties such as the ability to solve word analogies (`king' - `man' + `woman' = `queen') and the possibility of modeling semantic similarity between tokens as a distance relationship.
Given the recent developments in transformer training and architecture, we aim to examine if these properties still hold in the embedding spaces of recent causal LLMs and whether \todo{these properties} are consistent across various model architectures and sizes, while pointing out differences between the embedding and unembedding spaces in models \todo{where the two are different}.

\section{Is it possible to obtain first-order predictions by concatenating embedding spaces in LLMs?} \label{sec:rq_3}

Following the investigation on different embedding weights between input and output spaces of a LM, the third research question explores the feasibility of obtaining first-order predictions by concatenating the two embedding spaces in LLMs.
A first-order prediction can be seen as a guess of the next token based solely on the previous one, similar to a first-order Markov model's predictive process, which employs a conditioned probability on the previous token to obtain the most likely next token.
As observed by \todo[orange]{cite}, indeed concatenating the input and output embeddings of a LM should theoretically yield the equivalent of a first-order Markov model when we concatenate the input and output embeddings of a LM, however, is that always the case?
We will test this hypothesis against new state-of-the-art models and provide novel insights into the role of different embedding spaces in recent LLMs.
Moreover, this series of experiments should hopefully provide further understanding \todo{on} the practice of \emph{weight tying} \todo[orange]{cite} performed on the embedding layers of language models.

\chapter{Methodology}
\label{ch:methodology}
In this section we will explore the theoretical fundaments of our experiments.
To do so, we are going to subdivide our analysis into three main sections, where each one is paired with one of the previously identified research questions.

\section{Transformer Visualization}

The first section is dedicated to the exploration of the main features pertaining the proposed interactive tool for the exploration of autoregressive transformer architectures, \emph{InTraVisTo} (Inside Transformer Visualization Tool).

\todo[purple!20]{Start of paper section}

\emph{InTraVisTo} is an open-source visualization tool depicting the internal computations performed within a Transformer.
The tool provides visualizations of both the internal state of the LLM, using a heatmap of decoded embedding vectors for all layer/token positions, and the information flow between components of the LLM, using a Sankey diagram to depict paths through which information accumulates to produce next-token predictions.

\subsection{Decoding Internal States}\label{ssec:method_intravisto_decoding}

In this context, with the expression \emph{`decoding process'} we refer to the \emph{vocabulary decoding} process, which consists in the conversion of transformers hidden states, represented by vectors in a high-dimensional space, into human-readable content.
InTraVisTo enables the decoding and inspection of the main four vectors produced by each layer $\ell$ of a transformer:
\begin{itemize}
    \item $\gbm{\delta}_\textit{att}^{(\ell)}$ represents the output of the attention component.
    \item ${\gbm{x}'}^{(\ell)}$ is the \emph{intermediate state}, given by the addition of $\gbm{\delta}_\textit{att}^{(\ell)}$ to the residual stream.
    \item $\gbm{\delta}_\textit{ff}^{(\ell)}$ represents the output of the feedforward network component.
    \item $\gbm{x}^{(\ell)}$ is the \emph{layer output}, which can be seen as the residual stream with the contributions of both $\gbm{\delta}_\textit{ff}^{(\ell)}$ and $\gbm{\delta}_\textit{att}^{(\ell)}$.
\end{itemize}

Decoding the meaning of hidden state vectors at various depths of a transformer stack is essential for providing an intuition as to how the model is working.
We pose our focus on causal models, and assume the state-of-the-art LLM architecture, which involves a continuous \emph{decoration process} where each transformer layer adds the results of its computations to a residual embedding vector from the layer below.
InTraVisTo provides a human-interpretable representation of this internal decoration pipeline by decoding each hidden state with a specific decoder and displaying the most likely token from the model's vocabulary.

\subsubsection{Decoders and Normalization}

As for the decoding process, we compute the probability distribution over the vocabulary space for a hidden state $\gbm{x}$ in the following way:
\begin{equation}
    \label{eq:method_intravisto_decoding}
    P(\ \cdot \mid \gbm{x}, d_\textit{dec}, n_\textit{norm}) = \mathrm{softmax}\left(N_{n_\textit{norm}}(\gbm{x}) \cdot \gbm{W}_{d_\textit{dec}}\right)
\end{equation}
Where $\gbm{W}_{d_\textit{dec}}$ represents the matrix of decoder weights according to the user decoder choice $d_\textit{dec}$, and $N_{n_\textit{norm}}$ is used to identify the normalization operation selected by the user through $n_\textit{norm}$:
\begin{equation}
    N_{n_\textit{norm}}(\gbm{x}) = 
    \label{eq:method_intravisto_normalization}
    \left\{
    \begin{array}{cl}
        \gbm{x} &\ \text{if}\ n_\textit{norm} = \text{`no normalization'} \\
        \mathcal{N}{(\gbm{x})} &\ \text{if}\ n_\textit{norm} = \text{`normalize only'} \\
        \gbm{\gamma}_\ell \cdot \mathcal{N}{(\gbm{x})} + \gbm{\beta}_\ell &\ \text{if}\ n_\textit{norm} = \text{`normalize and scale'}
    \end{array}
    \right.
\end{equation}
\todo[cyan]{Fix notation}
Considering $\mathcal{N}{(\gbm{x})}$ the normalization component of the model's final normalization layer, being reliant on $\mu$ and $\sigma$ for LayerNorm implementations as defined in \Cref{eq:background_layernorm,eq:background_layernorm_extra}, and $RMS(\gbm{x})$ for RMSNorm implementations as defined in \Cref{eq:background_rmsnorm}

Two natural choices of decoders to use are the transpose of the \emph{input embedding matrix} $\gbm{W}_\textit{in}^\T$ used by the model to convert tokens to vectors on input, and the \emph{output decoder} $\gbm{W}_\textit{out}$ used upon output within the language modeling head.
Some models, like GPT-2 \todo[orange]{cite} and Gemma \todo[orange]{cite} tie these two parameter matrices together during training, while other popular models such as Mistral \todo[orange]{cite} and Llama \todo[orange]{cite} allow these two matrices to differ.
This structural weight difference will be analyzed more in depth in later sections, \todo{however,} in our current scope having different weight matrices for embedding and unembedding ($\gbm{W}_\textit{in}^\T \neq \gbm{W}_\textit{out}$) implies that earlier layers tend to be much more interpretable when decoded with the input embedding ($\gbm{W}_\textit{in}^\T$) while latter layers are more meaningful if the output decoder ($\gbm{W}_\textit{out}$) is used.

Previous work has looked to \emph{train specialized decoders} \todo[orange]{cite} for generating meaningful vocabulary distributions at any point in a model, at the cost of introducing a great deal of additional complexity and potential errors.
InTraVisTo employs a simpler and elegant alternative, by \emph{interpolating} the input and output decoders based on the depth $\ell\in\{0,\ldots,L\}$ of the model layer we wish to decode, we obtain a `hybrid' decoding weight matrix that acts as an equilibrium point calibrated on the current model depth.
We define various alternatives for decoder interpolation mainly focusing on \emph{linear interpolation} and \emph{quadratic interpolation}, defined as follows:
\begin{subequations}
    \begin{align}
        \gbm{W}_\textit{linear}^{(\ell)} =\left(1-\frac{\ell}{L}\right) \cdot \gbm{W}_\textit{in}^\T + \frac{\ell}{L} \cdot \gbm{W}_\textit{out} \label{eq:method_intravisto_linear-interp} \\
        \gbm{W}_\textit{quadratic}^{(\ell)} =\frac{{(L - \ell)}^2 \cdot \gbm{W}_\textit{in}^\T + \ell^2 \cdot \gbm{W}_\textit{out}}{L^2 - 2 \ell L + 2\ell^2} \label{eq:method_intravisto_quadratic-interp}
    \end{align}
\end{subequations}
\todo[cyan]{check again quadratic interpolation}
\todo[cyan]{It is important to note that the term `quadratic interpolation' comprises a small abuse of terminology, as it refers to a specific class of possible quadratic interpolations between the two decoders.
More specifically, the contributions given by the joint }

Any of the matrices $\gbm{W}_\textit{in}$, $\gbm{W}_\textit{out}$, $\gbm{W}_\textit{linear}$ and $\gbm{W}_\textit{quadratic}$ can be used as the decoder matrix $\gbm{W}_{d_\textit{dec}}$ to decode an embedding $\gbm{x}$ into a probability distribution over $\mathcal{V}$ as described in \Cref{eq:method_intravisto_decoding}.
As previously mentioned, this behavior selection is controlled by specifying $d_\textit{dec}$.

\subsubsection{Secondary Tokens}
    
As previously illustrated, each hidden representation is decoded by performing a normalization operation first, to then multiply the result for the chosen decoding matrix, and finally obtain a probability distribution over the model's vocabulary by applying a softmax function on the resulting logits.
At this point we utilize the same \emph{sampling policy} of the model in order to extract the `predicted' vocabulary token ID from the computed distribution.
We assume the policy to always be \emph{greedy decoding}, both as a simplifying factor and to reflect the default settings for causal models provided by Hugging Face's transformers library \todo[orange]{cite}.

Nevertheless, our interest in decoding the model's hypothetical intermediate predictions doesn't extend only to the first token, as important information that cannot be condensed into a single vocabulary token is usually withheld inside the hidden representation \todo[orange]{cite}.
In order to extract this `leftover information', we devise two main approaches that \todo{sublimate} it into \emph{secondary tokens}.
We define \emph{secondary tokens} as additional vocabulary tokens that are the result of a \emph{secondary decoding process}, aimed at obtaining tokens that hold less importance than the token obtained as a result of the \emph{primary decoding process} (as mentioned before) of the same hidden representation.

The first secondary decoding approach is \emph{Top-K probability decoding} and consists of expanding the number of selected tokens from the probability distribution to $k$, thus obtaining $k-1$ secondary tokens ordered by their probability values.
This is a rather simple and immediate technique, which is widely-used in many interpretability applications and non \todo[orange]{cite}.
The principal downside of this approach comes from the fact that the obtained secondary tokens might be overly similar to the primary token, resulting in redundant information.
This is likely caused by the fact that a large part of the hidden representation is used to represent the primary token, often skewing the embedding vector in favor of tokens that are semantically similar \todo[orange]{cite}.

To alleviate this issue we propose a novel way to extract secondary representations from a hidden representation: \emph{iterative decoding}.
The rationale behind the proposed approach is that hidden representations contain an overlap of concepts in an embedding space that is loosely related to both the input and output embedding spaces.
As a consequence, we postulate that hidden representations existing in these intermediate embedding spaces should retain the linear properties that have been ascertained to exist in the input and output embedding spaces of transformer architectures \todo[orange]{cite}.
Iterative decoding exploits these linear properties by performing a sequence of subtractions from the main hidden representation, removing the embedding of the most probable representation during each iteration.
\begin{algorithm}
    \caption{Iterative decoding algorithm}\label{alg:method_intravisto_iter-dec}
    \begin{algorithmic}
        \STATE{$tokens \gets \{\}$}
        \STATE{$norms \gets \{\}$}
        \STATE{$i \gets 0$}
        \WHILE{$i < {rep}_{max}$}
            \STATE{$id \gets \arg \max\{decode(emb)\}$}
            \STATE{${emb}_{real} \gets {(\gbm{W}_{d_{dec}})}_{id,\cdot}$}
            \IF{$\|emb\| \leq {norm}_{min} \vee \bigl( |norms| > 0 \wedge \|emb\| \geq norms\bigl[i-1\bigr] \wedge i \neq 0 \bigr)$}
                \STATE{\textbf{break}}
            \ENDIF{}
            \IF{$id \notin tokens$}
                \STATE{$tokens\bigl[|tokens|-1\bigr] \gets id$}
            \ENDIF{}
            \STATE{$norms\bigl[i\bigr] \gets \|emb\|$}
            \STATE{$emb \gets emb - {emb}_{real}$}
        \ENDWHILE{}
        \RETURN tokens
    \end{algorithmic}
\end{algorithm}
\todo[cyan]{check algorithm}
As it is possible to observe in \Cref{alg:method_intravisto_iter-dec}, we perform at most ${rep}_{max}$ iterations obtaining one primary token and between $0$ to ${rep}_{max} - 1$ secondary tokens.
This is due to the fact that a secondary token that is found in two separate iterations is recorded only on the first one, and the presence of a stopping condition that triggers in case the norm of the resulting hidden representation is under a certain threshold ${norm}_{min}$ or is higher than the norm found at the previous iteration.
This last condition is used to avoid situations where the embedding vector of the hidden state `flips' after the computation of the difference with the most probable representation, resulting in a vector that is not informative and can possibly reiterate the effect until ${rep}_{max}$ is reached, thus generating predictions based purely on noise. \todo[cyan]{check if necessary}

One downside of this approach, besides the linearity assumption of the intermediate embedding spaces \todo{which is based upon}, is the fact that there exists a notable shift in representation magnitudes throughout the layers of most state-of-the-art transformer models \todo[orange]{cite}.
This typically results in a steady increase in the norms of the embedding vectors, proportional to the layer number.
While this has only a marginal impact on the decoding process, it can significantly disrupt the embedding vector subtraction operation within the iterative decoding approach, as the quantities involved may differ in magnitude.

\subsubsection{Decoding Metrics}

Other meaningful quantities that are shown through the InTraVisTo visualization are tied to the actual probability distributions obtained through the decoding process.
The first option is \emph{``P(argmax term)''}, which directly translates into the probability of the most probable token output from the chosen decoder.
It gives an immediate idea of how much the model is sure about the token that has been greedily sampled.
On the other hand, a complementary measure is the \emph{entropy} of the probability distribution over the vocabulary space, which the higher it is, the more unsure the model is of the next token.
Entropy for an embedding $\gbm{x}$ is computed in the following way:
\begin{equation}
    \label{eq:method_intravisto_entropy}
    H_{\gbm{x}} = -\sum{P_{\gbm{x}} \cdot \log{P_{\gbm{x}}}}
\end{equation}
\todo[cyan]{check}
Where $P_{\gbm{x}} = P(\ \cdot \mid \gbm{x}, d_\textit{dec}, n_\textit{norm})$ as computed in \Cref{eq:method_intravisto_decoding}.

Other showcased metrics are the \emph{attention contribution} and \emph{feedforward contribution}, which measure respectively how much the output of the attention block, or feed forward, contributes in its summation with the residual stream of a transformer block.
The purpose of these metrics is to highlight where the main information of that block is coming from, whether from the attention or feed forward components.
We devised two main approaches to weigh the contribution of each component to the residual stream: one uses the \emph{norms} of hidden state vectors to compare the magnitude of their contribution, while the other uses the \emph{KL divergence} to compare the probability distribution similarity of the two hidden states against the final one.
In practice, we compute:
\begin{equation}
    \left\{
    \begin{aligned}
        &{\%}^{(\ell)}_{\textit{norm},\textit{att}} = \frac{\|\gbm{\delta}_\textit{att}^{(\ell)}\|_2}{\|\gbm{\delta}_\textit{att}^{(\ell)}\|_2 + \|\gbm{x}^{(\ell-1)}\|_2} \\
        &{\%}^{(\ell)}_{\textit{norm},\textit{ff}} = \frac{\|\gbm{\delta}_\textit{ff}^{(\ell)}\|_2}{\|\gbm{\delta}_\textit{ff}^{(\ell)}\|_2 + \|{\gbm{x}'}^{(\ell)}\|_2} \label{eq:method_intravisto_norm-contrib}
    \end{aligned}
    \right.
\end{equation}
\begin{equation}
    \left\{
    \begin{aligned}
        &{\%}^{(\ell)}_{\textit{KL},\textit{att}} = \frac{D_{\text{KL}}(\gbm{x}^{(\ell-1)} \parallel {\gbm{x}'}^{(\ell)})}{D_{\text{KL}}(\gbm{\delta}_\textit{att}^{(\ell)} \parallel {\gbm{x}'}^{(\ell)}) + D_{\text{KL}}(\gbm{x}^{(\ell-1)} \parallel {\gbm{x}'}^{(\ell)})} \\
        &{\%}^{(\ell)}_{\textit{KL},\textit{ff}} = \frac{D_{\text{KL}}({\gbm{x}'}^{(\ell)} \parallel \gbm{x}^{(\ell)})}{D_{\text{KL}}(\gbm{\delta}_\textit{ff}^{(\ell)} \parallel \gbm{x}^{(\ell)}) + D_{\text{KL}}({\gbm{x}'}^{(\ell)} \parallel \gbm{x}^{(\ell)})} \label{eq:method_intravisto_kl-contrib}
    \end{aligned}
    \right.
\end{equation}
Where $D_{\text{KL}}$ is the Kullback-Liebler divergence between two distributions, and the notation of hidden states (such as $\gbm{x}^{(\ell)}$, $\gbm{\delta}_\textit{att}^{(\ell)}$, \ldots) references the distinctions made at \Cref{ssec:method_intravisto_decoding}.
It is possible to note that the contributions computed through norms and KL divergence feature opposite terms at the fractional numerator, this is due to their inverse relationship as the KL divergence measures the \emph{dissimilarity} between probability distributions, while the norm can be directly translated into the positive contribution of a hidden state. 

\subsection{Flow}

The second visualization introduced in InTraVisTo is a Sankey diagram \todo[orange]{cite} that aims to depict the information flow through the transformer network.
Edges in the diagram indicate the amount of influence that the nodes have on each other and show how the information accumulates from the bottom of the diagram to the top in order to generate the final prediction.
The flow snakes its way through self-attention nodes, which combine information from attended tokens in the level below, feed-forward networks nodes, which introduce information based on detected patterns in the state vector, and aggregation nodes, where updates from the other two types of nodes are added to the residual vector.

In order to calculate the information flow, an attribution algorithm works backwards from the top layers of the network, recursively apportioning the incident flow from the components below based on their relative contributions to the internal state vector above using \Cref{eq:method_intravisto_norm-contrib,eq:method_intravisto_kl-contrib}.
The flow's constantly updating state can be defined in a way that reflects the recursive computations performed by InTraVisTo:
\begin{equation}
    \left\{
    \begin{alignedat}{2}
        &\textit{flow}_{\textit{ffnn}}^{(\ell,j)} &&= {\%}_{\textit{ffnn}}^{(\ell,j)} \cdot \textit{flow}_{x}^{(\ell,j)} \\
        &\textit{flow}_{x'}^{(\ell,j)} &&= \textit{flow}_{\textit{ffnn}}^{(\ell,j)} + (1 - {\%}_{\textit{ffnn}}^{(\ell,j)}) \cdot \textit{flow}_{\textit{x}}^{(\ell,j)} = \textit{flow}_{\textit{x}}^{(\ell,j)} \\
        &\textit{flow}_{\textit{att}}^{(\ell,j)} &&= {\%}_{\textit{att}}^{(\ell,j)} \cdot \textit{flow}_{x'}^{(\ell,j)} = {\%}_{\textit{att}}^{(\ell,j)} \cdot \textit{flow}_{x}^{(\ell,j)} \\
        &\textit{flow}_{x}^{(\ell-1,j)} &&= \sum_{i\in\{j,\ldots,k\}}{\overline{\textit{attend}{\,}}^{(\ell,i)}\bigl[j\bigr]}\cdot\textit{flow}_{\textit{att}}^{(\ell,i)} + ( 1 - {\%}_{\textit{att}}^{(\ell,j)})\cdot \textit{flow}_{\textit{x'}}^{(\ell,j)} \\
            &\quad &&= \biggl(\Bigl(\sum_{i\in\{j,\ldots,k\}}\overline{\textit{attend}{\,}}^{(\ell,i)}\bigl[j\bigr] - 1\Bigr)\cdot{\%}_{\textit{att}}^{(\ell,j)} + 1\biggr) \cdot \textit{flow}_{\textit{x}}^{(\ell,j)}
    \end{alignedat}
    \right.
\end{equation}
Where $\overline{\textit{attend}{\,}}^{(\ell,i)}$ denotes the average attention placed on token $j$ by the attention heads present at position $i$ of layer $\ell$.
This quantity is used to compute all outgoing contributions of a token $j$ to subsequent self-attention nodes, thus it is computed considering tokens between $j$ and $k$ as the attention sources, where $k$ denotes the index of the last token in the generated sentence.

Another type of information shown in the Sankey diagram concerns the computation of differences between residual representations, with the goal of visualizing the flow's `evolution' throughout the model's layers.
To achieve this objective we devised two main approaches that we implemented to explore this \todo{visualization dimension}.
The first approach consists \todo{of} computing the KL divergence between the probability distributions of hidden states belonging to each combination of component outputs and the residual stream state.
In practice, we compute the following quantity for five different state combinations:
\begin{equation}
    \label{eq:method_intravisto_kl-diff}
    \begin{gathered}
        {\textit{kl\_diff}{\,}\strut}_{\gbm{x}_a, \gbm{x}_b}^{(\ell)} = D_{\text{KL}}(P_{\gbm{x}_b} \parallel P_{\gbm{x}_a}) \\
        \text{for} \ (\gbm{x}_a, \gbm{x}_b) \in \Bigl\{
            (\gbm{x}^{(\ell-1)}, {\gbm{x}'}^{(\ell)}), 
            (\gbm{\delta}_\textit{att}^{(\ell)}, {\gbm{x}'}^{(\ell)}), 
            ({\gbm{x}'}^{(\ell)}, \gbm{\delta}_\textit{ffnn}^{(\ell)}), 
            ({\gbm{x}'}^{(\ell)}, \gbm{x}^{(\ell)}), 
            (\gbm{\delta}_\textit{ffnn}^{(\ell)}, \gbm{x}^{(\ell)})
        \Bigr\}
    \end{gathered}
\end{equation}
\todo[cyan]{check KL order}
Where $P_{\gbm{x}} = P(\ \cdot \mid \gbm{x}, d_\textit{dec}, n_\textit{norm})$ as computed in \Cref{eq:method_intravisto_decoding}, $D_{\text{KL}}$ is the Kullback-Liebler divergence between two distributions, and the notation of hidden states (such as $\gbm{x}^{(\ell)}$, $\gbm{\delta}_\textit{att}^{(\ell)}$, \ldots) references the distinctions made at \Cref{ssec:method_intravisto_decoding}.
Whereas, the second approach compares the residual stream states at the input and output location for each layer in the transformer stack.
It does so by calculating the difference between the two hidden states, and performing the decoding operation defined in \Cref{eq:method_intravisto_decoding} as to generate primary and secondary tokens for the resulting quantity.
This is showcased in the following computation:
\begin{equation}
    \label{eq:method_intravisto_state-diff}
    {\textit{state\_diff}{\,}\strut}^{(\ell)} = P(\ \cdot \mid \gbm{x}^{(\ell)} - \gbm{x}^{(\ell - 1)}, d_\textit{dec}, n_\textit{norm})
\end{equation}
It is possible to notice how this last approach could suffer from the same weaknesses that have been identified in iterative decoding as shown in \Cref{alg:method_intravisto_iter-dec}, due to the presence of an operation (difference) which assumes that its operands exist in a shared embedding space with linear properties.
Although this is true, the minimal distance between the hidden states used to compute the difference makes the presented issue have minimal impact on the actual decoding result.

\subsection{Injection}\label{ssec:method_intravisto_injection}

Lastly, we also include the possibility to perform \emph{injections} in InTraVisTo.
\emph{Injections} are an instance of \emph{activation patching} \todo[orange]{cite} utilized in a context that is not strictly tied to a formal \emph{causal intervention} framework \todo[orange]{cite} .
In fact, the main purpose of injections is to give the user the possibility to explore the model predictions in an interactive way, making it possible to change outputs of components and parts of the residual stream in order to unveil interesting patterns and properties.

Being a type of intervention, it is possible to express an injection utilizing the \emph{do-operator} as defined by \todo[orange]{cite}.
Assuming that we would like to inject an embedding $\gbm{\hat x}$ into the hidden state obtained as a result of the transformation $f_c^{(\ell)}$ of a component $c^{(\ell)}$ during the $i$-th token of the model's forward pass expressed as $f$, we can notate our injection utilizing the do-operator as:
\begin{equation}
    \label{eq:method_intravisto_injection}
    f\Bigl(\gbm{x}_i \, \Big| \, \text{do}\bigl(f_c^{(\ell)}(\gbm{x}_i) = \gbm{\hat x}\bigr)\Bigr)
    \; \text{where}\ \gbm{\hat x} = \textit{encode}(\textit{id}_{\textit{inject}} \,|\, n_\textit{norm}, d_{dec})
\end{equation}
As previously noted for similar approaches, it is important to specify that there might be some problems with the nature of $\gbm{\hat x}$, as it is a representation that is generated from an encoded user-defined textual input ${id}_{\textit{inject}}$, which undergoes an embedding and normalization process utilizing $\gbm{W}_{d_{dec}}$ and $N_{n_\textit{norm}}$ following \Cref{eq:method_intravisto_linear-interp,eq:method_intravisto_quadratic-interp,eq:method_intravisto_normalization}.
This results in the injection of a `clean' embedding that is unexpected by the model, \todo{and thus} may destabilize the generation process.
In most causal intervention scenarios, this problem is not present when performing activation patching using counterfactual examples, due to the fact that patches are generated utilizing resampling or averaging approaches \todo[orange]{cite}.
We choose to avoid these techniques since our main focus is directed towards giving the user complete control over the injected information, which is achieved by directly inserting the wanted embedding vector in the hidden state without letting the model elaborate it to create a plausible representation.

Another noteworthy step in the injection process is the actual encoding of the injected embedding, since the formula used to compute $\gbm{\hat x}$ that has been broadly mentioned in \Cref{eq:method_intravisto_injection} is used considering an injection input that is composed by a single token.
However, InTraVisTo's interface offers the possibility of injecting an embedding representation that contains more than one token, thus we would like to generalize the aforementioned formula to actually handle multiple tokens at once.
\todo[cyan]{check if true} This behavior is modeled by averaging the embedding representations of all tokens that compose the sentence, always under the assumption of a linear embedding space \todo[orange]{cite}. \todo[green]{remarks for sentence embedding?}
Considering a total of $id_1,\ldots,id_t$ tokens to be encoded, this averaging operation can be represented in the following way:
\begin{equation}
    \label{eq:method_intravisto_emb-avg}
    \gbm{\hat x} = N_{n_\textit{norm}}\Bigl(\frac{1}{t} \sum_{i=1}^{t}{{(\gbm{W}_{d_{dec}})}_{\textit{id}_{i},\cdot}}\Bigr)
\end{equation}
The soundness of this approach, including possible variants based on the same or different assumptions, will be analyzed more in depth in the next section (\Cref{sec:method_embeddings}).


\section{Embedding Analysis}\label{sec:method_embeddings}

This second section is geared towards understanding the extent to which text embeddings still retain some degree of semantic factuality in LLMs (Large Language Models).
Early transformer models have been noted for their text embedding representations possessing interesting spatial properties deeply correlated with the semantic properties of the embedded words \todo[orange]{cite DBLP:journals/corr/abs-2009-11226,DBLP:conf/icml/AllenH19}.
However, with the emergence of newer models that are larger and more intricate than ever before, our objective is to assess whether the semantic properties inherently tied with the geometry of embeddings still hold true.

\subsection{Word Encoding}

The first step for performing experiments on embedding representations, is defining a set of operations that make the encoding and comparison of embedding vectors possible.
In the following section we will illustrate the theoretical basis of the preliminary steps that were used to set up the actual experiments.

\subsubsection{Distance Metrics}

In order to compare and find `close' embeddings, we need to formalize a similarity measure between two embeddings. 
To this end, we define two main distance metrics that will be used to compare embedding vectors in their space:
\begin{equation}
    d_{\textit{dist}}(\gbm{x}, \gbm{y}) = 
    \label{eq:method_embeddings_distance}
    \left\{
    \begin{array}{cl}
        \sqrt{\sum_{i=1}^{n}{{(x_i - y_i)}^2}} &\ \text{if}\ \textit{dist} = \text{`euclidean'} \\
        1 - \frac{\sum_{i=1}^{n}{x_i y_i}}{\sqrt{\sum_{i=1}^{n}{x_i^2}} \cdot \sqrt{\sum_{i=1}^{n}{y_i^2}}} &\ \text{if}\ \textit{dist} = \text{`cosine'}
    \end{array}
    \right.
\end{equation}
\todo[cyan]{check dimension}
The \emph{cosine similarity/distance} metric is the most immediate and widely used approach to compare embeddings since it utilizes the angle between two vectors to determine their similarity, its inherent normalization means that the magnitude of the vectors is not taken into consideration and only their directionality is compared.
On the other hand, the \emph{euclidean distance} is a less common distance metric that we decided to include to lay out an alternate interpretation option, which can provide a different geometric perspective over the embedding space.
Unfortunately, in order to make the euclidean distance an effective metric we need to normalize it separately.

\subsubsection{Normalization}

On the topic of normalization, we also include the possibility of \emph{`pre-normalizing'} the embedding space before performing the experiments.
This practice should help improve the comparability of embedding vectors, enhancing the results of linear operations in the embedding space.
Although it is also possible for it to have detrimental effects in some specific scenarios, mainly due to the intrinsic nonlinearity of the normalization operation.
In practice, the `pre-normalization' step includes the normalization of all embedding vectors using the euclidean norm as follows:
\begin{equation}
    \label{eq:method_embeddings_normalization}
    \gbm{W}_{\textit{emb}} = (\bar w_{ij})_{\textit{vocab} \times n}
    \ \ \text{where}\ \bar w_{ij} = \frac{w_{ij}}{{\| w_{ij} \|}_2}
\end{equation}
\todo[cyan]{check dimension}

\subsubsection{Multi-token Words}

Handling multi-token words is a vital aspect of our analysis, as we may encounter words that are split into multiple sub-word tokens by the model's tokenization process.
As we will see later in \Cref{ssec:method_embeddings_analogies}, our main experiments for this section will concern the evaluation of analogies and similarities between word groups.
Therefore, we devised three main strategies to address multi-token words when they are fed into the embedding as inputs of analogies, along with two additional strategies to handle them as valid outputs of analogies.
Strategies applicable to both cases involve considering only the first token of a multi-token word (this causes results to be slightly worse on average, but the impact seems to be negligible in most cases).
For multi-token words as inputs, other approaches include calculating the average as already done in \Cref{eq:method_intravisto_emb-avg} or the sum over all the embedded tokens that form the multi-token word.
Conversely, for output multi-token words, we subdivide them into their token components, to then consider all tokens as targets for assessing the top-$k$ accuracy of an analogy result in the embedding space.
Formalizing the previously identified alternatives we obtain two $\textit{encode}$ functions controlled by $\textit{enc\_strat}$ for a series of $t$ tokens obtained from a word $w$:
\begin{equation}
    \label{eq:method_embeddings_multitok-in}
    {\textit{encode}}_{\textit{enc\_strat}}^{in}(w) = 
    \left\{
    \begin{array}{cl}
        \gbm{e}_1^w &\ \text{if}\ \textit{enc\_strat} = \text{`first\_only'} \\
        \frac{1}{t}\sum_{i=1}^{t}{\gbm{e}_i^w} &\ \text{if}\ \textit{enc\_strat} = \text{`average'} \\
        \sum_{i=1}^{t}{\gbm{e}_i^w} &\ \text{if}\ \textit{enc\_strat} = \text{`sum'}
    \end{array}
    \right.
\end{equation}
\begin{equation}
    \label{eq:method_embeddings_multitok-out}
    {\textit{encode}}_{\textit{enc\_strat}}^{out}(w) = 
    \left\{
    \begin{array}{cl}
        \{ tok_1^w \} &\ \text{if}\ \textit{enc\_strat} = \text{`first\_only'} \\
        \{ tok_1^w, \ldots, tok_t^w \} &\ \text{if}\ \textit{enc\_strat} = \text{`subdivide'}
    \end{array}
    \right.
\end{equation}
\vspace{0.25em}
\begin{equation*}
    \text{where}\  tok_1^w,\ldots,tok_t^w = \textit{tokenize}(w)
    ,\ \ \gbm{e}_1^w, \ldots, \gbm{e}_{t}^w = \Bigl\{ {{(\gbm{W}_{emb})}_{\textit{tok}_{i},\cdot}} \ |\ \forall tok_i \in \textit{tokenize}(w) \Bigr\}
\end{equation*}
One important distinction between the $\textit{encode}$ functions for input and output is the type of \todo{data} returned.
When performing input encoding we are interested in obtaining a single aggregated embedded representation of the target word, whereas, when encoding a word to be checked against the outputs of our experiment we need a set of token identifiers in the vocabulary space.
The first preliminary transformations performed on $w$ is tokenization, modeled as the $\textit{tokenize}$ function, and utilizing the model's tokenizer to obtain a set of token identifiers $tok_1^w,\ldots,tok_t^w$ from $w$.
After tokenization, tokens get transformed into embedding vectors $\gbm{e}_1^w, \ldots, \gbm{e}_{t}^w$ by selecting the corresponding row from the embedding matrix $\gbm{W}_{emb}$.
The role of the embedding matrix $\gbm{W}_{emb}$ can be fulfilled by either the input and output embedding matrices, as already seen previously.

As anticipated before in \Cref{ssec:method_intravisto_injection}, the main choice of computing multi-token representations using linear operations comes from the supposed preservation of linear features in the embedding space \todo[orange]{cite}.
Additionally, \todo[green]{add}.

Another approach that will be used to completely set aside the issues tied with multi-token words consists in reducing the dataset to exclusively consider analogies composed of single-token words, at the cost of less overall valid samples
This will be discussed more in depth in the section dedicated to the dataset of the experiment \todo[orange]{insert reference}.

\subsection{Word Analogies}\label{ssec:method_embeddings_analogies}

We wish to establish if geometric relationships can still be modeled inside the input or output embedding space of recent LLM architectures.
The first step \todo{towards the direction of this approach} was the direct experimentation on \emph{word analogy tasks}, as already illustrated in \citet{mikolov2013}.
Solving word analogies showcases the ability of an embedding space to model the semantics of words in a relatively consistent way, implying the existence of embedding dimensions with associated meanings (even if overlapping or under superposition as mentioned by \todo[orange]{cite}), which is not necessarily an expected feature of language models on a large scale.

\subsubsection{Analogy Computation}

\todo[green]{maybe slightly expand this text}
As far as the actual word analogy computation is concerned, given 4 words $\{w_1, w_2, w_3, w_4\}$ set up as an analogy of the type $`w_1 : w_2 = w_3 : w_4'$, we can formalize it as follows:
\begin{equation}
    \label{eq:method_embeddings_analogy}
    \begin{gathered}
        closest = \arg\min {\Bigl\{ d_{\textit{dist}}\bigl(\tilde w, \textit{encode}^{\textit{emb}}(v_{id})\bigr) \ |\ \forall v_{id} \in \mathcal{V} \Bigr\}} \\
        \tilde w = \textit{analogy}(w_1, w_2, w_3, w_4)
    \end{gathered}
\end{equation}
Where $d_{\textit{dist}}$ is the chosen distance metric as illustrated in \Cref{eq:method_embeddings_distance}, and $\textit{encode}^{\textit{emb}}$ refers to a simple encoding of a single token $v_{id} \in \mathcal{V}$ using the embedding matrix $\gbm{W}_{\textit{emb}}$.
On the other hand, \emph{`closest'} indicates the returned value, which is the actual token identifier of the closest embedding (of a word present in vocabulary $\mathcal{V}$) to the computed result of the word analogy.
\todo[green]{possibility of using top-k instead of closest}
Furthermore, the $\textit{analogy}$ function expresses the combination of words $\{w_1, w_2, w_3, w_4\}$ that composes the logic relationship of the word analogy.
Due to the fact that it is possible to define multiple analogy test cases from a single combination of words, the definitions of the actual \emph{analogy function} implementations slightly fall out of the scope of this section, while keeping in mind that they will be discussed more in depth inside the actual experimental setup section \todo[orange]{reference}.
Thus, for simplicity, we can assume to \todo{follow} the \todo{renowed} $`king - man + woman \approx queen'$ example from \citet{mikolov2013}, obtaining:
\begin{equation}
    \label{eq:method_embeddings_analogy-function}
    \begin{aligned}
        \tilde w &= \textit{analogy}(w_1, w_2, w_3, w_4) \\
        &= {\textit{encode}}_{\textit{enc\_strat}}^{in}(w_1) - {\textit{encode}}_{\textit{enc\_strat}}^{in}(w_2) + {\textit{encode}}_{\textit{enc\_strat}}^{in}(w_4) \\
        &\color{gray} \approx {\textit{encode}}_{\textit{enc\_strat}}^{out}(w_3) 
    \end{aligned}
\end{equation}
Where the ${\textit{encode}}_{\textit{enc\_strat}}$ functions follows the definitions laid out in \Cref{eq:method_embeddings_multitok-in,eq:method_embeddings_multitok-out}.

Another perspective over the classic word analogies task consists in the ``delta analogies'' task where the premise is similar, and the input words are the same.
However, we perform a further subdivision of the dataset, assigning each word analogy into a bigger group based on the actual relationship that is modeled by the analogy (such as `gender', `capital of', `royalty', \todo{etc.}).
Subsequently, we use multiple analogies from the same group to obtain the relationship term as a vector (delta term), and we observe if it is possible to shift the representation from an element of the analogy to the other utilizing the computed delta term.
\begin{equation}
    \label{eq:method_embeddings_delta-analogy-function}
    \begin{aligned}
        \tilde{w}^i &= \textit{delta\_analogy}(\gbm{\Delta}_{\textit{batch}}^{(w_3, w_4)}, w_3^i, w_4^i) \\
        &= {\textit{encode}}_{\textit{enc\_strat}}^{in}(w_3^i) + \gbm{\Delta}_{\textit{batch}}^{(w_3, w_4)} \\
        &\color{gray} \approx {\textit{encode}}_{\textit{enc\_strat}}^{out}(w_4^i) \\
        \gbm{\Delta}_{\textit{batch}}^{(w_a, w_b)} &= \frac{1}{B_{\textit{batch}}}\sum_{j=1}^{B_{\textit{batch}}}{\Bigl({\textit{encode}}_{\textit{enc\_strat}}^{in}(w_b^j) - {\textit{encode}}_{\textit{enc\_strat}}^{in}(w_a^j)\Bigr)} \\
        \text{where}&\ \textit{batch} = \Bigl\{ \{w_1^i, w_2^i, w_3^i, w_4^i\} \ |\ \forall i \in \{1,\ldots, B_{\textit{batch}}\} \Bigr\}
    \end{aligned}
\end{equation}
\todo{write better}
\todo[cyan]{explain why delta analogies are done on half analogy}
\todo[cyan]{change all instances of "delta analogies"}

\subsubsection{Analogy Evaluation}

\todo[green]{analogy evaluation metrics}

%To assess the models' performance in this particular task we opted to show the top-$k$ accuracy for each analogy, that is the accuracy with which the embeddings of a model can produce an object in the latent embedding space, with the correct answer to the analogy inside its $k$ closest elements.
%Consequently, only embedded elements that correspond to actual tokens within the model's vocabulary are considered when seeking the closest elements of an object.

\section{First Order Prediction}

\subsection{Matrix Comparison}
\subsection{Prediction Comparison}

\chapter{Datasets}
\label{ch:datasets}
\input{Chapters/5_Datasets.tex}

\chapter{Experiments}
\label{ch:experiments}
\input{Chapters/6_Experiments.tex}

\chapter{Conclusions}
\label{ch:conclusions}
\input{Chapters/7_Conclusions.tex}

%-------------------------------------------------------------------------
%	BIBLIOGRAPHY
%-------------------------------------------------------------------------

\addtocontents{toc}{\vspace{2em}} % Add a gap in the Contents, for aesthetics
\bibliography{Thesis_bibliography} % The references information are stored in the file named "Thesis_bibliography.bib"

%-------------------------------------------------------------------------
%	APPENDICES
%-------------------------------------------------------------------------

\cleardoublepage
\addtocontents{toc}{\vspace{2em}} % Add a gap in the Contents, for aesthetics
\appendix
\chapter{Appendix A}
If you need to include an appendix to support the research in your thesis, you can place it at the end of the manuscript.
An appendix contains supplementary material (figures, tables, data, codes, mathematical proofs, surveys, \dots)
which supplement the main results contained in the previous chapters.

\chapter{Appendix B}
It may be necessary to include another appendix to better organize the presentation of supplementary material.


% LIST OF FIGURES
\listoffigures

% LIST OF TABLES
\listoftables

% LIST OF SYMBOLS
% Write out the List of Symbols in this page
\chapter*{List of Symbols} % You have to include a chapter for your list of symbols (
\begin{table}[H]
    \centering
    \begin{tabular}{lll}
        \textbf{Variable} & \textbf{Description} & \textbf{SI unit} \\\hline\\[-9px]
        $\bm{u}$ & solid displacement & m \\[2px]
        $\bm{u}_f$ & fluid displacement & m \\[2px]
    \end{tabular}
\end{table}

% ACKNOWLEDGEMENTS
\chapter*{Acknowledgements}
Here you might want to acknowledge someone.

\cleardoublepage

\end{document}
