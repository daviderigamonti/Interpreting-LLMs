% A LaTeX template for MSc Thesis submissions to 
% Politecnico di Milano (PoliMi) - School of Industrial and Information Engineering
%
% S. Bonetti, A. Gruttadauria, G. Mescolini, A. Zingaro
% e-mail: template-tesi-ingind@polimi.it
%
% Last Revision: October 2021
%
% Copyright 2021 Politecnico di Milano, Italy. NC-BY

\documentclass{Configuration_Files/PoliMi3i_thesis}

%------------------------------------------------------------------------------
%	REQUIRED PACKAGES AND  CONFIGURATIONS
%------------------------------------------------------------------------------

% CONFIGURATIONS
\usepackage{parskip} % For paragraph layout
\usepackage{setspace} % For using single or double spacing
\usepackage{emptypage} % To insert empty pages
\usepackage{multicol} % To write in multiple columns (executive summary)
\setlength\columnsep{15pt} % Column separation in executive summary
\setlength\parindent{0pt} % Indentation
\raggedbottom

% PACKAGES FOR TITLES
\usepackage{titlesec}
% \titlespacing{\section}{left spacing}{before spacing}{after spacing}
\titlespacing{\section}{0pt}{3.3ex}{2ex}
\titlespacing{\subsection}{0pt}{3.3ex}{1.65ex}
\titlespacing{\subsubsection}{0pt}{3.3ex}{1ex}
\usepackage{color}

% PACKAGES FOR LANGUAGE AND FONT
\usepackage[english]{babel} % The document is in English  
\usepackage[utf8]{inputenc} % UTF8 encoding
\usepackage[T1]{fontenc} % Font encoding
\usepackage[11pt]{moresize} % Big fonts

% PACKAGES FOR IMAGES
\usepackage{graphicx}
\usepackage{transparent} % Enables transparent images
\usepackage{eso-pic} % For the background picture on the title page
\usepackage{subfig} % Numbered and caption subfigures using \subfloat.
\usepackage{tikz} % A package for high-quality hand-made figures.
\usetikzlibrary{}
\graphicspath{{./Images/}{./Images/Template/}{./Images/TikZ/img/}} % Directory of the images
\usepackage{caption} % Coloured captions
\usepackage{xcolor} % Coloured captions
\usepackage{amsthm,thmtools,xcolor} % Coloured "Theorem"
\usepackage{float}

% STANDARD MATH PACKAGES
\usepackage{amsmath}
\usepackage{amsthm}
\usepackage{amssymb}
\usepackage{amsfonts}
\usepackage{bm}
\usepackage[overload]{empheq} % For braced-style systems of equations.
\usepackage{fix-cm} % To override original LaTeX restrictions on sizes

% PACKAGES FOR TABLES
\usepackage{tabularx}
\usepackage{longtable} % Tables that can span several pages
\usepackage{colortbl}

% PACKAGES FOR ALGORITHMS (PSEUDO-CODE)
\usepackage{algorithm}
\usepackage{algorithmic}

% PACKAGES FOR REFERENCES & BIBLIOGRAPHY
\usepackage[colorlinks=true,linkcolor=black,anchorcolor=black,citecolor=black,filecolor=black,menucolor=black,runcolor=black,urlcolor=black]{hyperref} % Adds clickable links at references
\usepackage{cleveref}
\usepackage[square, numbers, sort&compress]{natbib} % Square brackets, citing references with numbers, citations sorted by appearance in the text and compressed
\bibliographystyle{abbrvnat} % You may use a different style adapted to your field

% OTHER PACKAGES
\usepackage{pdfpages} % To include a pdf file
\usepackage{afterpage}
\usepackage{lipsum} % DUMMY PACKAGE
\usepackage{fancyhdr} % For the headers
\fancyhf{}

% Input of configuration file. Do not change config.tex file unless you really know what you are doing. 
\input{Configuration_Files/config}

%----------------------------------------------------------------------------
%	NEW COMMANDS DEFINED
%----------------------------------------------------------------------------

% EXAMPLES OF NEW COMMANDS
\newcommand{\bea}{\begin{eqnarray}} % Shortcut for equation arrays
\newcommand{\eea}{\end{eqnarray}}
\newcommand{\e}[1]{\times 10^{#1}}  % Powers of 10 notation

%----------------------------------------------------------------------------
%	ADD YOUR PACKAGES (be careful of package interaction)
%----------------------------------------------------------------------------

\usepackage[threshold=1, thresholdtype=words]{csquotes}
\usepackage{booktabs}
\usepackage{cleveref}
\usepackage{dsfont}
\usepackage[export]{adjustbox}
\usepackage{makecell}
\usepackage{multirow}
\usepackage{hhline}
\usepackage{CJKutf8}
\usepackage{microtype}

% TODO: REMOVE after draft
\usepackage{soul}
\usepackage{todonotes}

%----------------------------------------------------------------------------
%	ADD YOUR DEFINITIONS AND COMMANDS (be careful of existing commands)
%----------------------------------------------------------------------------

\newcommand{\gbm}[1]{\bm{\mathbf{#1}}} % General Bold Math for both greek and latin letters

\renewcommand\mkblockquote[4]{\leavevmode\llap{``}\textit{#1#2#3}''#4} % Blockquote quotation marks
\renewcommand{\T}{\mathrm{T}} % Transpose

\setlength{\jot}{12pt} % Vertical equation spacing

\newcommand{\rowcolorhang}[1]{\rowcolor{#1}[\dimexpr\tabcolsep+0.1pt\relax]} % Add overang to table row coloring, to avoid vertical white lines

% TODO: REMOVE after draft
\renewcommand{\todo}[2][yellow]{\sethlcolor{#1}\hl{#2}} % TODO highlight

%----------------------------------------------------------------------------
%	BEGIN OF YOUR DOCUMENT
%----------------------------------------------------------------------------

\begin{document}

\fancypagestyle{plain}{%
\fancyhf{} % Clear all header and footer fields
\fancyhead[RO,RE]{\thepage} %RO=right odd, RE=right even
\renewcommand{\headrulewidth}{0pt}
\renewcommand{\footrulewidth}{0pt}}

%----------------------------------------------------------------------------
%	TITLE PAGE
%----------------------------------------------------------------------------

\pagestyle{empty} % No page numbers
\frontmatter % Use roman page numbering style (i, ii, iii, iv...) for the preamble pages

\puttitle{
	title={Interpreting Large Language \linebreak Models Through the Lens of \linebreak Embedding-Oriented \mbox{Visualizations}: \linebreak Markov Models, Sankey Diagrams and Comparative Approaches}, % Title of the thesis
	name=Davide Rigamonti, % Author Name and Surname
	course=Computer Science and Engineering \\ Ingegneria Informatica, % Study Programme (in Italian)
	ID=103152,  % Student ID number (numero di matricola)
	advisor=Prof. Mark Carman, % Supervisor name
	coadvisor={Nicolò Brunello, Vincenzo Scotti}, % Co-Supervisor name, remove this line if there is none
	academicyear={2024-25},  % Academic Year
} % These info will be put into your Title page 

%----------------------------------------------------------------------------
%	PREAMBLE PAGES: ABSTRACT (inglese e italiano), EXECUTIVE SUMMARY
%----------------------------------------------------------------------------
\startpreamble
\setcounter{page}{1} % Set page counter to 1

% ABSTRACT IN ENGLISH
\chapter*{Abstract} 
In recent years, the progress and widespread usage of Large Language Models (LLMs) has steadily increased.
Understanding the internal mechanisms of these complex and opaque engineering products poses an extremely relevant and critical challenge in the field of Natural Language Processing (NLP).
The NLP interpretability subfield aims to provide insights into the inner workings of language models (LMs).
This thesis presents InTraVisTo, an interactive visualization tool designed to provide an immediate and intuitive perspective on the generation process of LLMs by decoding their internal states, normally incomprehensible by humans, using tokens from their vocabularies.
InTraVisTo is implemented as an interactive framework that enables tracking of the flow of information through residual connections between model components, and it allows users to directly sway the model's predictions by modifying internal representations.
Additionally, this research investigates whether recent LLMs retain linear properties in their embedding spaces through the established semantic analysis framework of word analogies, finding that current state-of-the art LLMs are capable of encoding a surprising amount of semantic relationships within their embeddings.
Furthermore, this work examines the feasibility of obtaining first-order predictions by concatenating input and output embeddings in large-scale models, testing whether LLMs inherently approximate a Markov model structure.
Our findings demonstrate that first-order models (FOMs) extracted from LLMs exhibit a clear bias towards modeling a Markovian behavior, despite significant differences in accuracy across model architectures.
Overall, this thesis confirms previous findings on new architectures and introduces a robust tool aimed at making LLMs more transparent and accessible to researchers in the NLP field.
\\
\\
\textbf{Keywords:} AI, NLP, Interpretabilty, Word Embedding, LLM

% ABSTRACT IN ITALIAN
\chapter*{Abstract in lingua italiana}
Negli ultimi anni, lo sviluppo e la diffusione su larga scala dei Large Language Models (LLMs) hanno registrato un costante incremento.
Comprendere i meccanismi interni di simili prodotti ingegneristici di natura complessa e opaca costituisce una sfida di rilevanza e importanza critica nel campo dell'Elaborazione del Linguaggio Naturale (NLP), di conseguenza il sottocampo dell'interpretabilità nell'ambito dell'NLP si propone di fornire una comprensione approfondita del funzionamento interno dei modelli linguistici (LMs).
In questa tesi si presenta InTraVisTo, uno strumento di visualizzazione interattiva concepito per offrire una prospettiva intuitiva ed immediata sul processo generativo svolto dagli LLM mediante la decodifica dei loro stati interni, altrimenti umanamente incomprensibili, attraverso l'impiego dei token presenti nel relativo vocabolario.
InTraVisTo è implementato in qualità di framework di natura interattiva, che consente di monitorare il flusso informativo attraverso connessioni residue tra i vari componenti del modello e permette agli utenti di intervenire influenzando direttamente le previsioni modificando le rappresentazioni interne appartenenti al modello.
Inoltre, in questa ricerca vengono analizzati gli LLM più recenti al fine di verificare se conservino le proprietà lineari all'interno dei rispettivi spazi di embedding, adoperando il consolidato framework di analisi semantica basato sulle analogie tra parole, evidenziando come i modelli di ultima generazione siano capaci di codificare un sorprendente numero di relazioni semantiche nei loro embedding.
In aggiunta, questo lavoro esamina le possibilità di ottenere previsioni di primo ordine attraverso la concatenazione degli embedding di input e output su modelli di larga scala, verificando se gli LLM approssimino intrinsecamente la struttura di un modello di Markov.
I risultati ottenuti evidenziano che i modelli di primo ordine (FOM) derivati dagli LLM manifestano una chiara propensione verso un comportamento Markoviano, sebbene emergano significative differenze di accuratezza in funzione delle architetture adottate.
In sintesi, la presente tesi conferma risultati storici su nuove architetture e introduce un robusto strumento finalizzato a rendere gli LLM maggiormente trasparenti e accessibili alla comunità di ricerca nel campo dell'NLP.
\\
\\
\textbf{Parole chiave:} AI, NLP, Interpretabilità, Word Embedding, LLM

%----------------------------------------------------------------------------
%	LIST OF CONTENTS/FIGURES/TABLES/SYMBOLS
%----------------------------------------------------------------------------

% TABLE OF CONTENTS
\thispagestyle{empty}
\tableofcontents % Table of contents 
\thispagestyle{empty}
\cleardoublepage

%-------------------------------------------------------------------------
%	THESIS MAIN TEXT
%-------------------------------------------------------------------------
% In the main text of your thesis you can write the chapters in two different ways:
%
%(1) As presented in this template you can write:
%    \chapter{Title of the chapter}
%    *body of the chapter*
%
%(2) You can write your chapter in a separated .tex file and then include it in the main file with the following command:
%    \chapter{Title of the chapter}
%    \input{chapter_file.tex}
%
% Especially for long thesis, we recommend you the second option.

\addtocontents{toc}{\vspace{2em}} % Add a gap in the Contents, for aesthetics
\mainmatter % Begin numeric (1,2,3...) page numbering

\chapter{Background and motivations}\label{ch:background}
With the current escalation in popularity and pervasiveness of machine learning solutions applied to the realm of natural languages, there exists an increasingly relevant topic that is still being the subject of extensive studies by experts in this field: interpretabilty.
Interpretabilty in itself is far from being a novel concept, as from the early advent of machine learning there has always been the necessity of understanding the inner process by which an opaque model performs its inner computations, possibly even going as far as giving an algorithmic interpretation to it. 

With the introduction of deep learning, the need for interpretabilty spiked as models got increasingly more complex.
The entire paradigm of deep learning is centered around the removal of human control over the feature space, letting models figure out `what needs to be learnt' to solve the problem at hand.
In particular, we are interested in a specific class of machine learning models that are currently considered state-of-the art in the Natural Language Processing (NLP): Large Language Models (or LLMs).

This works intends to provide novel perspectives over the internal structure of Large Language Models, this goal is achieved by performing exploratory analyses, through the creation of tools and conducting of experiments aimed at confirming and possibly improving existing hypotheses and observations in the interpretability field.
Particular focus will be placed on the observation and interpretation of the models' hidden states, which are their internal representations, conveying condensed information between various internal components and acting as the `internal language' of the Language Model.

In this introductory section we will analyze the background and context of both NLP and machine learning techniques that will be relevant for this work.
This serves the purpose of providing a common knowledge base as a starting point, briefly covering all of the pertinent subjects.

\section{Machine Learning}

Machine Learning is commonly considered a subfield of Artificial Intelligence, and is characterized by the idea of training a machine to learn from past experience, without providing an explicit algorithm to execute.
The term `past experience' is often used loosely in this scenario: it is not necessarily tied with information directly perceived by an agent within a specific environment.
Instead, it commonly denotes the entirety of accumulated data, often referred to as the dataset, which may have been collected heterogenuously and asinchronously, and that is fed to the model in order to learn.

There exist three main paradigms of learning for machine learning:
\begin{itemize}
    \item \textbf{Supervised learning}: in supervised learning, the data that we feed to the model is labeled, meaning that each data point is associated to a specific label representing the ground truth.
The model receives a sample from the dataset as input, eliciting an output that fulfills the role of the model's prediction.
This prediction is then compared against the label corresponding to the input data point.
If there is a discrepancy between the two, the model's internals are algorithmically adjusted to minimize future suboptimal predictions.
Most machine learning techniques belong to this category, and it also will be our main focus for this work.
    \item \textbf{Unsupervised learning}: unsupervised learning detaches the labeled component from the dataset, and removes direct feedback from the model's training process.
In this scenario we let the model autonomously find patterns inside data.
This process usually implies some kind of summarization or condensation of information, however the common factor is given by the search of hidden structure inside data.
Clustering techniques are a fairly common and well-known instance of unsupervised learning. 
    \item \textbf{Reinforcement learning}: reinforcement learning mostly concerns the decision-making process of an agent placed inside a dynamic environment.
The inputs of a reinforcement learning algorithm are represented by the environment state and rewards, while the agent's actions are treated as output.
In this paradigm we are trying to discover the optimal policy to navigate the environment, balancing both the exploration of new states, and the exploitation of the current maximally rewarding actions.
\end{itemize}

\subsection{Neural Networks}

Neural networks are one of the most popular machine learning techniques.
They are heavily inspired by the structure of the human brain, which is generally considered to be constituted by multiple neurons connected by synapses.
Classical neural networks usually have a layer-based architecture, where a layer consists of an array of neurons connected to neurons belonging to both the previous and next layers.
A neural network may have any number of intermediate layers, an input layer (which collects information at its inputs) and an output layer (which aggregates the outputs of its neurons into the final network prediction). 

Neurons are generally mono-directional and perform a linear combination of their inputs, each one scaled by the weight associated to its synapse.
Before transmitting their output, they also apply an activation function which in most cases is a `S-shaped' nonlinear function.
This is mainly chosen in order to prevent the network from just degenerating into a single linear transformation due to the linearity properties.
Thus, nonlinearities directly affect the expressive power of neural networks and are a vital component to guarantee their success, even on complex problems that require elaborate decision boundaries.

The central process used by neural networks to learn new information is backpropagation.
It works by following the inverse flow of information and updating the network's parameters (represented by the synapses' weights) according to the gradient value accumulated inside each neuron.
The gradient is said to be `accumulated inside neurons' for computational reasons, however it is always computed starting from the loss function at the output of the network and with respect to the parameters of the network.

\subsection{Deep Learning}

Deep Learning is a further specialization of machine learning.
The `deep' adjective refers to the high number of hidden layers that compose the intermediate section of deep neural networks.

The most important property obtained through the dimensonal scale-up is the hierarchical feature extraction capability, which consists in obtaining rich condensed representations that follow a semantic hierarchy from the latent spaces defined by the network's layer projections.

This is not a property that is exclusive to deep architectures, even classic neural networks present the same hierarchical feature patterns.
However, the sheer scale of deep neural networks makes the exploitation of these condensed representations a viable option to build an artificial feature base that has many upsides with respect to human-defined features.
For instance, we may require expert knowledge in order to craft an appropriate dataset containing meaningful features, while this would be done automatically in a deep learning scenario.

The main drawback of this approach comes from the fact that learned features are not directly human-understandable, as they result from mathematical optimization.
Consequently, the process of explaining a model's prediction is not as straightforward as with classical machine learning techniques.

The latent feature space concept and its properties are of particular relevance for this work, as most of the experiments that will be performed are aimed at attempting to understand them in the context of state-of-the-art models in the Natural Language Processing field of study.

\section{Natural Language Processing}

Natural Language Processing (NLP) is a subfield of computer science that stems off the branches of artificial intelligence, information engineering and computational linguistics.
Its main purpose is to interface computers to languages that are commonly used between humans for the purpose of communication.
This entails a great variety of many different subtasks, from text classification and machine translation to text generation itself.
The main focus of most of the state-of-the art models is text generation, even though it is possible to repurpose them for many other tasks due to their modular nature.

One of the first promising approaches to text generation was the Recurrent Neural Network (RNN) architecture, which featured a structure much more similiar to classic neural networks, but with the addition of recurrent connections.
Recurrent connections enable RNN models to establish a distributed hidden state that acts as a memory, much like a flip-flop gate array in electronics.
In practice, each RNN module is fed the inputs at the current time instant and its own outputs at the previous time instant, making it suitable to process sequences and other time-based data.

The most popular architecture for text generation before the current state-of-the-art models was the Long Short-Term Memory (LSTM).
LSTMs are a direct improvement on the RNN, solving the vanishing gradient problem, which was the main failure point of classic RNNs.
The \todo{overwhelming} vanishing gradient in RNNs is a direct consequence of the co-occurrence between activation functions without a prominent zero-valued region and the Backpropagation Through Time process (being the backpropagation equivalent of a network using recurrent connections), resulting in drastically small gradient values which can limit the model's context window and make the training process slow and unreliable.
LSTMs solve this by applying specialized gating functions that let the model decide what information to rembember, forget and pass along into what is called the Constant Error Carousel.

\subsection{Word Embeddings}

One of the most relevant and groundbreaking developments in the field of NLP was the introduction of word embeddings.
Word embeddings came as straightforward application of deep learning concepts and approaches to the world of NLP.
Their introduction not only determined an improvement in most text-based applications, but also motivated a shift in text representation techniques, incentivizing a word-centric approach rather than focusing on documents and defining words as frequencies inside them.

The main appeal of word embeddings is the dimensionality reduction, since the original vocabulary representation (often referred as Bag of Words) suffers from the curse of dimensionalty to a great extent.
The `curse of dimensionality' is a common way to refer to feature spaces that present a large number of dimensions in conjunction with high data sparsity.
In a feature space affected by the curse of dimensionality, as the vocabulary space is, we can observe the fact that some common similarity measures and distances between data points (such as the Euclidean distance) almost lose meaning due to the majority of dimensions suppressing comparisons along the few dimensions that actually matter.
Word embeddings are not able to completely solve the curse of dimensionality as they still are relatively high-dimensional.
However, the improvement of adopting a dense feature space, with the choice of a normalized similarity function (such as cosine similarity) was still substantial enough to revolutionize the NLP field.

In fact, another enticing property is their ability to capture the semantic depth and context information inside the representation of each word.
Word embeddings are generally created by extrapolating the intermediate latent representations of a neural network solving a certain task, obtaining a continuous vector space for words to reside.
In practice, word embeddings are mappings from the vocabulary space to the embedding space which features a reduced number of dense dimensions devoid of any human-understandable meaning, characteristic of deep learning methodologies.

The concept of using distributed word representations was anticipated by \citet{bengio2020} with the Neural Network Language Model (NNLM) in the early 2000s.
The first real successful implementation of word embeddings was Word2Vec by \citet{mikolov2013}, greatly inspired by Bengio's previous work.
The main purpose of Word2Vec is to create word embeddings, it is composed by a single hidden layer and is trained discriminatevely by using both positive and negative examples.

\blockquote[J.R. Firth]{You Shall Know a Word by the Company It Keeps}

Mikolov's original work introduces two primary training variants for Word2Vec:
\begin{itemize}
    \item \textbf{Continuous Bag of Words (CBoW)}: predict a target word given its context.
The context is given as a symmetric window containing all terms occuring around the target term, modeling a many-to-1 prediction.
    \item \textbf{Skip-gram}: predicts surrounding words given a target word.
Single context words are predicted one at a time, modeling a 1-to-1 prediction.
\end{itemize}
The results are generally comparable between the two techniques.
The key aspect is that the full context of each word is integrated into the word representation itself.

\subsubsection*{Word Embeddings Semantic Properties}

One particular property that emerges from the majority of latent representations is their flexibility to arithmetic manipulations.
It is not uncommon to observe these properties showcased in many other deep learning applications, such as computer vision.
However, in the word embeddings scenario, these arithmetic properties have been reported to be much more prominent and \todo{reflecting} of the actual word semantics.

In particular, the semantic meaningfulness of certain arithmetic and geometric transformations applied to word embeddings was noticed by Mikolov himself in the Word2Vec paper \cite{mikolov2013}.
This went from finding similarities in embeddings of words that manifested syntactic regularities, to complete double-word analogies located in common semantic fields.
One of the most famouse examples presented in \cite{mikolov2013} is: \texttt{emb(`King') - emb(`Man') + emb(`Woman') $\simeq$ emb(`Queen')}, which models both the relationships of `gender' and `royalty'.
Other \todo{relevant} analogies include country-capital, family relation, and verb tenses.
However, any kind of semantic relationship could be potentially modeled by word embeddings, given enough support in the training dataset.

While these properties have some direct use cases principally involving clustering and similarity comparisons, their study has been proven to be insightful from the explainability standpoint.
Possibly understanding the semantic associations that exist \todo{at the base} of text models can provide an educated perspective on how they interpret their inputs, but also underline human and language biases.
Additionally, this topic is of particular relevance due to the focus on interpretability of this work.

\section{Transformer Architectures}

Transformers are the current state-of-the art model architecture for most text applications and not only.
Although they were initially conceived for textual sequential inputs, in recent years saw discrete success in a wide variety of other applications (in particular image processing), albeit occasionally with some slight variations.
This architecture has proven to be flexible, efficient and parellizable, in contrast with previous popular choices for sequence learning problems such as RNNs and LSTMs.

The first transformer architecture was introduced by \citet{vaswani2017} in the seminal 2017 paper ``Attention Is All You Need''.
This architecture is of the encoder-decoder type, following the typical paradigm of sequence-to-sequence (seq2seq) learning.
This structure implies the existence of two separate modules (an encoder and a decoder), which work in tandem to elaborate inputs and provide an output.
Both input and output consist of sequential data.

The purpose of the encoder module is to process the inputs and create an intermediate representation that is read and processed by the decoder, which outputs the next token.
The encoder and the decoder have a fairly similar structure, both featuring multiple basic blocks that can be sequentially connected to each other (operation commonly referred to as `stacking').

\subsection{Structure}

\subsubsection*{Preprocessing}

The first step to process inputs is tokenization, as it is fundamental to reduce the possibly infinite vocabulary to a finite set of tokens that can be embedded and understood by the model.
This is normally done by a tokenizer, which performs some preprocessing on the input sentence to obtain a sequence of tokens that will be fed to the model.
Most transformer architectures use different variants of sub-word tokenization algorithms, meaning that each word may correspond to a combination of multiple tokens.
This type of tokenization is usually obtained as the result of a training process where common character sequences get progressively aggregated until they reach a certain frequency threshold, at this point a compact and efficient representation of the vocabulary that is balanced between granularity and size is achieved.
Sub-word tokenization improves the ability of the model to deal with rare and unknown words, greatly limiting the use of unknown tokens (special tokens used in place of certain words that cannot be correctly tokenized and encoded) and potentially facilitating the creation of multi-lingual models.

The tokenization and embedding steps are perhaps the most critical processes due to the fact that they shape both inputs and outputs of the transformer, framing how it interfaces itself with the environment.
Additionally, the transformer architecture is word position agnostic, meaning that embedded tokens do not directly contain any additional information that \todo{tells} the model their position inside the input sentence.
This is commonly solved by implementing \textbf{positional encoding}: an additive mask that gets applied to the input tokens right after the embedding step.
The original ``Attention Is All You Need'' \cite{vaswani2017} paper suggests the use of fixed positional encodings based on the \textit{sine} and \textit{cosine} functions, although alternatives exist.

\subsubsection*{Basic Block Structure}

Transformer blocks \todo{compose} large part of the transformer architecure and contain \textbf{attention blocks}, \textbf{feed-forward blocks}, \textbf{residual connections} and \textbf{normalization blocks}.

\textbf{Attention blocks} are \todo{powered} by the attention mechanism, the \todo{principal mechanic} responsible for the outstanding results achieved by transformer models.
The concept of attention is not novel and did already see some applications in seq2seq models \todo[orange]{cite https://arxiv.org/pdf/1409.0473, https://arxiv.org/pdf/1702.00887, https://arxiv.org/pdf/1507.01053}, determining substantial improvements in most LSTM architectures.
Attention in encoder-decoder models is implemented by allowing the decoder to \todo{look} at the internal states generated by the encoder, thus eliminating the information bottleneck represented by the single point of connection between the two.
The most important part of attention is its scoring function, which allows the decoder to `focus' on certain tokens by modeling a weight distribution on the attended hidden states using an attention function.
There exist different attention function implementations such as \textit{simple dot product}, \textit{Luong (multiplicative)} \todo[orange]{cite} and \textit{Bahdanau (additive)} \todo[orange]{cite}.
The original transformer architecture mentioned in ``Attention Is All You Need'' \cite{vaswani2017} uses \textit{scaled dot product attention}, which is implemented with the following formula:
\begin{equation}
    Attention(Q,K,V) = softmax\left(\frac{QK^\mathrm{T}}{\sqrt{d_K}}\right)V
\end{equation}
Where $Q$, $K$ and $V$ represent the \textit{query}, \textit{key} and \textit{value} components of attention which are obtained by multiplying the hidden state vectors to their respective projection matrices $W_Q$, $W_K$ and $W_V$, which are composed of leaernable model parameters.

The transformer arhitecure implements three different types of attention:
\begin{itemize}
    \item An \textbf{encoder-decoder attention} (or \textbf{cross-attention}), which is found inside the decoder block, and attends to the encoder blocks' hidden states using its current state as the \textit{query}.
    \item The \textbf{self-attention}, which is used by both the encoder block to process the hidden states at its input.
It works in following the same principles as cross-attention however, it attends to representations of the same nature, meaning that \textit{query}, \textit{key} and \textit{value} are extrapolated by the same set of hidden states.
In practice, the model is able to manage long-range dependencies and possibly capture patterns or associations between representations in a completely parallelizable way.
    \item Lastly, \textbf{masked self-attention} is a specific implementation of self-attention that can be found in the decoder block, due to its need of autoregressive modeling.
In fact, the decoder is not allowed to use information about future words, since it is capable of returning a single output token per stack execution.
Masked self-attention differs from self-attention in scenarios where the full output is available (e.g.\ training), and it works by adding a mask with the effect of nullyfing the contribution of future tokens in the attention computation.
The masked self-attention formula can be summarized as follows:
\begin{equation}
    \begin{gathered}
        MaskedAttention(Q,K,V) = softmax\left(\frac{QK^\mathrm{T} + M}{\sqrt{d_K}}\right)V \\
        M_{ij} = \begin{cases}
            0 & \text{if}\ i \ge j, \\
            -\infty & \text{otherwise}.
        \end{cases}
    \end{gathered}
\end{equation}
\todo[cyan]{check M definition equation}
\end{itemize}

In reality, transformers use a more refined attention implementation, called \textbf{multi-head attention} which can be applied in all the previously identified scenarios.
Multi-head attention is a novelty introduced in the ``Attention Is All You Need'' \cite{vaswani2017} paper and it consists in replicating the attention structure multiple times, thus obtaining multiple representation spaces for each attention layer.
This operation influences the dimension of attention vectors, which needs to be divided by $h$, where $h$ is the chosen number of separate attention heads for multi-head attention.
The attention output for multi-head attention is obtained as the concatenation between the attention outputs of all $h$ heads as follows:
\begin{equation}
    \begin{aligned}
    MultiHead(Q,K,V)    &= Concat(head_1,\ \ldots,\ head_h) \\
                        &= Concat(Attention_1(Q,K,V),\ \ldots,\ Attention_h(Q,K,V))
    \end{aligned}
\end{equation}
In practice, multi-head attention enables the model to simultaneously focus on information present at differnet positions, avoiding excessive attention on single tokens.

\todo[green]{attention dimensional analysis}

The second main \todo{block} contained in the transformer block is the \textbf{feed-forward block}.
As the name suggests, it consists of a simple feed-forward network and is normally situated as the last step of the transformer block, using the outputs of the previous attention layers as inputs.
The intermediate representations are processed by the feed-forward layer flow in a completely independent manner, therefore the execution of this block is parallelizable by nature.
Feed-forward blocks' principal contribution to the transformer architecture consists in the introduction of nonlinearities, since attention is still a predominantly linear process.
The feed-forward network acts as a `grouping' mechanism, where information gathered in previous steps through attention is aggregated, processed and reformulated, adding depth to the computation.

\textbf{Layer normalization} and \textbf{residual connections} are techniques that are commonly found in most deep learning models and are generally used to improve gradient flow, training stability and overall generalization.
In particular:
\begin{itemize}
    \item \textbf{Layer normalization} \todo[orange]{cite https://arxiv.org/pdf/1607.06450} is applied after each attention and feed-forward block.
It consists of two steps: a batch-independent normalization step performed with parameters ($\mu$, $\sigma$) which model a normal distribution over the feature space of the vector representations, and a linear transformation using a bias and a scale factor.
The bias and scale factor are usually referred to as $\beta$ and $\gamma$ respectively, and are trainable layer-level parameters which perform a shift and rescale operation on the normalized vector.
If $v_i$ is a vector representation, after layer normalization at layer $\ell$ we obtain:
\begin{equation}
    \begin{gathered}
        \bar v_{i,\ell} = \gamma_\ell \cdot \left( \frac{v_i - \mu}{\sigma} \right) + \beta_\ell, \\
        \text{where}\ \mu = \frac{1}{n}\sum_{i=1}^{n}{v_i},\ \sigma = \sqrt{\frac{1}{n}\sum_{i=1}^{n}{(v_i - \mu)^2}}
    \end{gathered}
\end{equation}
\todo[cyan]{Check equation pedices}
    \item \textbf{Residual connections} \todo[orange]{cite https://arxiv.org/pdf/1512.03385} are a strategy employed in deep architectures to streamline gradient propagation and avoid vanishing gradients.
Additionally, it helps the transformer to preserve local information as attention and feed-forward blocks are added to the base residual flow, limiting their freedom in the effective output space.
In practice, residual connections sum the identity function of the input of a block to the output of that block, in the transformer's case the resiudal summation happens before layer normalization.
\end{itemize}

\subsubsection*{Postprocessing}

Of particular importance is the last step in the transformer generative process.
After the inputs have been embedded and have been passed through all the decoder's transformer blocks, iteratively refinining them into new vector representations, the decoder has to output a new token.
Given the fact that a transformer model generates a single token at a time, we would presumably obtain a single hidden representation at the end of the decoder stack with a size compatible to the embedding space.
However, in order to output a vocabulary token we need the vector to lay in the vocabulary space.
To achieve this goal, the transformer architecture features a last linear layer outside of the stack with the purpose of bringing the last hidden representation from the embedding space to the vocabulary space, modeling an `unembedding' operation.
In some specific model architectures, this last weight matrix is forced to be equal to the transpose of the initial embedding matrix for the sake of retaining a consistent representation between the first and last layers; other architectures opt to learn a different embedding representation space for the outputs of the model.
After the `unembedding' operation, the resulting vector of floating point numbers (commonly referred to as logits) will be laying in the vocabulary space.
However, in order to get an actual probability distribution over the vocabulary for the next token of the input sequence, a last softmax application is needed.

\subsection{Current Decoder Architectures}

What was previously described is generally considered the first transformer architecture that was originally mentioned in the ``Attention Is All You Need'' \cite{vaswani2017} paper in 2017.
However, with time, a large number of variations and improvements on the base architecture has struck the NLP field.
We are going to be mainly concerned with the Llama 2 \todo[orange]{cite} and Llama 3 \todo[orange]{cite} (from here just Llama) models, developed by Meta AI \todo{reference} in 2023 and 2024 respectively; other minor models may also be taken into consideration, although without particular focus on their architecture.

The first main novelty in the new state-of-the-art models is the complete removal of the encoder from the architecture.
These models are often called \textbf{`decoder-only'} or \textbf{`GPT-like'} from the Global Pretrained Transformer (GPT) architecture \todo[orange]{cite https://openai.com/index/language-unsupervised/} that popularized the use of transformer models without the decoder, although the first documented use of a `decoder-only' architecture can be ascribed to this 2018 paper by Liu et al.\todo[orange]{cite https://arxiv.org/pdf/1801.10198} 
The major improvement of decoder-only architectures consists in a substantial performance enhancement for longer sequences of text.
In addition, removing the encoder also removes possible information redundancy between the two components.
The decoder remains mostly unchanged from this modification, with the exception of the cross-attention block since it can no longer fetch the representations generated by the encoder
Cross-attention is consequently removed, \todo{leaving} a single masked self-attention block \todo{inside} the decoder architecture.

Another important change in the transformer architecture is the \textbf{Rotary Position Embeddings (RoPE)} mechanism, proposed in 2021 by Su et al. \todo[orange]{https://arxiv.org/pdf/2104.09864}
This technique completely replaces the positional encoding strategy that was previously employed in transformer models (absolute positional encoding) by guaranteeing better flexibility for long sequences and introducting the decay of inter-token dependency with increasing relative token distance.
One of the main flaws of absolute positional encoding is its inability to model dependence between tokens, as each token positional encoding is independent from the others and, in particular, from the distance between tokens.
RoPE is able to model both absolute and relative token positional information by encoding them as a geometric rotation inside the embedding space.
This rotation operation, which has a multiplicative nature (opposed to the additive one of absolute positional encoding), has the additional advantages of avoiding any change in vector norm for the subjected hidden representation.
This property is vital in determining RoPE's capability of being directly applied inside the query and key matrices in linear self-attention, improving overall efficiency.

The original transformer implementation establishes the presence of layer normalization right after each attention and feed-forward blocks, after residual addition.
Extensive research has been performed on possible alternatives and a notable shift from models implementing classic normalization (post-norm) to models implementing pre-normalization emerged.
The \textbf{pre-normalization (pre-norm)} \todo[orange]{cite https://arxiv.org/pdf/1809.10853} \todo[orange]{cite https://aclanthology.org/P18-1167.pdf} \todo[orange]{cite https://github.com/tensorflow/tensor2tensor?tab=readme-ov-file} approach consists in applying layer normalization directly inside the residual block, right before the attention or feed-forward blocks.
Additionally, a further layer normalization operation is performed before prediction, right after the last transformer block.
The main benefits of this technique are tied to training efficiency and convergence speed \todo[orange]{cite https://arxiv.org/pdf/2002.04745 (pre-norm advantages)} as it was shown to improve overall gradient stability, even at initialization time.
This improvement allowed the removal of the `learning rate warm-up stage': a technique that was commonly used in post-norm transformer architectures to avoid diverging in training \todo[orange]{cite https://arxiv.org/pdf/1804.00247} and involved a gradual increase of the learning rate in the first training epochs \todo[orange]{cite https://github.com/tensorflow/tensor2tensor?tab=readme-ov-file}.

Another variation on the transformer normalization mechanism was achieved with the introduction of \textbf{root mean square layer normalization (RMSNorm)} \todo[orange]{cite https://arxiv.org/pdf/1910.07467}.
The fundamental improvement introduced by RMSNorm is a substantial reduction in the amount of computation performed with respect to classical layer normalization and an overall gain in efficiency, this is especially true for deep architectures where the computational overhead of layer normalization is meaningful.
Implementation wise, RMSNorm only focuses on the rescaling aspect rather than including both centering and rescaling as layer normalization does.
The central quantity used to perform the rescaling computation is the Root Mean Square (RMS) statistic and only the $\gamma$ parameter is retained from the set of learnable parameters due to the fact that the re-centering operation performed by $\beta$ was deemed to be mostly ininfluential for both layer normalization and RMSNorm.
If $v_i$ is a vector representation, after RMSNorm at layer $\ell$ we obtain:
\begin{equation}
    \begin{aligned}
        \bar v_{i,\ell} = \gamma_\ell \cdot \frac{v_i}{RMS(v)}, &&
        \text{where}\ RMS(v) = \sqrt{\frac{1}{n}\sum_{i=1}^{n}{v_i^2}}
    \end{aligned}
\end{equation}
One peculiar side-effect of RMSNorm is the fact it forces the summed inputs into a $\sqrt{n}$-scaled unit sphere which benfits the stability of layer activations and output distribution, positively influencing the interpretability of the hidden space.

The last main change featured in most recent decoder-only transformer architectures concerns the structure of the feed-forward block.
The base transformer implementation esetablishes the feed-forward layer as the combination of two linear transformations separated by a non-linear activation function, tipically a choice between \textit{ReLU} and \textit{GELU}.
However, newer architectures have begun to incorporate Gated Linear Units (GLUs) \todo[orange]{cite https://arxiv.org/pdf/1612.08083} implemented using a \textit{Swish} (also called \textit{SiLU}) activation function, this particular architectural combination for the feed-forward block is commonly referred to as \textbf{SwiGLU} \todo[orange]{cite https://arxiv.org/pdf/2002.05202v1}.
\textit{Swish} \todo[orange]{cite https://arxiv.org/pdf/2002.05202v1} is an activation function similar to \textit{ReLU}, characterized by a non-monotonic depression in its zero-region, and is implemented in the following way: $Swish(x) = x \cdot \sigma(x)$ where $\sigma(x)$ represents the logistic sigmoid function.
\textit{Swish} was shown to offer marginal improvements over \textit{ReLU} and other similar variations, based on its smoothness and possibility of returning small negative values for inputs close to zero, implying the efficient convergence of non-zero gradients and consequently minimizing the problem of dead neurons (caused by the nullification of gradient).
On the other hand, GLUs determine an entire revision of the feed-forward architecture, as they introduce a gating mechanism where one linear transformation is modulated by the output of another, which allows the model to dynamically control the flow of information.
In the case of Llama models, the gating mechanism happens through an element-wise multiplication where $W_{in} \in \mathbb{R}^{d_{model} \times d_{hid}}$ represents the input transformation matrix with its respective bias $b_{in}$, $V \in \mathbb{R}^{d_{model} \times d_{hid}}$ is the gate transformation matrix with its respective bias $b_{V}$, $W_{out} \in \mathbb{R}^{d_{hid} \times d_{model}}$ performs the out-projection to the model's hidden representation dimensionality with its respective bias $b_{out}$ and $f$ represents the activation/gating function (in the SwiGLU case $f(x) = Swish(x)$):
\begin{equation}
    FFN(x) = GLU(x) * W_{out} + b_{out} 
    = \Bigl( f(xV + b_V) \odot xW_{in} + b_{in} \Bigr) * W_{out} + b_{out}
\end{equation}
\todo[cyan]{check dimensionality}
For Llama models, all biases of the transformation matrices referring to the SwiGLU implementation are set to zero ($b_{in} = b_V = b_{out} = \vec 0$).
The benefits of SwiGLU are largely tied to providing a better downstream performance on fine-tuning tasks and pre-training objectives, despite proof of SwiGLU actually having an impactful effect on transformer models being purely empirical.
However, the reasons for its effectiveness may potentially be associated with its capability to model complex functions through the gating mechanism. \todo[orange]{cite https://arxiv.org/pdf/2002.05202v1 https://kikaben.com/swiglu-2020/}

\subsection{The Large Language Model Paradigm}

\todo[green]{Insert "LLM" subsection containing pre-training, fine-tuning, open source models...}

\chapter{Related works}\label{ch:related_works}
The field of transformer interpretability has garnered significant attention over the past decade, resulting in a substantial and ever-increasing body of literature.
Particular \todo{care} must be put into the fact that this is a relatively novel field of research, which is constantly being subjected by a great number of contributions \todo{as of the time of writing}.
Consequently, it is very possible that some of the information provided in this work may be obsoleted or invalidated by more recent works.
This section reviews the key contributions and developments in this area, highlighting the foundational studies and recent advancements that are pertinent to the present research.

In such a fast-moving and prolific field it is nearly impossible to consider every relevant contribution, and it is inevitable that there might be gaps in the considered material.
To this end we propose a cutoff date that standardizes a fixed knowledge basis to build upon.
It is important to note that there may be instances where the proposed cutoff date will be disregarded, particularly in cases of recent exceptional contributions to the field that potentially revolutionize, subvert or significantly alter the context of the present research, and therefore deserve to be acknowledged.
The chosen cutoff date is July 2024, which corresponds to current time of writing.

\todo[gray]{V: Too much text outside subsections, consider adding an additional "Overview on Transformers Interpretability" section}

\citet{rai2024} propose a taxonomy for interpretability techniques centered around the concept of Mechanistic interpretability (MI).
It is possible to identify two main fundamental objects of study in this context: \textbf{features} and \textbf{circuits}.
\todo[green]{Maybe explain features and circuits}
These objects of study serve as starting points for interpretability inquiries, while specific techniques act as tools to explore and verify those inquiries.
By using MI tools to pursue interpretability inquiries, possibly through the use of evaluation techniques, we obtain findings: true generalizable statements about the model's inner workings.

Mechanistic interpretability offers a novel perspective over the interpretability research field, its primary aim being the reverse-engineering of language models (LMs) from an in-depth perspective~\cite{olah2022}.
Previously identified model-agnostic techniques have been proven to offer limited insight for the transformer architecture~\cite{neely2022,pruthi2022,bibal2022,krishna2024}, whereas MI embraces the opposite philosophy by removing model abstractions and analyzing LMs in \todo{terms} of their components and how their interactions.

Mechanistic interpretability was initially mentioned as being the main driving ideology behind the `transformer circuits thread'~\cite{elhage2021}.
Nonetheless, by following~\citet{rai2024} approach, we can observe that the concept of MI is not limited to the application of circuits.
Envisioning MI as being characterized by a general bottom-up approach for interpreting LMs, its interpretation can be extended to include some \todo{preceding} techniques such as the logit lens~\cite{nostalgebraist2020} and other probing approaches.

Another possible taxonomy for interpretability techniques, more focused on their nature rather than their use, is presented by~\citet{ferrando2024}.
They identify two main classes of interpretability approaches: \textbf{behavior localization} and \textbf{information decoding}.
In the next sections we will follow their insightful classification to provide a synthetic analysis of the state of the art, with a specific focus on a restricted number of techniques that are especially relevant for the purpose of this work.

\begin{figure}[H]
    \centering
    \includegraphics[width=0.6\textwidth]{related_ferrando-tax.pdf}
    \caption{Taxonomy for transformer interpretability methods proposed by~\citet{ferrando2024}.}
    \label{fig:related_ferrando-tax}
\end{figure}
\todo[cyan]{Remove non-taxonomy sections from image}
\todo[cyan]{Consider replacing with a handmade diagram}
\todo[gray]{V: Image too small}

\section{Behavior localization}

Behavior localization techniques consist in the localization elements inside language models that are responsible for specific predictions or certain prediction patterns.
It is a generally broad task, but an important distinction can be made between the localization of behaviors towards input features (\textbf{input attribution}) and towards model components (\textbf{model component attribution})~\cite{ferrando2024}. 

\subsection{Input attribution}

In the \textbf{input attribution} case, the model's predictions are directly traced back to the inputs via some kind of attribution mechanism.
The two main input attribution \todo{pathways} are either gradients~\cite{denil2014, ding2021, sanyal2021, enguehard2023} or perturbations~\cite{li2016, amara2024, mohebbi2023}.
In both cases the great majority of techniques was directly influenced by model-agnostic approaches~\cite{sundararajan2017, smilkov2017, ribeiro2016, lundberg2017} that were initially studied and applied in the context of deep learning.

More recent input attribution techniques experimented with the aggregation of intermediate information to provide token-wise attributions exploiting context mixing properties of transformers~\cite{ferrando2022, modarressi2022, mohebbi2023}, while other approaches focused on providing counterfactual explanations based on contrastive gradient attributions~\cite{yin2022} or studying specific training examples to model their influence on model predictions~\cite{kwon2024, grosse2023}.
It is important to note that, through the years, some critiques have been moved towards input attribution methods, mainly concerning their limited reliability~\cite{sixt2019, adebayo2018, atanasova2020}.

\subsection{Model component attribution}

In \textbf{model component attribution}, the main research focus shifts towards analyzing the effects of individual or groups of transformer components, such as attention heads, feedforward layers, and neurons.
This shift is principally motivated by the inherent sparsity of LMs, where only a subset of the model's parameters significantly contributes to its predictions~\cite{zhao2021}.
By isolating and understanding the effects of these key components, it is possible to shed light on their contribution to the actual model's prediction.
We can identify three main distinct approaches for model component attribution: \textbf{logit attribution}, \textbf{causal interventions} and \textbf{circuit analysis}

\textbf{Logit attribution} is based upon the concept of direct logit attribution (DLA), a metric specifically devised to measure the contribution of a certain component $c$ to the logit of the output token $w$ exploiting the inherent linearity of the transformer model's components.
Some variation on this idea enabled the computation of the logit attribution metric in more specialized cases.
For example:~\citet{geva2022} managed to measure the DLA of each FFN neuron,~\citet{ferrando2023} identified an alternative to measure the DLA of each path involving a certain attention head, and~\citet{wang2023} proposed the direct logit difference attribution (DLDA) using the logit difference (LD) as a comparative mean to measure contrastive attribution.

\textbf{Causal interventions} approaches are centered around the interpretation of the LM as a causal model~\cite{geiger2021,mcgrath2023}, which takes the form of a directed acyclic graph (DAG) having model computations as nodes and activations as edges.
The primary purpose of this representation is to enable specific interventions (known as activation patching or causal tracing) directly on the model's components, allowing for comparisons of different computational outcomes.
There are three main choices which influence the result of causal intervention: choice of model component to patch, patching function and evaluation metric.
Different authors have suggested a variety of possible patched activation functions that accomplish different goals and have different uses.
There have been cases of null vectors being used as patched activations (zero intervention)~\cite{olsson2022, mohebbi2023}, noise being added to the input of the component (noise intervention)~\cite{meng2022} and counterfactual data being fed to the component either by sampling (resample intervention)~\cite{hanna2023, conmy2023} or averaging (mean intervention)~\cite{wang2023}.
\citet{zhang2024} provide an insightful overview for common practices of activation patching in language models, identifying KL divergence, probability and logit difference as common evaluation metrics.
Additionally, it is possible to identify an alternate `denoising' setup, which subverts the classic activation patching operation by applying a patched activation from a clean run to a corrupted one~\cite{lieberum2023, meng2022}.

\todo[green]{important, expand meng2022}
\todo[green]{Locating and Editing Factual Associations in GPT}
\todo[green]{Transformer Feed-Forward Layers Are Key-Value Memories}
\todo[green]{How Pre-trained Language Models Capture Factual Knowledge? A Causal-Inspired Analysis}

\subsubsection{Circuit analysis}

\textbf{Circuit analysis} is closely related to the mechanistic interpretability (MI) subfield analyzed previously as its main goal is tied to the discovery of circuits inside LMs.
Circuits are subsets of model components that can be seen as acting independently while carrying out a specific task, and can possibly be synthesized into an algorithm.
Despite their successful application on LMs, circuits were not originally identified with the transformer architecture in mind; in fact, their first application was on vision models~\cite{cammarata2020}.
Most of the initial work regarding transformer circuits was performed on publications belonging to the `transformer circuits thread'~\cite{elhage2021,olsson2022,elhage2022,bricken2023}, heavily inspired by the preceding vision counterpart~\cite{cammarata2020}.
By applying the circuit concept to the previously causal intervention techniques, we can extend the idea of activation patching to edge patching and path patching: novel circuits-based techniques that take into account the interactions between model components.
Edge patching~\cite{li2023} considers edges that directly connect pairs of model components due to the fact that each component input can be modeled as the sum of the outputs of the previous model components inside the residual stream, while path patching~\cite{wang2023} is a generalization of edge patching to multiple edges.

\todo[green]{important, expand circuits}
\todo[green]{A Mathematical Framework for Transformer Circuits}
\todo[green]{Softmax Linear Units}
\todo[green]{Toy Models of Superposition}

\section{Information decoding}

\textbf{Information decoding} takes a step back from behavior localization techniques by focusing on the extraction of single pieces of information from model components, rather than trying to explain entire predictions by attributing them to various internal mechanisms.
These pieces of information take the name of features (or concepts) and are commonly characterized by being human interpretable properties of the input~\cite{kim2018}.
The three main categories that can be identified in this approach consist of \textbf{probing} which can be seen as the LM adaptation of a popular technique in deep learning, a broader categorization named \textbf{sparse autoencoders} that includes the application of sparse autoencoders following the linear representation hypothesis, and \textbf{vocabulary space decoding} which tackles the representation of models' representations using vocabulary tokens.

\textbf{Probing} techniques are used to analyze the inner workings of LMs and, more generally, any kind of deep neural network.
Probing usually implies the supervised training of ad-hoc models (often classifiers) to interpret the features present in the intermediate representations of the main model.
The probing classifier is specifically trained to evaluate how much information about a certain property is encoded inside an intermediate representation.
While the actual property that the probe seeks out often depends on the purpose of the analysis, some critics have \todo{been moved out towards} the limitations of probing classifiers~\cite{belinkov2022}.
Particular attention has been put towards probing transformer models~\cite{chwang2024, zou2023, macdiarmid2024, burns2023}, especially the family of encoder-only models related to BERT~\cite{devlin2019}.
Some exceptional results include the discovery of syntactic information inside the hidden representations of BERT models~\cite{tenney2019a, lin2019, liu2019}, even to the extent of uncovering entire syntax trees~\cite{hewitt2019} and hierarchical computation structures along the residual stream, reminiscent of classical NLP pipelines~\cite{tenney2019b}.

\todo[green]{important, expand probing}
\todo[green]{Factual Probing Is [MASK]: Learning vs. Learning to Recall}

The \textbf{linear representation hypothesis}~\cite{park2023} is a theory that assumes a linear representation for high-level concepts inside the representation space of a model.
The central idea for this hypothesis is based upon the early discoveries of linearity inside the embedding space done by~\citet{mikolov2013}, as the resulting analogies and geometric properties are direct consequence of a linear embedding space.
Recent work has uncovered many instances of FFN neurons that consistently fire with specific patterns \todo{traceable} to input features~\cite{voita2024}, suggesting that this behavior is an effect of the next token prediction training paradigm~\cite{jiang2024}.
Additionally, there have been many attempts aimed at modifying the internal representations of a model by exploiting their linear properties.
These types of linear interventions resulted successful in erasing concepts and features from intermediate model representations~\cite{ravfogel2020, ravfogel2022, belrose2023b}, and even meaningfully changing the model's behavior~\cite{nanda2023, belrose2023b}, opening up new \todo{venues} for model steering and alignment. 
Another important aspect of the linear representation hypothesis is the presence of polysemanticity and superposition in the identified features.
The effects of information compression performed by dimensionality reduction algorithms resulting in distributed representations has widely been observed and studied in many fields, however Olah makes an important distinction between the separate phenomena of composition and superposition~\cite{olah2023}.
Many extend these observations to actual experiments, successfully proving the existence of superposition both in simplified scenarios~\cite{elhage2022, arora2018} and in the early layers of transformer-based LMs~\cite{gurnee2023}.

\todo[green]{important, expand linear representation hypothesis}
\todo[green]{All Bark and No Bite: Rogue Dimensions in Transformer Language Models Obscure Representational Quality}
\todo[green]{Do Llamas Work in English? On the Latent Language of Multilingual Transformers}

Autoencoders with sparsity regularization, also known as \textbf{Sparse autoencoders} (SAEs), have been extensively used to reconstruct the internal representation of neural networks that exhibit superposition by finding an overcomplete feature basis via dictionary learning and promoting feature sparsity~\cite{bricken2023, huben2024}.

\subsubsection{Vocabulary space decoding}

One of the most direct methods to comprehend a model's hidden representations is by employing its own vocabulary to derive plausible interpretations. \textbf{Vocabulary space decoding} techniques are founded on this principle, by utilizing the model's existing vocabulary they can generate outputs that are immediately understandable and may unveil hidden patterns inside the model's generation process.

The first real implementation of vocabulary space decoding is the logit lens~\cite{nostalgebraist2020}, which proposed the decoding of interlayer hidden representation using the model's own unembedding matrix following the intuition of an iterative refining of the model's prediction throughout the forward pass~\cite{jastrzebski2018}.
The contribution of the logit lens was groundbreaking and, despite some acknowledged shortcomings by the author, inspired numerous similar techniques aimed at improving its design or offering alternative functionalities.
Some significant advancements include the introduction of translators, which act as probing classifiers to enhance the logit lens' predictions by applying either linear mappings~\cite{din2024} or affine transformations~\cite{belrose2023a}.
Additionally, the attention lens~\cite{sakarvadia2023} applies the concepts of the logit lens and translators to the outputs of attention heads, while the future lens~\cite{pal2023} extends logit lens predictions to also include the next most probable tokens by exploiting causal intervention methods.
Another crucial contribution, inspired by the future lens, is the \emph{patchscopes} framework~\cite{ghandeharioun2024}, which aims to generalize all prior interpretability methods based on vocabulary space decoding and causal interventions.
Other significant approaches include the direct decoding of model weights~\cite{dar2023}, potentially using singular value decomposition techniques to factorize the weight matrices~\cite{millidge2022}, and logit spectroscopy~\cite{cancedda2024}, which employs a spectral analysis of the residual stream and parameter matrices interacting with it.
This last method aims to identify and analyze specific parts of the hidden representation spectrum that are most likely to be overlooked by the classic logit lens.

\begin{figure}[H]
    \centering
    \includegraphics[width=0.8\textwidth]{related_patchscopes.pdf}
    \caption{\todo[red]{placeholder caption: Patchscopes Visualization.}}
    \label{fig:related_patchscopes}
\end{figure}

\todo[green]{important, expand logit-lens approaches}
\todo[green]{Interpreting GPT: the logit lens}
\todo[green]{Eliciting Latent Predictions from Transformers with the Tuned Lens}

Other unrelated approaches based on vocabulary space decoding involve using maximally-activating inputs to explain the behavior of units and neurons that exhibit significant responses to specific features~\cite{dalvi2019}.
Additionally, other LMs have been used as zero-shot explainers to provide insights into possible shared features between input sequences that cause substantial activations of specific neurons in the target model~\cite{bills2023}.
Unfortunately, the maximally-activating input analysis has been criticized for generating false positives~\cite{bolukbasi2021}, while the elicitation of natural language explanations from LMs approach has faced criticism for its general lack of causal influence between the identified concept-neuron pairs~\cite{huang2023}.

\todo[green]{Add that one paper (LM-TT) with a very nice, comprehensive and interactive transformer execution visualization (maybe out of time scope?)}

\begin{figure}[H]
    \centering
    \includegraphics[width=\textwidth]{related_lm-tt.pdf}
    \caption{\todo[red]{placeholder caption: Visualization of That One Paper (LM-TT).}}
    \label{fig:related_lm-tt}
\end{figure}

\chapter{Research questions}\label{ch:research_questions}
In this section, we delineate the specific research questions regarding transformer interpretability that will be addressed in the present work.
The purpose of these research questions is to provide a comprehensive informative analysis on the hidden representations and embedding space of recent transformer-based models.
The novelty of this work does not necessarily lie in the chosen approach itself, but resides in the integration of several established techniques to offer fresh perspectives and interpretability tools, ultimately aimed at confirming and possibly expanding upon previous findings.

\section{How Can We Visualize Internal States \texorpdfstring{ \\ }{} of LLMs in an Immediate and \texorpdfstring{ \\ }{} Interactive Way?}\label{sec:rq_intravisto}

The first research question focuses on the development of an interactive visualization tool for LLMs.
The goal is to create a tool that allows users to visually explore and interpret the complete computational process performed by an LLM when generating a sentence.
The chosen approach is centered around the concept of vocabulary space decoding and follows the direction of the already mentioned logit lens~\cite{nostalgebraist2020}.
Additionally, novel methods will be introduced in the context of hidden representation decoding and visualization.
One of the primary objectives of the proposed tool is to provide an intuitive visualization of the residual stream, enabling the observation of the influence pathway composed of residuals and attention, cumulating into each token prediction.

The development of this transformer visualization framework is motivated by a comprehensive revisitation of the logit lens analysis from a new, multifaceted perspective, with the ultimate goal of confirming established results from a fresh angle and potentially uncovering new phenomena.
This framework serves as a foundation for many of the inquiries tackled in this work, as it provides the means for a preliminary analysis needed to initiate other, more focused experiments.

\section{Do Autoregressive LLMs Retain Linear Properties in Their Embedding Spaces?}\label{sec:rq_embeddings}

With the aid of the previously developed tool, we observed some peculiarities in the behavior of LLMs compared to smaller, less sophisticated language models.
In particular, the embedding representations of LLMs appeared to encode less information about actual word semantics.
Building upon this observation, the second research question examines whether autoregressive LLMs retain linear properties in their embedding spaces.

As discussed in previous chapters, linearity in the embedding space is a well-established property in both the original Word2Vec embedding space~\cite{mikolov2013} and in transformer-based encoder models such as BERT~\cite{devlin2019}.
A linear embedding space enables certain geometric properties, such as solving word analogies (`king' - `man' + `woman' = `queen') and modeling semantic similarity between tokens as a function of distance in the embedding space.
Given recent advances in transformer training and architecture, we aim to assess whether these properties persist in the embedding spaces of contemporary causal LLMs and whether they are consistent across various model architectures and sizes, while also highlighting differences between the embedding and unembedding spaces in models where they differ.

\section{Is it Possible to Obtain First-order \texorpdfstring{ \\ }{} \mbox{Predictions} by \mbox{Concatenating} \texorpdfstring{ \\ }{} Embedding Spaces in LLMs?}\label{sec:rq_fom}

Following the investigation on the differences between input and output embedding weights in language models, the third research question explores the feasibility of obtaining first-order predictions by concatenating the two embedding spaces in LLMs.
A first-order prediction can be understood as an estimate of the next token based solely on the previous one, similar to the predictive process of a first-order Markov model, which employs a probability conditioned on the previous token to determine the most likely next token.
As observed by~\citet{elhage2021}, indeed concatenating the input and output embeddings of a language model should theoretically yield the equivalent of a first-order Markov model; however, is that always the case?
We test this hypothesis on new state-of-the-art models and provide novel insights into the role of different embedding spaces in recent LLMs.
Moreover, this series of experiments should hopefully offer further insights into the practice of \emph{weight tying}~\cite{inan2017,press2017} applied to the embedding layers of language models.


\chapter{Methodology}\label{ch:methodology}
In this section we will explore the theoretical fundaments of our experiments.
To do so, we are going to subdivide our analysis into three main sections, where each one is paired with one of the previously identified research questions.

\section{Transformer Visualization}\label{sec:method_intravisto}

The first section is dedicated to the exploration of the main features pertaining the proposed interactive tool for the exploration of autoregressive transformer architectures, \emph{InTraVisTo} (Inside Transformer Visualization Tool).

\emph{InTraVisTo} is an open-source visualization tool depicting the internal computations performed within a Transformer.
The tool provides visualizations of both the internal state of the LLM, using a heatmap of decoded embedding vectors for all layer/token positions, and the information flow between components of the LLM, using a Sankey diagram to depict paths through which information accumulates to produce next-token predictions.

\subsection{Decoding Internal States}\label{ssec:method_intravisto_decoding}

In this context, with the expression \emph{`decoding process'} we refer to the \emph{vocabulary decoding} process, which consists in the conversion of transformers hidden states, represented by vectors in a high-dimensional space, into human-readable content.
InTraVisTo enables the decoding and inspection of the main four vectors produced by each layer $\ell$ of a transformer:
\begin{itemize}
    \item $\gbm{\delta}_\textit{att}^{(\ell)}$ represents the output of the attention component.
    \item ${\gbm{x}'}^{(\ell)}$ is the \emph{intermediate state}, given by the addition of $\gbm{\delta}_\textit{att}^{(\ell)}$ to the residual stream.
    \item $\gbm{\delta}_\textit{ff}^{(\ell)}$ represents the output of the feedforward network component.
    \item $\gbm{x}^{(\ell)}$ is the \emph{layer output}, which can be seen as the residual stream with the contributions of both $\gbm{\delta}_\textit{ff}^{(\ell)}$ and $\gbm{\delta}_\textit{att}^{(\ell)}$.
\end{itemize}

Decoding the meaning of hidden state vectors at various depths of a transformer stack is essential for providing an intuition as to how the model is working.
We pose our focus on causal models, and assume the state-of-the-art LLM architecture, which involves a continuous \emph{decoration process} where each transformer layer adds the results of its computations to a residual embedding vector from the layer below.
InTraVisTo provides a human-interpretable representation of this internal decoration pipeline by decoding each hidden state with a specific decoder and displaying the most likely token from the model's vocabulary.

\subsubsection{Decoders and Normalization}\label{sssec:method_intravisto_decoding_norm}

As for the decoding process, we compute the probability distribution over the vocabulary space for a hidden state $\gbm{x}$ in the following way:
\begin{equation}
    \label{eq:method_intravisto_decoding}
    P(\ \cdot \mid \gbm{x}, d_\textit{dec}, n_\textit{norm}) = P(\ \cdot \mid \gbm{x}, \gbm{W}_{d_\textit{dec}}, N_{n_\textit{norm}}) = \operatorname{softmax}\Bigl(N_{n_\textit{norm}}(\gbm{x}) \cdot \gbm{W}_{d_\textit{dec}}\Bigr)
\end{equation}
Where $\gbm{W}_{d_\textit{dec}}$ represents the matrix of decoder weights according to the user decoder choice $d_\textit{dec}$, and $N_{n_\textit{norm}}$ is used to identify the normalization operation selected by the user through $n_\textit{norm}$:
\begin{equation}
    N_{n_\textit{norm}}(\gbm{x}) = 
    \label{eq:method_intravisto_normalization}
    \left\{
    \begin{array}{cl}
        \gbm{x} &\ \text{if}\ n_\textit{norm} = \text{`no normalization'} \\
        \mathcal{N}{(\gbm{x})} &\ \text{if}\ n_\textit{norm} = \text{`normalize only'} \\
        \gbm{\gamma}_\ell \cdot \mathcal{N}{(\gbm{x})} + \gbm{\beta}_\ell &\ \text{if}\ n_\textit{norm} = \text{`normalize and scale'}
    \end{array}
    \right.
\end{equation}
Considering $\mathcal{N}{(\gbm{x})}$ the normalization component of the model's final normalization layer, being reliant on $\mu$ and $\sigma$ for LayerNorm implementations as defined in~\cref{eq:background_layernorm,eq:background_layernorm_extra}, and $RMS(\gbm{x})$ for RMSNorm implementations as defined in~\cref{eq:background_rmsnorm}

Two natural choices of decoders to use are the transpose of the \emph{input embedding matrix} $\gbm{W}_\textit{in}^\T$ used by the model to convert tokens to vectors on input, and the \emph{output decoder} $\gbm{W}_\textit{out}$ used upon output within the language modeling head.
Some models, like GPT-2~\cite{radford2019} and Gemma~\cite{mesnard2024,rivi2024} tie these two parameter matrices together during training, while other popular models such as Mistral~\cite{jiang2023} and Llama~\cite{touvron2023,dubey2024} allow these two matrices to differ.
This structural weight difference will be analyzed more in depth in later sections, \todo{however,} in our current scope having different weight matrices for embedding and unembedding ($\gbm{W}_\textit{in}^\T \neq \gbm{W}_\textit{out}$) implies that earlier layers tend to be much more interpretable when decoded with the input embedding ($\gbm{W}_\textit{in}^\T$) while latter layers are more meaningful if the output decoder ($\gbm{W}_\textit{out}$) is used.

Previous work has looked to \emph{train specialized decoders}~\cite{belrose2023a,sakarvadia2023,pal2023} for generating meaningful vocabulary distributions at any point in a model, at the cost of introducing a great deal of additional complexity and potential errors.
InTraVisTo employs a simpler and elegant alternative, by \emph{interpolating} the input and output decoders based on the depth $\ell\in\{0,\ldots,L\}$ of the model layer we wish to decode, we obtain a `hybrid' decoding weight matrix that acts as an equilibrium point calibrated on the current model depth.
We define various alternatives for decoder interpolation mainly focusing on \emph{linear interpolation}, \emph{quadratic interpolation} and \emph{max-probability interpolation}, defined as follows:
\begin{subequations}
    \begin{align}
        \gbm{W}_\textit{linear}^{(\ell)} &=\left(1-\frac{\ell}{L}\right) \cdot \gbm{W}_\textit{in}^\T + \frac{\ell}{L} \cdot \gbm{W}_\textit{out} \label{eq:method_intravisto_linear-interp} \\
        \gbm{W}_\textit{quadratic}^{(\ell)} &=\left(1-\left(\frac{\ell}{L}\right)^2\right) \cdot \gbm{W}_\textit{in}^\T + \left(\frac{\ell}{L}\right)^2 \cdot \gbm{W}_\textit{out} \label{eq:method_intravisto_quadratic-interp} \\
        \gbm{W}_\textit{max\_p} &=\operatornamewithlimits{argmax}_{\gbm{W} \in \{\gbm{W}_\textit{in}^\T, \gbm{W}_\textit{out}\}} \operatornamewithlimits{max}_{v \in \mathcal{V}} P(v \mid \gbm{x}, \gbm{W}, N_{n_\textit{norm}}) \label{eq:method_intravisto_max-p}
    \end{align}
\end{subequations}

Any of the matrices $\gbm{W}_\textit{in}^\T$, $\gbm{W}_\textit{out}$, $\gbm{W}_\textit{linear}$, $\gbm{W}_\textit{quadratic}$ and $\gbm{W}_\textit{max\_p}$ can be used as the decoder matrix $\gbm{W}_{d_\textit{dec}}$ to decode an embedding $\gbm{x}$ into a probability distribution over $\mathcal{V}$ as described in~\cref{eq:method_intravisto_decoding}.
As previously mentioned, this behavior selection is controlled by specifying $d_\textit{dec}$.
It is important to note that the term `quadratic interpolation' can be misleading, as the interpolation between decoding matrices is still linear and \todo{it is the} layer index ratio $\frac{\ell}{L}$ that scales quadratically through the models' layers.

\subsubsection{Secondary Tokens}\label{sssec:method_intravisto_decoding_tokens}
    
As previously illustrated, each hidden representation is decoded by performing a normalization operation first, to then multiply the result for the chosen decoding matrix, and finally obtain a probability distribution over the model's vocabulary by applying a softmax function on the resulting logits.
At this point we utilize the same \emph{sampling policy} of the model in order to extract the `predicted' vocabulary token ID from the computed distribution.
We assume the policy to always be \emph{greedy decoding}, both as a simplifying factor and to reflect the default settings for causal models provided by Hugging Face's transformers library~\cite{wolf2020}.

Nevertheless, our interest in decoding the model's hypothetical intermediate predictions does not only extend to the first token, as important information that cannot be condensed into a single vocabulary token is usually withheld inside the hidden representation~\cite{elhage2022,henighan2023,elhage2023}.
In order to extract this `leftover information', we devise two main approaches that \todo{sublimate} it into \emph{secondary tokens}.
We define \emph{secondary tokens} as additional vocabulary tokens that are the result of a \emph{secondary decoding process}, aimed at obtaining tokens that hold less importance than the token obtained as a result of the \emph{primary decoding process} (as mentioned before) of the same hidden representation.

The first secondary decoding approach is \emph{Top-$k$ probability decoding} and consists of expanding the number of selected tokens from the probability distribution to $k$, thus obtaining $k-1$ secondary tokens ordered by their probability values.
This is a rather simple and immediate technique, which is widely-used in many interpretability applications \todo{and non}~\cite{belrose2023a,pal2023,tufanov2024}.
The principal downside of this approach comes from the fact that the obtained secondary tokens might be overly similar to the primary token, resulting in redundant information.
This is likely caused by the fact that a large part of the hidden representation is used to represent the primary token, often skewing the embedding vector in favor of tokens that are semantically similar~\cite{elhage2022}.

To alleviate this issue we propose a novel way to extract secondary representations from a hidden representation: \emph{iterative decoding}.
The rationale behind the proposed approach is that hidden representations contain an overlap of concepts in an embedding space that is loosely related to both the input and output embedding spaces.
As a consequence, we postulate that hidden representations existing in these intermediate embedding spaces should retain the linear properties that have been ascertained to exist in the input and output embedding spaces of transformer architectures~\cite{mikolov2013,park2023}.
Iterative decoding exploits these linear properties by performing a sequence of subtractions from the main hidden representation, removing the embedding of the most probable representation during each iteration.

\begin{algorithm}
    \caption{Iterative decoding algorithm.}\label{alg:method_intravisto_iter-dec}
    \begin{algorithmic}
        \STATE{$tokens \gets \{\}$}
        \STATE{$norms \gets \{\}$}
        \STATE{$i \gets 0$}
        \WHILE{$i < {rep}_{max}$}
            \STATE{$id \gets \arg \max\{decode(emb)\}$}
            \STATE{${emb}_{real} \gets {(\gbm{W}_{d_{dec}})}_{id,\cdot}$}
            \IF{$\|emb\| \leq {norm}_{min} \vee \bigl( |norms| > 0 \wedge \|emb\| \geq norms\bigl[i-1\bigr] \wedge i \neq 0 \bigr)$}
                \STATE{\textbf{break}}
            \ENDIF{}
            \IF{$id \notin tokens$}
                \STATE{$tokens\bigl[|tokens|-1\bigr] \gets id$}
            \ENDIF{}
            \STATE{$norms\bigl[i\bigr] \gets \|emb\|$}
            \STATE{$emb \gets emb - {emb}_{real}$}
        \ENDWHILE{}
        \RETURN tokens
    \end{algorithmic}
\end{algorithm}

As it is possible to observe in~\cref{alg:method_intravisto_iter-dec}, we perform at most ${rep}_{max}$ iterations obtaining one primary token and between $0$ to ${rep}_{max} - 1$ secondary tokens.
This is due to the fact that a secondary token that is found in two separate iterations is recorded only on the first one, and the presence of a stopping condition that triggers in case the norm of the resulting hidden representation is under a certain threshold ${norm}_{min}$ or is higher than the norm found at the previous iteration.
This last condition is used to avoid situations where the embedding vector of the hidden state `flips' after the computation of the difference with the most probable representation, resulting in a vector that is not informative and can possibly reiterate the effect until ${rep}_{max}$ is reached, thus generating predictions based purely on noise.

One downside of this approach, besides the linearity assumption of the intermediate embedding spaces \todo{which is based upon}, is the fact that there exists a notable shift in representation magnitudes throughout the layers of most state-of-the-art transformer models~\cite{heimersheim2023}.
This typically results in a steady increase in the norms of the embedding vectors, proportional to the layer number.
While this has only a marginal impact on the decoding process, it can significantly disrupt the embedding vector subtraction operation within the iterative decoding approach, as the quantities involved may differ in magnitude.

\subsubsection{Decoding Metrics}\label{sssec:method_intravisto_decoding_metrics}

Other meaningful quantities that are shown through the InTraVisTo visualization are tied to the actual probability distributions obtained through the decoding process.
The first option is \emph{``P(argmax term)''}, which directly translates into the probability of the most probable token output from the chosen decoder. % chktex 36
It gives an immediate idea of how much the model is sure about the token that has been greedily sampled.
On the other hand, a complementary measure is the \emph{entropy} of the probability distribution over the vocabulary space, which the higher it is, the more unsure the model is of the next token.
Entropy for an embedding $\gbm{x}$ is computed in the following way:
\begin{equation*}
    \label{eq:method_intravisto_entropy}
    H_{\gbm{x}} = -\sum{P_{\gbm{x}} \cdot \log{P_{\gbm{x}}}}
\end{equation*}
Where $P_{\gbm{x}} = P(\ \cdot \mid \gbm{x}, d_\textit{dec}, n_\textit{norm})$ as computed in~\cref{eq:method_intravisto_decoding}.

Other showcased metrics are the \emph{attention contribution} and \emph{feedforward contribution}, which measure respectively how much the output of the attention block, or feed forward, contributes in its summation with the residual stream of a transformer block.
The purpose of these metrics is to highlight where the main information of that block is coming from, whether from the attention or feed forward components.
We devised two main approaches to weigh the contribution of each component to the residual stream: one uses the \emph{norms} of hidden state vectors to compare the magnitude of their contribution, while the other uses the \emph{KL divergence} to compare the probability distribution similarity of the two hidden states against the final one.
In practice, we compute:
\begin{equation}
    \left\{
    \begin{aligned}
        &{\%}^{(\ell)}_{\textit{norm},\textit{att}} = \frac{\|\gbm{\delta}_\textit{att}^{(\ell)}\|_2}{\|\gbm{\delta}_\textit{att}^{(\ell)}\|_2 + \|\gbm{x}^{(\ell-1)}\|_2} \\
        &{\%}^{(\ell)}_{\textit{norm},\textit{ff}} = \frac{\|\gbm{\delta}_\textit{ff}^{(\ell)}\|_2}{\|\gbm{\delta}_\textit{ff}^{(\ell)}\|_2 + \|{\gbm{x}'}^{(\ell)}\|_2} \label{eq:method_intravisto_norm-contrib}
    \end{aligned}
    \right.
\end{equation}
\begin{equation}
    \left\{
    \begin{aligned}
        &{\%}^{(\ell)}_{\textit{KL},\textit{att}} = \frac{D_{\text{KL}}(\gbm{x}^{(\ell-1)} \parallel {\gbm{x}'}^{(\ell)})}{D_{\text{KL}}(\gbm{\delta}_\textit{att}^{(\ell)} \parallel {\gbm{x}'}^{(\ell)}) + D_{\text{KL}}(\gbm{x}^{(\ell-1)} \parallel {\gbm{x}'}^{(\ell)})} \\
        &{\%}^{(\ell)}_{\textit{KL},\textit{ff}} = \frac{D_{\text{KL}}({\gbm{x}'}^{(\ell)} \parallel \gbm{x}^{(\ell)})}{D_{\text{KL}}(\gbm{\delta}_\textit{ff}^{(\ell)} \parallel \gbm{x}^{(\ell)}) + D_{\text{KL}}({\gbm{x}'}^{(\ell)} \parallel \gbm{x}^{(\ell)})} \label{eq:method_intravisto_kl-contrib}
    \end{aligned}
    \right.
\end{equation}
Where $D_{\text{KL}}$ is the Kullback-Liebler divergence between two distributions, and the notation of hidden states (such as $\gbm{x}^{(\ell)}$, $\gbm{\delta}_\textit{att}^{(\ell)}$, \ldots) references the distinctions made at~\cref{ssec:method_intravisto_decoding}.
It is possible to note that the contributions computed through norms and KL divergence feature opposite terms at the fractional numerator, this is due to their inverse relationship as the KL divergence measures the \emph{dissimilarity} between probability distributions, while the norm can be directly translated into the positive contribution of a hidden state. 

\subsection{Flow}\label{ssec:method_intravisto_flow}

The second visualization introduced in InTraVisTo is a \emph{Sankey diagram} that aims to depict the information flow through the transformer network.
Edges in the diagram indicate the amount of influence that the nodes have on each other and show how the information accumulates from the bottom of the diagram to the top in order to generate the final prediction.
The flow snakes its way through self-attention nodes, which combine information from attended tokens in the level below, feed-forward networks nodes, which introduce information based on detected patterns in the state vector, and aggregation nodes, where updates from the other two types of nodes are added to the residual vector.
The proposed Sankey diagram, technically qualifies as an \emph{Alluvial diagram} due to the fact that nodes are grouped into `steps' (in our case corresponding to layers), which provide intermediate subdivisions of the flow mass, and help to \todo{focalize attention} on the changes in flow composition throughout the model's layers.

In order to calculate the information flow, an attribution algorithm works backwards from the top layers of the network, recursively apportioning the incident flow from the components below based on their relative contributions to the internal state vector above using~\cref{eq:method_intravisto_norm-contrib,eq:method_intravisto_kl-contrib}.
The flow's constantly updating state can be defined in a way that reflects the recursive computations performed by InTraVisTo:
\begin{equation}
    \label{eq:method_intravisto_flows}
    \left\{
    \begin{alignedat}{2}
        &\textit{flow}_{\textit{ffnn}}^{(\ell,j)} &&= {\%}_{\textit{ffnn}}^{(\ell,j)} \cdot \textit{flow}_{x}^{(\ell,j)} \\
        &\textit{flow}_{x'}^{(\ell,j)} &&= \textit{flow}_{\textit{ffnn}}^{(\ell,j)} + (1 - {\%}_{\textit{ffnn}}^{(\ell,j)}) \cdot \textit{flow}_{\textit{x}}^{(\ell,j)} = \textit{flow}_{\textit{x}}^{(\ell,j)} \\
        &\textit{flow}_{\textit{att}}^{(\ell,j)} &&= {\%}_{\textit{att}}^{(\ell,j)} \cdot \textit{flow}_{x'}^{(\ell,j)} = {\%}_{\textit{att}}^{(\ell,j)} \cdot \textit{flow}_{x}^{(\ell,j)} \\
        &\textit{flow}_{x}^{(\ell-1,j)} &&= \sum_{i\in\{j,\ldots,k\}}{\overline{\textit{attend}{\,}}^{(\ell,i)}\bigl[j\bigr]}\cdot\textit{flow}_{\textit{att}}^{(\ell,i)} + ( 1 - {\%}_{\textit{att}}^{(\ell,j)})\cdot \textit{flow}_{\textit{x'}}^{(\ell,j)} \\
            &\quad &&= \biggl(\Bigl(\sum_{i\in\{j,\ldots,k\}}\overline{\textit{attend}{\,}}^{(\ell,i)}\bigl[j\bigr] - 1\Bigr)\cdot{\%}_{\textit{att}}^{(\ell,j)} + 1\biggr) \cdot \textit{flow}_{\textit{x}}^{(\ell,j)}
    \end{alignedat}
    \right.
\end{equation}
Where $\overline{\textit{attend}{\,}}^{(\ell,i)}$ denotes the average attention placed on token $j$ by the attention heads present at position $i$ of layer $\ell$.
This quantity is used to compute all outgoing contributions of a token $j$ to subsequent self-attention nodes, thus it is computed considering tokens between $j$ and $k$ as the attention sources, where $k$ denotes the index of the last token in the generated sentence.

Another type of information shown in the Sankey diagram concerns the computation of differences between residual representations, with the goal of visualizing the flow's `evolution' throughout the model's layers.
To achieve this objective we devised two main approaches that we implemented to explore this \todo{visualization dimension}.
The first approach consists \todo{of} computing the KL divergence between the probability distributions of hidden states belonging to each combination of component outputs and the residual stream state.
In practice, we compute the following quantity for five different state combinations:
\begin{equation}
    \label{eq:method_intravisto_kl-diff}
    \begin{gathered}
        {\textit{kl\_diff}{\,}\strut}_{\gbm{x}_a, \gbm{x}_b}^{(\ell)} = D_{\text{KL}}(P_{\gbm{x}_b} \parallel P_{\gbm{x}_a}) \\
        \text{for} \ (\gbm{x}_a, \gbm{x}_b) \in \Bigl\{
            (\gbm{x}^{(\ell-1)}, {\gbm{x}'}^{(\ell)}), 
            (\gbm{\delta}_\textit{att}^{(\ell)}, {\gbm{x}'}^{(\ell)}), 
            ({\gbm{x}'}^{(\ell)}, \gbm{\delta}_\textit{ffnn}^{(\ell)}), 
            ({\gbm{x}'}^{(\ell)}, \gbm{x}^{(\ell)}), 
            (\gbm{\delta}_\textit{ffnn}^{(\ell)}, \gbm{x}^{(\ell)})
        \Bigr\}
    \end{gathered}
\end{equation}
Where $P_{\gbm{x}} = P(\ \cdot \mid \gbm{x}, d_\textit{dec}, n_\textit{norm})$ as computed in~\cref{eq:method_intravisto_decoding}, $D_{\text{KL}}$ is the Kullback-Liebler divergence between two distributions, and the notation of hidden states (such as $\gbm{x}^{(\ell)}$, $\gbm{\delta}_\textit{att}^{(\ell)}$, \ldots) references the distinctions made at~\cref{ssec:method_intravisto_decoding}.
Whereas, the second approach compares the residual stream states at the input and output location for each layer in the transformer stack.
It does so by calculating the difference between the two hidden states, and performing the decoding operation defined in~\cref{eq:method_intravisto_decoding} as to generate primary and secondary tokens for the resulting quantity.
This is showcased in the following computation:
\begin{equation}
    \label{eq:method_intravisto_state-diff}
    {\textit{state\_diff}{\,}\strut}^{(\ell)} = P(\ \cdot \mid \gbm{x}^{(\ell)} - \gbm{x}^{(\ell - 1)}, d_\textit{dec}, n_\textit{norm})
\end{equation}
It is possible to notice how this last approach could suffer from the same weaknesses that have been identified in iterative decoding as shown in~\cref{alg:method_intravisto_iter-dec}, due to the presence of an operation (difference) which assumes that its operands exist in a shared embedding space with linear properties.
Although this is true, the minimal distance between the hidden states used to compute the difference makes the presented issue have minimal impact on the actual decoding result.

\subsection{Injection}\label{ssec:method_intravisto_injection}

Lastly, we also include the possibility to perform \emph{injections} in InTraVisTo.
\emph{Injections} are an instance of \emph{activation patching}~\cite{olsson2022,meng2022,hanna2023,conmy2023,wang2023,mohebbi2023,zhang2024} utilized in a context that is not strictly tied to a formal \emph{causal intervention} framework~\cite{geiger2021,mcgrath2023}.
In fact, the main purpose of injections is to give the user the possibility to explore the model predictions in an interactive way, making it possible to change outputs of components and parts of the residual stream in order to unveil interesting patterns and properties.

Being a type of intervention, it is possible to express an injection utilizing the \emph{do-operator} as defined by~\citet{pearl2009}.
Assuming that we would like to inject an embedding $\gbm{\hat x}$ into the hidden state obtained as a result of the transformation $f_c^{(\ell)}$ of a component $c^{(\ell)}$ during the $i$-th token of the model's forward pass expressed as $f$, we can notate our injection utilizing the do-operator as:
\begin{equation}
    \label{eq:method_intravisto_inj_complete}
    f\Bigl(\gbm{x}_i \, \Big| \, \text{do}\bigl(f_c^{(\ell)}(\gbm{x}_i) = \gbm{\hat x}\bigr)\Bigr)
    \;\; \text{where} \;\; \gbm{\hat x} = \textit{encode}(\textit{id}_{\textit{inject}} \,|\, n_\textit{norm}, d_{dec})
\end{equation}
As previously noted for similar approaches, it is important to specify that there might be some problems with the nature of $\gbm{\hat x}$, as it is a representation that is generated from an encoded user-defined textual input ${id}_{\textit{inject}}$, which undergoes an embedding and normalization process utilizing $\gbm{W}_{d_{dec}}$ and $N_{n_\textit{norm}}$ following~\cref{eq:method_intravisto_linear-interp,eq:method_intravisto_quadratic-interp,eq:method_intravisto_normalization}.
This results in the injection of a `clean' embedding that is unexpected by the model, \todo{and thus} may destabilize the generation process.
In most causal intervention scenarios, this problem is not present when performing activation patching using counterfactual examples, due to the fact that patches are generated utilizing resampling or averaging approaches~\cite{hanna2023,conmy2023,wang2023}.
We choose to avoid these techniques since our main focus is directed towards giving the user complete control over the injected information, which is achieved by directly inserting the wanted embedding vector in the hidden state without letting the model elaborate it to create a plausible representation.
We \todo{refer to} this approach as \emph{complete replacement injection}.

Given the shortcomings of completely replacing an internal representation belonging to the model, we also devise a novel alternative to perform injections in a less destructive \todo{way}.
This idea stems from an interpretation of internal states as linear combinations of token embeddings, akin to the approach taken for the iterative decoding algorithm illustrated in~\cref{alg:method_intravisto_iter-dec}.
In practice, we directly modify the hidden state by computing a term consisting of the difference between the embedding representations of our injection $\gbm{\hat x}$ and the most likely token $\gbm{e}_{\gbm{h}}^{\textit{max}}$ obtained by decoding the internal state following~\cref{eq:method_intravisto_decoding}.
The additional term is weighted by a singleton scaling factor, computed as the dot product between the original hidden state and the embedding of the most likely decoded token $\gbm{e}_{\gbm{h}}^{\textit{max}}$.
We \todo{refer to} this approach as \emph{main component replacement injection}, and we \todo{it as}:
\begin{equation}
    \label{eq:method_intravisto_inj_main}
    \begin{aligned}
    \gbm{h'} = \gbm{h} + (\gbm{h} \cdot \gbm{e}_{\gbm{h}}^{\textit{max}})(\gbm{\hat x} - \gbm{e}_{\gbm{h}}^{\textit{max}})
    \;\; \text{where} \;\; &\gbm{e}_{\gbm{h}}^{\textit{max}} = {(\gbm{W}_{d_{dec}})}_{v_{\gbm{h}}^{\textit{max}},\cdot} \; , \\
    &v_{\gbm{h}}^{\textit{max}} = \operatornamewithlimits{argmax}_{v \in \mathcal{V}} P(v \mid \gbm{h}, d_\textit{dec}, N_{n_\textit{norm}})
    \end{aligned}
\end{equation}
In this case, referencing~\cref{eq:method_intravisto_inj_complete}, $\gbm{h}$ corresponds to the target hidden state $f_c^{(\ell)}(\gbm{x}_i)$, which gets \todo{modified} into $\gbm{h'}$ by $\gbm{\hat x}$.

Another noteworthy step in the injection process is the actual encoding of the injected embedding, since the formula used to compute $\gbm{\hat x}$ that has been broadly mentioned in~\cref{eq:method_intravisto_inj_complete,eq:method_intravisto_inj_main} is used considering an injection input that is composed by a single token.
However, InTraVisTo's interface offers the possibility of injecting an embedding representation that contains more than one token, thus we would like to generalize the aforementioned formula to actually handle multiple tokens at once.
This behavior is modeled by averaging the embedding representations of all tokens that compose the sentence, always under the assumption of a linear embedding space~\cite{mikolov2013,park2023}.
Considering a total of $id_1,\ldots,id_t$ tokens to be encoded, this averaging operation can be represented in the following way:
\begin{equation}
    \label{eq:method_intravisto_emb-avg}
    \gbm{\hat x} = N_{n_\textit{norm}}\Bigl(\frac{1}{t} \sum_{i=1}^{t}{{(\gbm{W}_{d_{dec}})}_{\textit{id}_{i},\cdot}}\Bigr)
\end{equation}
The soundness of this approach, including possible variants based on the same or different assumptions is \todo{tackled} in the following section (\cref{sec:method_embeddings}).


\section{Embedding Analysis}\label{sec:method_embeddings}

This second section is geared towards understanding the extent to which text embeddings still retain some degree of semantic factuality in LLMs (Large Language Models).
Early transformer models have been noted for their text embedding representations possessing interesting spatial properties deeply correlated with the semantic properties of the embedded words~\cite{allen2019,kalinowski2020}.
However, with the emergence of newer models that are larger and more intricate than ever before, our objective is to assess whether the semantic properties inherently tied with the geometry of embeddings still hold true.

\subsection{Word Encoding}

The first step for performing experiments on embedding representations, is defining a set of operations that make the encoding and comparison of embedding vectors possible.
In the following section we will illustrate the theoretical basis of the preliminary steps that were used to set up the actual experiments.

\subsubsection{Distance Metrics}

In order to compare and find `close' embeddings, we need to formalize a similarity measure between two embeddings. 
To this end, we define two main distance metrics that will be used to compare embedding vectors in their space:
\begin{equation}
    d_{\textit{dist}}(\gbm{x}, \gbm{y}) = 
    \label{eq:method_embeddings_distance}
    \left\{
    \begin{array}{cl}
        \sqrt{\sum_{i=1}^{D}{{(x_i - y_i)}^2}} &\ \text{if}\ \textit{dist} = \text{`euclidean'} \\
        1 - \frac{\sum_{i=1}^{D}{x_i y_i}}{\sqrt{\sum_{i=1}^{D}{x_i^2}} \cdot \sqrt{\sum_{i=1}^{D}{y_i^2}}} &\ \text{if}\ \textit{dist} = \text{`cosine'}
    \end{array}
    \right.
\end{equation}
The \emph{cosine similarity/distance} metric is the most immediate and widely used approach to compare embeddings since it utilizes the angle between two vectors to determine their similarity, its inherent normalization means that the magnitude of the vectors is not taken into consideration and only their directionality is compared.
On the other hand, the \emph{euclidean distance} is a less common distance metric that we decided to include to lay out an alternate interpretation option, which can provide a different geometric perspective over the embedding space.
Unfortunately, in order to make the euclidean distance an effective metric we need to normalize it separately.

\subsubsection{Normalization}

On the topic of normalization, we also include the possibility of \emph{`pre-normalizing'} the embedding space before performing the experiments.
This practice should help improve the comparability of embedding vectors, enhancing the results of linear operations in the embedding space.
Although it is also possible for it to have detrimental effects in some specific scenarios, mainly due to the intrinsic nonlinearity of the normalization operation.
In practice, the `pre-normalization' step includes the normalization of all embedding vectors using the euclidean norm as follows:
\begin{equation}
    \label{eq:method_embeddings_normalization}
    \gbm{W}_{\textit{emb}} = (\bar w_{ij})_{V \times D}
    \ \ \text{where}\ \bar w_{ij} = \frac{w_{ij}}{{\| w_{ij} \|}_2}
\end{equation}

\subsubsection{Multi-token Words}

Handling multi-token words is a vital aspect of our analysis, as we may encounter words that are split into multiple sub-word tokens by the model's tokenization process.
As we will see later in~\cref{ssec:method_embeddings_analogies}, our main experiments for this section will concern the evaluation of analogies and similarities between word groups.
As anticipated in~\cref{ssec:method_intravisto_injection}, the main choice of computing multi-token falls towards the usage of linear operations from the supposed preservation of linear features in the embedding space~\cite{mikolov2013,park2023}.
Therefore, we devised three main strategies to address multi-token words when they are fed into the embedding as inputs of analogies, along with two additional strategies to handle them as valid outputs of analogies.

Strategies applicable to both cases involve considering only the first token of a multi-token word (this causes results to be slightly worse on average, but the impact seems to be negligible in most cases).
For multi-token words as inputs, other approaches include calculating the average as already done in~\cref{eq:method_intravisto_emb-avg} or the sum over all the embedded tokens that form the multi-token word.
Conversely, for output multi-token words, we subdivide them into their token components, to then consider all tokens as targets for assessing the top-$k$ accuracy of an analogy result in the embedding space.
Formalizing the previously identified alternatives we obtain two $\textit{encode}$ functions controlled by $\textit{enc\_strat}$ for a series of $t$ tokens obtained from a word $w$:
\begin{equation}
    \label{eq:method_embeddings_multitok-in}
    {\textit{encode}}_{\textit{enc\_strat}}^{in}(w) = 
    \left\{
    \begin{array}{cl}
        \gbm{e}_1^w &\ \text{if}\ \textit{enc\_strat} = \text{`first\_only'} \\
        \frac{1}{t}\sum_{i=1}^{t}{\gbm{e}_i^w} &\ \text{if}\ \textit{enc\_strat} = \text{`average'} \\
        \sum_{i=1}^{t}{\gbm{e}_i^w} &\ \text{if}\ \textit{enc\_strat} = \text{`sum'}
    \end{array}
    \right.
\end{equation}
\begin{equation}
    \label{eq:method_embeddings_multitok-out}
    {\textit{encode}}_{\textit{enc\_strat}}^{out}(w) = 
    \left\{
    \begin{array}{cl}
        \{ tok_1^w \} &\ \text{if}\ \textit{enc\_strat} = \text{`first\_only'} \\
        \{ tok_1^w, \ldots, tok_t^w \} &\ \text{if}\ \textit{enc\_strat} = \text{`subdivide'}
    \end{array}
    \right.
\end{equation}
\vspace{0.25em}
\begin{equation*}
    \text{where}\  tok_1^w,\ldots,tok_t^w = \textit{tokenize}(w)
    ,\ \ \gbm{e}_1^w, \ldots, \gbm{e}_{t}^w = \Bigl\{ {{(\gbm{W}_{emb})}_{\textit{tok}_{i},\cdot}} \ |\ \forall tok_i \in \textit{tokenize}(w) \Bigr\}
\end{equation*}
One important distinction between the $\textit{encode}$ functions for input and output is the type of \todo{data} returned.
When performing input encoding we are interested in obtaining a single aggregated embedded representation of the target word, whereas, when encoding a word to be checked against the outputs of our experiment we need a set of token identifiers in the vocabulary space.
The first preliminary transformations performed on $w$ is tokenization, modeled as the $\textit{tokenize}$ function, and utilizing the model's tokenizer to obtain a set of token identifiers $tok_1^w,\ldots,tok_t^w$ from $w$.
After tokenization, tokens get transformed into embedding vectors $\gbm{e}_1^w, \ldots, \gbm{e}_{t}^w$ by selecting the corresponding row from the embedding matrix $\gbm{W}_{emb}$.
The role of the embedding matrix $\gbm{W}_{emb}$ can be fulfilled by either the input and output embedding matrices, as already seen previously.

Another approach that will be used to completely set aside issues tied with multi-token words consists in reducing the dataset to exclusively consider analogies composed of single-token words, at the cost of less overall valid samples.
This topic is discussed more in depth in the experimental setup section dedicated to the related set of experiments (\cref{sssec:exp_emb_exp1_expset}).

\subsection{Word Analogies}\label{ssec:method_embeddings_analogies}

We wish to establish if geometric relationships can still be modeled inside the input or output embedding space of recent LLM architectures.
The first step \todo{towards the direction of this approach} was the direct experimentation on \emph{word analogy tasks}, as already illustrated in~\citet{mikolov2013}.
Solving word analogies showcases the ability of an embedding space to model the semantics of words in a relatively consistent way, implying the existence of embedding dimensions with associated meanings (even if overlapping or under superposition as mentioned by~\citet{elhage2022,henighan2023}), which is not necessarily an expected feature of language models on a large scale.

\todo{In their purest form}, word analogies consist of $4$ words that satisfy a mutual analogy relationship of the type:
\begin{center}
    \emph{``The dog barks as the cat meows''} $\Longleftrightarrow$ $\textit{`dog'}\ :\ \textit{`bark'}\ =\ \textit{`cat'}\ :\ \textit{`meow'}$
\end{center}
As previously mentioned, in the context of transformer models, these types of analogies are commonly used to test whether semantic relationships between words are encoded as linear transformations in the embedding space of models.
There exist \todo{many} types of analogies, encoding different paradigms across various semantic fields and morphological connections.
For example~\citet{drozd2016} identify $4$ main relationship types between words in their \emph{BATS (Bigger Analogy Test Set)}: 
\begin{itemize}
    \item \textbf{Inflectional morphology}: $\textit{`user'}\!:\!\textit{`users'}$, $\textit{`hot'}\!:\!\textit{`hotter'}$, $\textit{`enjoy'}\!:\!\textit{`enjoying'}$, \dots
    \item \textbf{Derivational morphology}: $\textit{`able'}\!:\!\textit{`unable'}$, $\textit{`apply'}\!:\!\textit{`reapply'}$, $\textit{`nice'}\!:\!\textit{`nicely'}$, \dots
    \item \textbf{Encyclopedic semantics}: $\textit{`madrid'}\!:\!\textit{`spain'}$, $\textit{`dante'}\!:\!\textit{`poet'}$, $\textit{`cabbage'}\!:\!\textit{`green'}$, \dots
    \item \textbf{Lexicographic semantics}: $\textit{`armchair'}\!:\!\textit{`chair'}$, $\textit{`nap'}\!:\!\textit{`sleep'}$, $\textit{`after'}\!:\!\textit{`before'}$, \dots
\end{itemize}
By combining two pairs of words from the same category we obtain word analogies in their full form, and they can be resolved by performing appropriate geometric transformation on the vectors that represent words.

\subsubsection{Analogy Computation}

As far as the actual word analogy computation is concerned, given 4 words $\{w_1, w_2, w_3, w_4\}$ set up as an analogy of the type `$w_1 : w_2 = w_3 : w_4$', we can formalize it as follows:
\begin{equation}
    \label{eq:method_embeddings_analogy}
    \begin{gathered}
        closest = \operatornamewithlimits{argmin}_{\forall v \in \mathcal{V}} d_{\textit{dist}}\bigl(\tilde w, \textit{encode}^{\textit{emb}}(v)\bigr) \\
        \tilde w = \textit{analogy}(w_1, w_2, w_3, w_4)
    \end{gathered}
\end{equation}
Where $d_{\textit{dist}}$ is the chosen distance metric as illustrated in~\cref{eq:method_embeddings_distance}, and $\textit{encode}^{\textit{emb}}$ refers to a simple encoding of a single token $v \in \mathcal{V}$ using the embedding matrix $\gbm{W}_{\textit{emb}}$.
On the other hand, \emph{`closest'} indicates the returned value, which is the actual token identifier of the closest embedding (of a word present in vocabulary $\mathcal{V}$) to the computed result of the word analogy.
Furthermore, the $\textit{analogy}$ function expresses the combination of words $\{w_1, w_2, w_3, w_4\}$ that composes the logic relationship of the word analogy.
In case we need a more robust evaluation support for comparing models, there also exists the possibility of gathering the top-$k$ closest embeddings to the obtained result instead of only representing the first one.
Additionally, due to the fact that it is possible to define multiple analogy test cases from a single combination of words, the definitions of the actual \emph{analogy function} implementations slightly falls out of the scope of this section, while keeping in mind that they will be discussed more in depth inside the experimental setup section of the corresponding experiment (\cref{sssec:exp_emb_exp1_expset}).
Thus, for simplicity, we can assume to \todo{follow} the \todo{renowed} `$king - man + woman \approx queen$' example from~\citet{mikolov2013}, obtaining:
\begin{equation}
    \label{eq:method_embeddings_analogy-function}
    \begin{aligned}
        \tilde w &= \textit{analogy}(w_1, w_2, w_3, w_4) \\
        &= {\textit{encode}}_{\textit{enc\_strat}}^{in}(w_1) - {\textit{encode}}_{\textit{enc\_strat}}^{in}(w_2) + {\textit{encode}}_{\textit{enc\_strat}}^{in}(w_4) \\
        &\color{gray} \approx {\textit{encode}}_{\textit{enc\_strat}}^{out}(w_3) 
    \end{aligned}
\end{equation}
Where the ${\textit{encode}}_{\textit{enc\_strat}}$ function follows the definitions laid out in~\cref{eq:method_embeddings_multitok-in,eq:method_embeddings_multitok-out}.

Another perspective over the classic word analogies resolution strategy consists in the \emph{offset analogies} task.
It holds a similar premise and the set of input analogies is shared between tasks, however, we perform a further subdivision of the dataset, assigning each word analogy into a bigger group based on the actual relationship that is modeled by the analogy (such as `gender', `capital of', `royalty', \todo{etc.}).
Subsequently, we use multiple analogies from the same group to obtain the relationship term as a vector (offset term), and we observe if it is possible to shift the representation from an element of the analogy to the other utilizing the computed offset.
This techniques shares many similarities with the \emph{3CosAvg} approach suggested by \citet{drozd2016}, however, we offer a more specific computation for the offset term identified as $\gbm{\Delta}$.
\begin{equation}
    \label{eq:method_embeddings_delta-analogy-function}
    \begin{aligned}
        \tilde{w}^i &= \textit{offset\_analogy}(\gbm{\Delta}_{\textit{batch}}^{(w_3, w_4)}, w_3^i, w_4^i) \\
        &= {\textit{encode}}_{\textit{enc\_strat}}^{in}(w_3^i) + \gbm{\Delta}_{\textit{batch}}^{(w_3, w_4)} \\
        &\color{gray} \approx {\textit{encode}}_{\textit{enc\_strat}}^{out}(w_4^i) \\
        \gbm{\Delta}_{\textit{batch}}^{(w_a, w_b)} &= \frac{1}{B_{\textit{batch}}}\sum_{j=1}^{B_{\textit{batch}}}{\Bigl({\textit{encode}}_{\textit{enc\_strat}}^{in}(w_b^j) - {\textit{encode}}_{\textit{enc\_strat}}^{in}(w_a^j)\Bigr)} \\
        \text{where}&\ \textit{batch} = \Bigl\{ \{w_1^i, w_2^i, w_3^i, w_4^i\} \ |\ \forall i \in \{1,\ldots, B_{\textit{batch}}\} \Bigr\}
    \end{aligned}
\end{equation}
As it is possible to observe in~\cref{eq:method_embeddings_delta-analogy-function}, we are computing the analogy term considering only \todo{half} of the word available in the analogy; in the cited case $w_3$ and $w_4$.
This is \todo{done} since, as explained in the first paragraphs of~\cref{ssec:method_embeddings_analogies}, word analogies are effectively composed by two pairs of words, each representing a semantic or morphological relationship.
Thus, considering that in the offset analogies scenario we are looking to extract a vector that represents the common relationship modeled by a subgroup of analogies, we only need a single occurrence for each \todo{single} pair of words.

\subsubsection{Analogy Evaluation}\label{subsubsec:method_embeddings_evaluation}

One of the key aspects of the proposed analogy approaches is the actual evaluation process of a set of model's embeddings on a given analogy task.
To assess the models' performance in this particular scenarios we considered two main metrics: the top-$k$ accuracy across all analogies, and the \emph{rankscore}.

The top-$k$ accuracy is a well established scoring metric that considers the number of times that a correct analogy solution appears inside the first $k$ results, averaging over all $n$ analogy samples present in the dataset.
In our case, it can be written using the following notation:
\begin{equation}
    \label{eq:method_embeddings_topk-accuracy}
    \textit{acc}_k = \frac{1}{n} \sum_{i=1}^{n} \mathbb{I}\Bigl( {\left\{ {\textit{result}\,}^i_j \ |\ \forall j = 1,\ldots,k \right\}} \cap {\textit{sol}\,}^i \neq \emptyset \Bigr)
\end{equation}
where ${\textit{result}\,}^i_j$ represents the $j$-th result of the $i$-th analogy, ${\textit{sol}\,}^i$ indicates the set of solution for the $i$-th analogy obtained through the use of the ${\textit{encode}}^{out}$ function as illustrated in~\cref{eq:method_embeddings_multitok-out}, and $n$ is the dimension of the considered dataset.

%\todo[purple!30]{Mean Reciprocal Rank / Rankscore evaluation of analogies}
%Alternatively, we devised an additional measure which fulfills a similar purpose to the top-$k$ accuracy by evaluating a set of embeddings on an analogy task.
%The chosen metric is a slight variation of the Mean Reciprocal Rank statistic measure, commonly referred to as $MRR = \frac{1}{n}\sum_{i=1}{n}{\frac{1}{rank_i}}$ \todo[orange]{cite?}.
%We considered this approach in order to fully utilize the rank information given by the \todo{comparison analysis}, rather than just limiting the computation of our metric to check for the presence of matches between the results and the solutions, as it is done in the top-$k$ accuracy case.
%Rankscore:
%\begin{equation}
%    \label{eq:method_embeddings_rankscore}
%    \textit{rankscore}_k = 1 - \frac{\sum_{i=1}^{n}{rank_i}}{n \cdot k}
%\end{equation}
%\todo[cyan]{Understand if rankscore makes sense, eventually adapt it to MRR or remove MRR as a reference}


\section{First Order Prediction}\label{sec:method_fom}

This third section explores the theoretical background of experiments analyzing the relationship between output embeddings (represented by output decoder weights) and input embeddings in the model's initial embedding layer.
In particular, we are interested in observing if there exists any particular function that is being performed by the output decoder against the input embeddings, such as inversion or even an attempt at predicting the most likely next token.

\subsection{Markov Models}

The hypothesis regarding output embeddings carrying out a prediction of the most likely subsequent token based on the input is inspired by architectural deconstructions of decoder-only transformer models, revealing residual links directly connecting the embedding layer to the decoder~\cite{vaswani2017}.
On this topic, \citet{elhage2021} claim that zero layer transformers model bigram statistics and that the bigram table can be accessed directly from the weights, although their experimental results only confirm their hypotheses for a restricted set of models.
Since the task of decoder-only models is to predict the next word given the previous context~\cite{radford2019}, by stripping from the architecture all the transformer blocks while keeping only a direct link from the input embedding layer to the output decoder layer, yields a \emph{First Order Model (FOM)} at best capable of predicting the next token given the previous one, akin to the behavior of first-order Markov models~\cite{markov2006}.
Such Markov models leverage bigram probability estimates retrieved from a corpus of text in order to generate the next word, provided with the previous one.
Given words $w_i$ and $w_j$, the bigram probability can be expressed as:
\begin{equation}
    \label{eq:method_fom_markov-prob}
    P(w_j|w_i) = P_{ij} = \frac{P(w_i, w_j)}{P(w_i)}
\end{equation}

The state-based stochastic transitions of a Markov model can be entirely described \todo{by} a transition matrix $\gbm{Q} \in \mathbb {R}_{+}^{|S| \times |S|}$ given the state space $S$.
The transition matrix is defined as:
\begin{equation*}
    \gbm{Q} = \left(P_{ij}\right)_{i,j} \;\; \forall i,j \in 1,\dots,|S|
\end{equation*}
Where $P_{ij}$ follows the definition specified in~\cref{eq:method_fom_markov-prob} and, representing the transition probability of moving to state $j$ from state $i$, must respect $\sum_{j=1}^{|S|}{P_{ij}} = 1 \;\; \forall j \in 1,\dots,|S|$.

\subsection{Matrix Comparison}\label{ssec:method_fom_matrix}

As a preliminary way to explore various relationships between the input and output embeddings, we decide to perform an analysis of their matrices.
To do so, we compute the `transition matrix' for our First Order Model (FOM) by combining and transposing the input and output weight matrices as follows:
\begin{equation}
    \label{eq:method_fom_fom-matrix}
    \gbm{Q}_{\textit{FOM}} = \operatorname{softmax}(\gbm{Q}_{\textit{FOM}}^{log}) = \operatorname{softmax}(\gbm{W}_{in} \cdot \gbm{W}_{out}^\T)
\end{equation}
Where $\gbm{W}_{in}$ and $\gbm{W}_{out}$ are, respectively, the input and output embedding weight matrices.
One important detail is the presence of the \emph{softmax} operator, which is applied to each row of the resulting logit transition matrix $\gbm{Q}_{\textit{FOM}}^{log}$ in order to obtain a transition matrix representing actual probabilities over the vocabulary space.
This operation is necessary \todo{since} the direct product between embedding matrices returns series of logit distributions over the vocabulary, which do not benefit from the \todo{properties of} probability distributions.
However, not all experiments require the computation of the actual transition matrix $\gbm{Q}_{\textit{FOM}}$ as experiments that consider top-$k$ aggregations and other greedy approximations of next-token predictions can also be performed with $\gbm{Q}_{\textit{FOM}}^{log}$ since the ordering of elements is not \todo{modified} by the softmax operator.
To avoid confusion, the \todo{aforementioned} distinction is \todo{made} only for practical experiments present in~\cref{sec:exp_fom}, whereas the \todo{theoretical groundwork} \todo{present} in this section \todo{only acknowledges} the probability matrix $\gbm{Q}_{\textit{FOM}}$ as the FOM transition matrix.

\todo{For reasons examined in depth} in~\cref{sec:exp_fom,sssec:exp_fom_exp1_expset}, we also define a FOM transition matrix with the addition of RMSNorm (\todo{from here} simply RMS normalization).
RMS normalization is a normalization technique defined in~\cref{eq:background_rmsnorm}, \todo{which} is commonly found in multiple places inside recent transformer architectures as highlighted in~\cref{ssec:background_transf_current}.
In our case, we split the RMS computation between input and output embedding matrices by performing the actual RMS normalization portion of the computation in the input and the weight product on the output.
\begin{equation}
    \label{eq:method_fom_fom-matrix-rms}
    \begin{gathered}
        \gbm{Q}_{\textit{FOM}}^{RMS} = \operatorname{softmax}(\gbm{W}_{in,RMS} \cdot \gbm{W}_{out,RMS}^\T) \\
        \text{where}\ \left\{
            \begin{array}{cl}
                \gbm{W}_{in,RMS} &= \left(\frac{\gbm{w}_i^{in}}{RMS(\gbm{w}_i^{in})}\right)_{i,\cdot} = \left(\frac{\gbm{w}_i^{in}}{\sqrt{\frac{1}{D}\sum_{j=1}^{D}{(w_{ij}^{in})^2}}}\right)_{i,\cdot} \\
                \gbm{W}_{out,RMS} &= \left(\gbm{\gamma}^\T \odot \gbm{w}_j^{out}\right)_{\cdot,j}
            \end{array}
        \right.
    \end{gathered}
\end{equation}
In particular, $\gbm{w}_i^{in}$ refers to row vectors extracted from $\gbm{W}_{in}$, $w_{ij}^{in}$ refers to elements extracted from $\gbm{W}_{in}$ and $\gbm{w}_j^{out}$ refers to column vectors extracted from $\gbm{W}_{out}$, therefore:
\begin{equation*}
    \gbm{w}_i^{in} = \left(\gbm{W}_{in}\right)_{i,\cdot},\ \ w_{ij}^{in} = \left(\gbm{W}_{in}\right)_{i,j},\ \ \gbm{w}_j^{out} = \left(\gbm{W}_{out}\right)_{\cdot,j},
\end{equation*}
Additionally, the multiplicative weights $\gbm{\gamma}$ associated with RMS normalization in~\cref{eq:method_fom_fom-matrix-rms} \todo{are the} weights of the RMSNorm layer positioned at the end of the stack of transformer layers, before the unembedding layer in recent decoder architecture, as seen in~\cref{fig:background_llama-arch}.

At this point, we are able to use the newly defined matrix to compare it against \todo{others} in order to quantify its resemblance against other known transition matrices.
Our metric of choice for direct matrix comparison is the Frobenius norm, which is a relatively simple way to estimate how two matrices differ from each other.
Due to its inherent simplicity, the Frobenius norm can be considered a naïve option for evaluating matrix distance in a \todo{scenario composed of} transition matrices.
Indeed, the results obtained in~\cref{sssec:exp_fom_exp1_results} \todo{expose} the Frobenius norm as an unreliable comparison metric with respect to other measures, \todo{but} still capable of providing some amount of valuable insight to the experiments formalized in~\cref{ssec:exp_fom_exp1,ssec:exp_fom_exp2}. 

To explore the potential inverse relationship, we compute the distance between the FOM transition matrix $\gbm{Q}_{\textit{FOM}}$ and the identity matrix of equivalent dimensions $\gbm{I}_V$.
This follows the idea that a Markov model where every state leads into itself is represented by an identity matrix, thus if our input/output embedding pair is attempting to converge towards being equal (as it is enforced by weight tying) we should observe the FOM transition matrix degenerating into an identity one.
We can model the distance between the FOM matrix and the identity matrix in the following way:

\begin{equation}
    \label{eq:method_fom_fom-i-comp}
    d_{\textit{FOM},I} = \| \gbm{Q}_{\textit{FOM}} - \gbm{I}_V \|_F
\end{equation}

Following the same intuition, it is also possible to compare the FOM transition matrix $\gbm{Q}_{\textit{FOM}}$ with the transition matrix of a bigram Markov model $\gbm{Q}_{\textit{markov}}$ trained on a \todo{statistically relevant dataset}:

\begin{equation}
    \label{eq:method_fom_fom-markov-comp}
    d_{\textit{FOM},\textit{markov}} = \| \gbm{Q}_{\textit{FOM}} - \gbm{Q}_{\textit{markov}} \|_F
\end{equation}

\subsection{Prediction Comparison}\label{ssec:method_fom_pred}

Additionally, we wanted to approach the previous ideas from a more practical point of view.
To this end we recorded the top-$k$ answers elicited using the FOM matrix $\gbm{Q}_{\textit{FOM}}$ across all tokens inside the model's vocabulary, using each token as the known word and obtaining the FOM's most probable next tokens.
Once we have the set of generated next tokens for each token inside the vocabulary we can compare them to our actual hypothesized reference tokens, obtained applying the identity function in the `inversion' hypothesis case, and in the `Markov' hypothesis case by querying the Markov model in the same way that we did with the FOM\@.
As already done before in~\cref{subsubsec:method_embeddings_evaluation} using~\cref{eq:method_embeddings_topk-accuracy}, we employ the top-$k$ accuracy metric to establish the similarity between the predictions of the FOM against the two baselines defined by the identity function and the Markov model.
The top-$k$ accuracy is computed by taking the $k$ tokens corresponding to the $k$ highest values between logits, and observing if the input token is present among them.

Generalizing, we can formalize the average top-$k$ accuracy between two models represented by their transition matrices $\gbm{Q}_1$ and $\gbm{Q}_2$ according to the respective $k_1$ and $k_2$ variables in the following way:
\begin{equation}
    \label{eq:method_fom_topk}
    \begin{aligned}
    \textit{accuracy}_{k_1,k_2}(\gbm{Q}_1, \gbm{Q}_2) = \frac{1}{V} \sum_{v \in \mathcal{V}}{\textit{tok\_overlap}_{k_1,k_2}(\gbm{Q}_1, \gbm{Q}_2, v)} \\
    \textit{tok\_overlap}_{k_1,k_2}(\gbm{Q}_1, \gbm{Q}_2, v) = \left\{
        \begin{array}{cl}
            1 &\ \text{if}\ \operatorname{top-k}(\gbm{Q}_1, k_1, v) \cap \operatorname{top-k}(\gbm{Q}_2, k_2, v) \neq \emptyset \\
            0 &\ \text{otherwise}
        \end{array}
        \right.
    \end{aligned}
\end{equation}
Where, as stated before, $\operatorname{top-k}(\gbm{Q}, k, v)$ represents the set of $k$ vocabulary tokens corresponding to the $k$ indices with highest probability obtained from indexing $Q$ with $v$.
In addition, when comparing $\gbm{Q}_1 = \gbm{Q}_{FOM}$ with the identity matrix we have that $\gbm{Q}_2 = \gbm{I}_V$ and $k_2 = 1$.

On the other hand, we also expand on the prediction-based comparisons by introducing two specific set metrics to provide a weighted perspective on the distributions of transition matrices.
We choose to employ the overlap coefficient (which is equal to the Dice-S\o{}rensen index when considering sets with equal cardinality) and the Jaccard index to evaluate the predictions obtained from pairs of models averaged over the target vocabulary.
We refer \todo{to them} as overlap metrics and we formalize them in the following way:
\begin{equation}
    \label{eq:method_fom_overlap}
    \begin{aligned}
        \operatorname{overlap}_k(\gbm{Q}_1, \gbm{Q}_2) &= \frac{1}{V}\sum_{v \in \mathcal{V}}{\frac{|\operatorname{top-k}(\gbm{Q}_1, k, v) \cap \operatorname{top-k}(\gbm{Q}_2, k, v)|}{\operatornamewithlimits{min}\bigl\{|\operatorname{top-k}(\gbm{Q}_1, k, v)|, |\operatorname{top-k}(\gbm{Q}_2, k, v)|\bigr\}}} \\
        \operatorname{J}_k(\gbm{Q}_1, \gbm{Q}_2) &= \frac{1}{V}\sum_{v \in \mathcal{V}}{\frac{|\operatorname{top-k}(\gbm{Q}_1, k, v) \cap \operatorname{top-k}(\gbm{Q}_2, k, v)|}{|\operatorname{top-k}(\gbm{Q}_1, k, v) \cup \operatorname{top-k}(\gbm{Q}_2, k, v)|}} \\
    \end{aligned}
\end{equation}

\subsection{Probabilistic Comparison}\label{ssec:method_fom_prob}

In order to obtain a wider perspective on the probability distributions included inside \todo{the} transition matrices, we evaluate our models through the lenses of metrics that take into account the entirety of the information included in said probability distributions.
These metrics can be utilized, either directly or indirectly, to measure the similarity or dissimilarity between two probability distributions.

\todo{The first chosen metric} is the perplexity, which quantifies how well a certain discrete probability distribution is able to predict a sample; that is, the likelihood of the sample being drawn from the distribution itself.
In other words, it is commonly referred to as the `surprise' of the distribution in observing a certain sample: high values mean that the sample is unlikely to align with the distribution (greater surprise), whereas low values imply that the sample is likely to be drawn from the distribution (low surprise).
\todo{Given} the \todo{generic} formulation for perplexity for a probability distribution $p$ over a sequence of words $w_1,\dots,w_N$ defined as:
\begin{equation}
    \label{eq:method_fom_perplexity-base}
    PP(p) = \exp{\Bigl(-\frac{1}{N}\sum_{i=1}^{N}{\log{p(w_i)}}\Bigr)}
\end{equation}
In practice, we compute our version of perplexity considering the transition matrix $\gbm{Q}$ as:
\begin{equation}
    \label{eq:method_fom_perplexity}
    PP(\gbm{Q}, S) = e^{-\frac{1}{T}\sum_{t=1}^{T}{\ln{\left((\gbm{Q})_{S(t+1),S(t)}\right)}}}
\end{equation}
Where $S$ \todo{indicates} a sentence composed of a total of $T$ tokens identified as $S(1),\dots,S(T)$ and $(\gbm{Q})_{i,j}$ represents the indexing operation done on the transition matrix $\gbm{Q}$, thus obtaining the transition probability from state $i$ to state $j$.

Interestingly, we highlight the fact that concept of perplexity is strictly related to \todo{that of} cross-entropy.
Cross-entropy is generally computed between two probability distributions and measures the information quantity needed to represent a sample as if it was drawn from a model distribution rather than the true one.
\todo{In fact}, by simply looking at~\cref{eq:method_fom_perplexity-base} we can clearly observe that it directly \todo{includes} the formulation for an estimation of cross-entropy $H$ over a distribution $p$, \todo{that being}:
\begin{equation*}
    H(p) = - \frac{1}{N}\sum_{i=1}^{N}{\log{p(w_i)}}
\end{equation*}
One crucial fact to \todo{keep in mind} is that the logarithm's base doesn't affect the \todo{ground truth} of the result, only its unit of measure.
In most NLP scenarios, cross-entropy is measured in \emph{bits}, thus $2$ is used as a base for the logarithm.
However, when computing the perplexity measure, the base choice must be consistent between logarithm and exponentiation \todo{such as in the case} of~\cref{eq:method_fom_perplexity}, where we utilize $e$ for both \todo{operations}.

On the other hand, we also consider the Kullback-Leibler divergence (KL divergence) as a way to \todo{gauge at} the direct dissimilarity between the probability distributions encoded inside transition matrices.
By definition, the KL divergence cannot be classified as a metric due to its inherent \todo{asymmetricity} since it is possible to identify a reference distribution and a model distribution.
The KL divergence effectively measures the divergence of the model distribution $Q$ from the reference distribution $P$ over a common domain $X$, and it is \todo{generally} expressed as:
\begin{equation*}
    D_{KL}(P||Q) = \sum_{x \in X}{P(x)\ln{\frac{P(x)}{Q(x)}}}
\end{equation*}
In \todo{our case}, we compute the mean KL divergence over all tokens present in the vocabulary using the following formula:
\begin{equation}
    \label{eq:method_fom_kldiv}
    \begin{gathered}
    {\overline {D_{KL}}}(\gbm{Q}_{ref} || \gbm{Q}_{model}) = \frac{1}{V}\sum_{v \in \mathcal{V}}{D_{KL}^v(\gbm{Q}_{ref} || \gbm{Q}_{model})} \\
    \text{where}\ D_{KL}^v(\gbm{Q}_{ref} || \gbm{Q}_{model}) = \sum_{w \in \mathcal{V}}{\bigl((\gbm{Q}_{ref})_{v,w}\bigr)\ln{\frac{(\gbm{Q}_{ref})_{v,w}}{(\gbm{Q}_{model})_{v,\cdot}}}}
    \end{gathered}
\end{equation}
Where, as mentioned before for~\cref{eq:method_fom_perplexity}, $(\gbm{Q})_{i,j}$ represents the indexing operation on the transition matrix $\gbm{Q}$.


\chapter{Experiments}\label{ch:experiments}
This chapter presents the experiments conducted to evaluate the methodology proposed in~\cref{ch:methodology} for the reference scenarios described in~\cref{ch:research_questions}.
The experiments were initially designed to shed light onto some specific aspects of LLM internal state interpretability via vocabulary decoding, although after some interesting findings, more attention was given to the actual initial and final embedding representations in LLMs.

This chapter is organized similarly to~\cref{ch:research_questions,ch:methodology}, following the three primary research questions defined as a baseline.
Each research question contemplates multiple experiments aimed at providing empirical evidence to elicit a deeper understanding of large Transformer architectures.
This understanding is achieved through the perspective of the question at hand and, more generally, with the aid of InTraVisTo.

Where applicable, experiments are organized around a fixed set of key points to formalize the analysis in a structured way.
These points include:
\begin{itemize}
    \item \textbf{Experimental Setup}: a complete description of the performed experiment in retlation to the theoretical background provided in~\cref{ch:methodology}.
This may include parameter combinations, hardware specifications and resources employed.
    \item \textbf{Dataset}: a summary of the data and information used in the experiment, whether for training models or as direct testing material.
    \item \textbf{Models}: a list of models, along with brief details regarding their configurations, that have been utilized for the experiment or other auxiliary tasks.
    \item \textbf{Results}: a presentation of the results obtained from the experiment,  paired with comments, explanations and an analysis of potential implications.
\end{itemize}
If any of the previously identified points overlap for all experiments within a given research question, they are directly incorporated into the main section belonging to the question in order to avoid unnecessary repetition.
Furthermore, each research question concludes with a discussion section aimed at comparing the results of the various experiments and providing final remarks on the overall outcomes for the specific inquiry.

\section{Transformer Visualization}\label{sec:exp_intravisto}

As anticipated in~\cref{sec:rq_intravisto} and explored in~\cref{sec:method_intravisto}, the experiments tied to this research question are centered around the InTraVisTo tool.
The first points (\cref{ssec:exp_intravisto_exp1,ssec:exp_intravisto_exp2,ssec:exp_intravisto_exp3}) within this section provide an exhaustive overview of the interface, diving into the technical aspects and explaining the role of components that are present in the application.
Consequently, these points do not present the experimental structure defined in the introductory section of this chapter.
Whereas, the last point (\cref{ssec:exp_intravisto_exp4}) is dedicated to the exploration of the proposed tool's actual capabilities, taking into consideration specific use cases and small investigations to provide concrete examples of possible usage scenarios.

\subsection{Decoding Interface}\label{ssec:exp_intravisto_exp1}

Decoding the meaning of hidden state vectors at various depths of a Transformer stack is essential for providing an intuition as to how the model is working.
InTraVisTo allows decoding and inspection of the main four vectors (defined in~\cref{ssec:method_intravisto_decoding}) that compose each layer, while offering a human-interpretable representation of each hidden state by performing a decoding operation.
This decoding operation is carried out using a specialized decoder which, given a hidden state as input, finds related tokens from the model's vocabulary with the goal of returning an interpretable output.

In this section, we present the first visual output of InTraVisTo.
Our goal revolves around a layer-by-layer interpretation of the model, thus layers are stacked vertically starting from the bottom with the embedding layer up to the top with the normalized outputs of the last layer.
Due to the inference process of the Transformer architecture, each stack of layers is repeated for every token, resulting in a grid where the x-axis represents token positions in the sequence and the y-axis represents layer numbers.
A natural visual representation for this grid-like structure is a heatmap where each cell represents a token-layer combination.
Visually, inside each cell we can find the main decoded hidden state, and by hovering on it, a pop-up with its secondary representations along with additional information appears.
In addition, for the sake of being able to tell apart input tokens (prompted by the user) from output tokens (autonomously generated by the model), a vertical line is put in place to divide the former from the latter.
As it is possible to notice from~\cref{fig:exp_intravisto_1_A}, the heatmap features two additional layers: one at the beginning (bottom) and one at the end (top).
The first layer is used to represent states before entering the Transformer stack, right after the embedding layer.
Whereas the last layer offers a representation that is forcibly normalized and decoded using the output embedding, in order for it to align with the raw generation output provided by the model.
The generation of the heatmap requires the user to choose a \emph{target embedding}, a \emph{decoding strategy} and a \emph{probability} to display.
Those parameters can be tweaked using a dedicated panel present in the interface, shown in~\cref{fig:exp_intravisto_1_B}.

\begin{figure}[t!]
    \centering
    \includegraphics[width=\textwidth]{exp_intravisto_1A_heatmap.png}
    \caption[InTraVisTo's heatmap visualization given the prompt \emph{``What is the capital of Italy?''} to Mistral.]{InTraVisTo's heatmap visualization given the prompt \emph{``What is the capital of Italy?''} to Mistral, utilizing linear interpolation for decoding and probability of the argmax term for color.}
    \label{fig:exp_intravisto_1_A}
\end{figure}

\begin{figure}[t!]
    \centering
    \includegraphics[width=\textwidth]{exp_intravisto_1B_selectors.png}
    \caption[Selector panel containing InTraVisTo controls.]{Selector panel containing InTraVisTo controls to set (from left to right) \emph{decoding strategy}, \emph{target embedding}, \emph{probability} to display and \emph{attribution formula} of component outputs to the residual stream.}
    \label{fig:exp_intravisto_1_B}
\end{figure}

The \emph{embedding} refers to the position of the hidden state vector to be decoded within the layer.
Therefore, by selecting different embedding positions, the user can inspect distinct subcomponents within each layer and understand which information is propagated between layers.
In order to prevent saturating the visualization with information, only a single embedding position can be visualized at any given time using the heatmap.

Conversely, the \emph{decoding strategy} choice refers to the decoding matrix employed during the decoding process of each visualized hidden state, fulfilling the role of ``perspective'' for the interpretation.
Selecting an effective decoding strategy is a nontrivial task and can be considered the key element of the vocabulary decoding approach to interpretability.
As reviewed in~\cref{ssec:related_vocab}, numerous approaches for decoding hidden states already exist in literature.
InTraVisTo's decoding operation is formalized in~\cref{eq:method_intravisto_decoding} and utilizes a user-selected decoding matrix.
In addition to immediate choices such as input embeddings or output embeddings, which are focused on providing meaningful decoding for lower and upper layers of the network respectively, InTraVisTo also offers the possibility of decoding states using interpolated and max-probability decoders, defined in~\cref{eq:method_intravisto_linear-interp,eq:method_intravisto_quadratic-interp,eq:method_intravisto_max-p}.
These novel decoding techniques are designed in order to provide users with semantically meaningful representations for every layer of the model simultaneously.

Finally, the \emph{probability selector} directly affects the quantity used to weight the color grading in the heatmap.
The four main options consist of P(argmax term), entropy, attention contribution and feedforward contribution. % chktex 36
Additionally, a secondary control labeled as ``Residual Contribution'' affects the attention and feedforward contribution options for the probability selector.
This control determines the metric used to evaluate the concept of contribution between Transformer components and the residual stream according to~\cref{eq:method_intravisto_norm-contrib,eq:method_intravisto_kl-contrib}.
The mathematical background for these selectors is discussed in detail in~\cref{sssec:method_intravisto_decoding_metrics}.

Additionally, users are provided with several secondary controls to further explore the models' generation process, as illustrated in~\cref{fig:exp_intravisto_1_C}.
For example, the `Embedding normalization' control allows users to select the type of normalization to apply to hidden states before decoding, as defined in~\cref{eq:method_intravisto_normalization}.
Another selector is dedicated to handling the strategy for decoding secondary tokens (see~\cref{sssec:method_intravisto_decoding_tokens}), offering either a `top-$5$' approach or by using the iterative decoding algorithm, as illustrated in~\cref{alg:method_intravisto_iter-dec}.
Moreover, users are given the option to exclude the column corresponding to the \emph{<start>} token from the visualization, as decoding it often does not yield meaningful results.
Furthermore, users can influence the generation process by choosing a different model from a customizable selection pool or by adjusting the number of generated tokens using the appropriate selectors.

\begin{figure}[t!]
    \centering
    \includegraphics[width=\textwidth]{exp_intravisto_1C_controls.png}
    \caption{Selector panel containing additional generation and visualization controls for InTraVisTo.}
    \label{fig:exp_intravisto_1_C}
\end{figure}

\subsection{Flow Interface}\label{ssec:exp_intravisto_exp2}

The second visualization provided by InTraVisTo is a Sankey diagram that aims to depict the information flow through the Transformer network (\cref{fig:exp_intravisto_2_A}).

\begin{figure}[t!]
    \centering
    \includegraphics[width=0.9\textwidth]{exp_intravisto_2A_sankey.png}
    \caption[InTraVisTo's Sankey diagram visualization given the prompt \emph{``What is the capital of Italy?''} to Mistral.]{InTraVisTo's Sankey diagram visualization of the last 7 layers given the prompt \emph{``What is the capital of Italy?''} to Mistral, utilizing linear interpolation for decoding, showing only the top attention trace and calculating residual contributions using the norm.}
    \label{fig:exp_intravisto_2_A}
\end{figure}

Nodes in the diagram depict all the hidden states contained in each layer, visualizing each one of the four vectors referenced in~\cref{ssec:method_intravisto_decoding} at the same time.
Similarly to the heatmap visualization, nodes display the main decoding result as their label and, when hovered upon, provide additional information in the form of a pop-up tooltip containing secondary tokens.
One additional piece of information, exclusively found in the tooltip of output nodes, is the decoded difference from the previous layer.
This detail shows the primary and secondary tokens obtained from decoding the difference between the output state of the current layer with the output state of the previous layer as if it were a separate state.
The goal of this representation is trying to visualize in a human interpretable way, the information added by the current layer with respect to the previous one.

On the other hand, edges represent the amount of relevance carried by the residual stream, illustrating how various components accumulate or disperse this flow as they process information scattered throughout the model.
Being a Sankey diagram, no amount of flow is ever lost between layers, as the flow corresponding to all tokens in a horizontal section (representing a layer) always sums to $100\%$.
The flow originates from the topmost layer of nodes, and is recursively computed considering the contributions of each encountered node.
More specifically, flow computation is discussed in detail in~\cref{ssec:method_intravisto_flow} and can be formalized using~\cref{eq:method_intravisto_flows}.

When considering a new generation run, the totality of the flow is evenly split between the output nodes at the end of the Transformer stack.
However, if the user decides to inspect a specific cell in the heatmap by clicking on it, the Sankey diagram also adapts by recalculating the flow considering the corresponding node as the sole topmost node, consequently accounting for $100\%$ of the flow.
We subdivide nodes into three categories, color-coded for visualization convenience: intermediate and output nodes in blue, attention nodes in green and feedforward nodes in pink.
Moreover, flows exiting from each node inherit the color of their originating node, leading to a clear display of the main information paths and their direction.
Additionally, flows are given a further shading factor that is proportional to the KL divergence between the decoded hidden state distributions of the nodes that they connect.
This allows users to appreciate in an immediate way states that exhibit rapid changes in distribution, thus providing a better localization for possible zones of interest.

The only component that is able to redistribute the flow through various tokens is the attention node, which does so according to the attention weights computed for all preceding tokens.
Furthermore, for each aggregation node (intermediate and output), the flow contribution of the preceding nodes is computed following the ``Residual contribution'' user control mentioned in~\cref{ssec:exp_intravisto_exp1} according to~\cref{eq:method_intravisto_norm-contrib,eq:method_intravisto_kl-contrib}.

InTraVisTo is equipped with additional settings exclusively dedicated to the Sankey diagram visualization as illustrated in~\cref{fig:exp_intravisto_2_B}.
First, we provide a flag for hiding the starting token similar to the one defined in~\cref{ssec:exp_intravisto_exp1}.
However, in the Sankey's case it comes with an additional control that allows users to reapport the hidden flow to remaining nodes, meaning that the values of hidden flows are set to $0$ and the missing percentage is redistributed to visible tokens.
Another important setting is the ``Attention highlight'', which controls the criteria for highlighting attention traces, affecting the number of visible flows related to each attention node.
Users can choose between visualizing all flows, only the top-$k$ flows (based on attention weights), displaying only flows with a corresponding attention weight greater than a certain threshold, or showing no flows at all.
Other graphic-oriented options allow users to remove node labels for nodes that do not constitute the output of a layer, adjust the scale of the diagram and select the depth of the visualization by choosing the number of visible layers.

\begin{figure}[t!]
    \centering
    \includegraphics[width=\textwidth]{exp_intravisto_2B_controls.png}
    \caption{Selector panel containing specific visualization controls related to InTraVisTo's Sankey diagram.}
    \label{fig:exp_intravisto_2_B}
\end{figure}

\subsection{Dynamically Changing the Network}\label{ssec:exp_intravisto_exp3}

InTraVisTo is designed as an interactive tool, and embedding injection is another feature intended to enhance understanding of the internal workings of Transformers.
As formalized in~\cref{ssec:method_intravisto_injection}, injection involves substituting a hidden state with a custom embedding representation, forcing the model to adjust its behavior based on the injected information.
Injections can be performed by clicking on a cell in the grid-layout heatmap; this actions opens a pop-up menu (\cref{fig:exp_intravisto_3_A}) associated with the cell, which then prompts the user for the following information:
\begin{itemize}
    \item A string of text with the purpose of being encoded into an embedding representation and injected inside the selected hidden state.
The application automatically converts the string into an embedding using the inverse transformation of the chosen decoder.
If the string contains multiple tokens, then all encoded representations are averaged to obtain a single embedding in accordance to~\cref{eq:method_intravisto_emb-avg}.
    \item The injection technique, which controls how the new embedding is integrated into the preexisting state, as specified in~\cref{eq:method_intravisto_inj_complete,eq:method_intravisto_inj_main}.
    \item The position of the injection inside the selected layer, allowing users to inject embeddings in every position without changing the heatmap visualization.
    \item The decoding technique used to interpret the aforementioned string of text.
    \item The option to normalize the injected embedding according to the selected injection technique.
\end{itemize}

\begin{figure}[t!]
    \centering
    \includegraphics[width=0.25\textwidth]{exp_intravisto_3A_popup.png}
    \caption[InTraVisTo's pop-up menu utilized for injections.]{InTraVisTo's pop-up menu utilized for injections, containing an input box for specifying the injection prompt and selectors for \emph{injection technique}, \emph{injection position}, \emph{decoding technique} and eventual \emph{normalization} options.}
    \label{fig:exp_intravisto_3_A}
\end{figure}

Once a valid injection is compiled and added, a small card summarizing the injection is created and displayed in a dedicated section at the top of the interface.
This operation triggers an immediate reload of the current generated result by the model to incorporate the newly defined injection.
When the heatmap visualization presents the result of a generation process that included an injection, the corresponding cell location is highlighted in \emph{green}.
Injections can be removed at any time from the generation process by pressing the appropriate `X' button on their card.
The standard layout of injection cards is shown in~\cref{fig:exp_intravisto_3_B}.

\begin{figure}[t!]
    \centering
    \includegraphics[width=0.7\textwidth]{exp_intravisto_3B_cards.png}
    \caption[Injection and ablation cards containing summaries of information that will be used to affect the model generation process.]{Two injection cards (left and center) and an ablation card (right) containing summaries of information that will be used to affect the model generation process.}
    \label{fig:exp_intravisto_3_B}
\end{figure}

As mentioned earlier in~\cref{ssec:exp_intravisto_exp1}, it is also possible to perform injections in the Sankey diagram by clicking on any visible node, triggering the same injection pop-up menu described above.
If the chosen node corresponds to a feed-forward or an attention node, there is an additional option that allows the user to remove the node, performing an ablation.
Ablations are processed similarly to injections; a card is created in the top section of the interface (\cref{fig:exp_intravisto_3_B}), and the generation process is repeated to incorporate the selected changes.
Removing a node is handled by nullifying its contribution to the residual, therefore its hidden state is still properly decoded and can be analyzed from the heatmap, but does not influence the rest of the model.
As with injections, ablations are highlighted in \emph{red} within the heatmap visualization.

\subsection{Case Study: Utilizing InTraVisTo to Inspect the Internal Behavior of Models}\label{ssec:exp_intravisto_exp4}

For this experiment we submit various prompts to a selection of different models and explore specific aspects of the extracted internal representations and information flows utilizing InTraVisTo.

It is a well known fact that current general-purpose language models have exhibited poor overall performance on tasks involving the use of numbers and mathematical operators.
Consequently, significant focus is placed on this particular topic, with some prompts being specifically set up to induce models to perform computations of numerical nature.
Nonetheless, we explore a great variety of prompts and gather valuable insight from different perspectives, with the overall goal of demonstrating the potential of the proposed tool.

\subsubsection{Experimental Setup}\label{sssec:exp_intravisto_exp4_expset}

By directly experimenting with the proposed tool, we establish a basic workflow that enables researchers to ultimately collect new insights about how LLMs generate tokens in a layer-by-layer fashion.
The first step consists in exploring the secondary representations of internal states in order to find additional information relevant for the task at hand, since the model may encode such details in the latent dimensions of hidden states alongside the main token, utilizing the residual stream as a communication channel between modules.

Based on the information previously gathered, the next step is to examine the cumulative influence on hidden states of interest generated by the model.
This can be achieved by utilizing the Sankey diagram visualization of InTraVisTo, piecing together the contribution of tokens and components that had a major role in the creation of certain intermediate states, enabling the formation of conjectures about the model's inner workings.

Finally, by acting upon these conjectures using the state injection and component ablation tools provided by InTraVisTo, it is possible to generate new knowledge by identifying the root causes that determine certain internal behaviors manifested in the preliminary inspection.
This knowledge can be further generalized by replicating the experiments on multiple models and observing possible similar mechanics at play, even between different architectures.

It is important to note that not every included discovery was made using the suggested workflow, since it only represents a general guideline to our approach.
A modest portion of our findings includes observed facts emerged from the thorough use of InTraVisTo, collected and formalized as empiric observations on the analyzed models.

\subsubsection{Dataset}

The dataset employed for this experiment consists of a small set of curated examples used to elicit model predictions in order to observe meaningful internal states.
A significant portion of the proposed prompts involves solving tasks of mathematical and numerical nature due to their simplicity and tendency towards eliciting erroneous answers from models.
This increase in the likeness of reasoning errors in LLMs for numerical tasks is often associated with \emph{attention glitches}: minor errors propagated throughout the attention pattern in the model~\cite{liu2023}.

The following constitutes a comprehensive list containing all main prompts used for visualizations and explorations in the current experiment.

\begin{multicols}{2}
    \begin{itemize}
        \item \emph{`What is the capital of Italy?'}
        \item \emph{`Write numbers in reverse order. Number: 13843234 Reverse:'}
        \item \emph{`10000000 + 1 ='}
        \item \emph{`1357 + 2291 ='}
        \item \emph{`15984 - 1 ='}
        \item \emph{`124101 - 4 ='}
        \item \emph{`124101 * 3 ='}
    \end{itemize}
\end{multicols}

\subsubsection{Models}

For the sake of our analysis we propose a limited number of models to be analyzed through InTraVisTo.
In particular, we consider the $4$-bit quantized versions of Mistral 7B instruct v0.2~\cite{jiang2023}, Llama 2 7B~\cite{touvron2023} and the full unquantized version of GPT-2~\cite{radford2019}.
However, the application has been developed with generality in mind, thus it can be deployed in such a way to include most popular model architectures available on Hugging Face~\cite{wolf2020}.

A key aspect of InTraVisTo is the inspection and modification of models' hidden states.
Unfortunately, under normal circumstances, only a small fraction of these states is made readily accessible by the standard Hugging Face Transformers library~\cite{wolf2020}, and the method for accessing internal components varies between model architectures and providers.
In order to obtain a comprehensive perspective of Transformer architectures, we utilize the \emph{Transformers-wrappers} framework\footnotemark: an open-source Python library initially developed by Vincenzo Scotti with the goal of exposing a uniform and extensible interface for Transformer models, while retaining a standardized internal structure for models provided by Hugging Face Transformers.
Specifically, we leverage the state inspection and API standardization capabilities offered by the library, and we contribute to the codebase by designing an extensible framework that supports the injection and ablation techniques utilized in InTraVisTo.
\footnotetext{\rlap{\url{https://github.com/vincenzo-scotti/transformer_wrappers}}}

\subsubsection{Results}

One of the earliest discoveries made via InTraVisTo is the fact that models which employ dedicated tokens for each digit, when asked to perform arithmetic operations, happen to represent decimal positional information along with digits of the result.
For example, the number $1\,492$ could be represented by having $1$ + ``thousand'', $4$ + ``hundred'', $9$ + ``ninety'' and $2$.
This was initially discovered through the unique usage of the iterative decoding technique specified in~\cref{alg:method_intravisto_iter-dec} to inspect secondary decodings of internal states, although it can also be observed in certain primary decodings where the `decimal term' interpretation takes precedence over the digit one.
We speculate that models intentionally use this technique to keep track of the current decimal position of the result while performing arithmetic operations.
This pattern can also be occasionally spotted in the reverse order as seen in~\cref{fig:exp_intravisto_4_A}, meaning that the model starts assigning decimal markers to the most significant digit, in what can be thought to be an attempt to circumvent the constraints entailed by carry operations.
Some relevant examples of decimal markers include:
\begin{itemize}
    \item \texttt{\_hundred}, \texttt{\_hundreds}, \begin{CJK}{UTF8}{goth}百\end{CJK} and \texttt{\_century} being used to indicate hundreds;
    \item \texttt{\_thousand} and \texttt{\_thousands} being used to indicate thousands;
    \item \texttt{\_thousands} and \begin{CJK}{UTF8}{goth}万\end{CJK} to indicate tens of thousands;
    \item \texttt{\_million} to indicate millions.
\end{itemize}

\begin{figure}[t!]
    \centering
    \includegraphics[width=0.8\textwidth]{exp_intravisto_4A_decimal.png}
    \caption[InTraVisTo heatmap visualization given the prompt \emph{``10000000 + 1 =''} to Llama 2.]{InTraVisTo heatmap visualization given the prompt \emph{``10000000 + 1 =''} to Llama 2, highlighting the use of progressive decimal markers in the residual stream representations.}
    \label{fig:exp_intravisto_4_A}
\end{figure}

Other interesting representations of numbers can be noticed by directly analyzing single instances of numerical tokens and observing their alternative representations as in~\cref{fig:exp_intravisto_4_B}.
For example:
\begin{itemize}
    \item Using month names such as ``July'' and ``February'' to represent single digit numbers such as $7$ and $2$ respectively (\cref{fig:exp_intravisto_4_B3});
    \item Using Roman numerals to represent single-digit numbers (\cref{fig:exp_intravisto_4_B2});
    \item Addressing numbers via plain text representations in English and other languages (\cref{fig:exp_intravisto_4_B1,fig:exp_intravisto_4_B2});
    \item Identifying the number $7$ with the token \texttt{\_lucky} (\cref{fig:exp_intravisto_4_B4}).
\end{itemize}

\begin{figure}[t!]
    \centering
    \begingroup
    \captionsetup{width=0.8\textwidth/2}
    \subfloat[Mistral addressing the digit $8$ via text representations.\label{fig:exp_intravisto_4_B1}]{%
        \includegraphics[width=0.5\textwidth]{exp_intravisto_4B_eight.png}%
    }%
    \subfloat[Llama 2 addressing the digit $9$ via text and alternate representations, including the Roman numeral ``IX''.\label{fig:exp_intravisto_4_B2}]{%
        \includegraphics[width=0.4\textwidth]{exp_intravisto_4B_roman.png}%
    }%
    \quad
    \subfloat[Llama 2 addressing the digit $7$ via text representations and with the name of the corresponding month of the year: ``July''.\label{fig:exp_intravisto_4_B3}]{%
        \includegraphics[width=0.45\textwidth]{exp_intravisto_4B_month.png}%
    }%
    \subfloat[Mistral addressing the digit $7$ with the token \texttt{\_lucky}.\label{fig:exp_intravisto_4_B4}]{
        \includegraphics[width=0.4\textwidth]{exp_intravisto_4B_lucky.png}
    }%
    \endgroup
    \caption[InTraVisTo heatmap visualization for various hidden states.]{InTraVisTo heatmap visualization for various hidden states, highlighting the alternative token characterizations emerging from secondary representation obtained via iterative and top-$5$ decoding.}
    \label{fig:exp_intravisto_4_B}
\end{figure}

Alternatively, we can also perform a complete analysis over a specific query, for example \emph{``What is the capital of Italy?''}, to which the model should answer \emph{`Rome'}.
By observing the heatmap in~\cref{fig:exp_intravisto_4_C1} we can see that the model reaches the correct answer at the first generated token by gradually encoding \emph{`city'}, \emph{`capital'} and finally \emph{`Rome'} into the embeddings.
However, in the final layer, the output shifts to the determiner \emph{`The'} to produce a more formal response incorporating the original question's formulation into the answer.
In addition, by observing the Sankey diagram shown in~\cref{fig:exp_intravisto_4_C2}, it is possible to notice that during the generation process of the actual occurrence of \emph{`Rome'} at position $16$, the model collects a significant amount of attention directly from the first generated token mentioned beforehand.

\begin{figure}[tp!]
    \centering
    \subfloat[Heatmap visualization calling attention to the first generated token generative process on the left.\label{fig:exp_intravisto_4_C1}]{%
        \includegraphics[width=0.7\textwidth]{exp_intravisto_4C_heatmap.png}%
    }%
    \quad
    \subfloat[Sankey diagram visualization computing flow starting from the $16$th token at layer $32$.\label{fig:exp_intravisto_4_C2}]{%
        \includegraphics[width=0.7\textwidth]{exp_intravisto_4C_sankey.png}%
    }%
    \caption{InTraVisTo visualizations given the prompt \emph{``What is the capital of Italy?''} to Mistral.}
    \label{fig:exp_intravisto_4_C}
\end{figure}

To go one step further, we attempt to sway the model's prediction toward generating a counterfactual output.
We do so by replacing the token \texttt{\_Rome} present at position $16$ with \texttt{\_Paris} using the injection interface described in~\cref{ssec:exp_intravisto_exp3}.
Interestingly, we notice a significant difference in behavior depending on the chosen layer of injection.
If we inject our token at the first occurrence of the token \texttt{\_Rome} (layer $20$), as can be observed in~\cref{fig:exp_intravisto_4_D1}, we can see that our injection quickly vanishes through the layers in favor of the correct answer.
On the other hand, if we perform our injection at a layer where the token \texttt{\_Rome} actually starts gaining a meaningful probability (around $90.45\%$ at layer $25$) the model is not able to recover immediately and, as we can appreciate in~\cref{fig:exp_intravisto_4_D2}, ends up inserting \emph{`Paris'} in its answer.
Notably, observing the rest of the generated sentence in~\cref{fig:exp_intravisto_4_D3} reveals that the model is still able to recognize the introduced error and give an overall correct answer by recontextualizing its erroneous output.

\begin{figure}[tp!]
    \centering
    \subfloat[Injection at layer $20$.\label{fig:exp_intravisto_4_D1}]{%
        \includegraphics[width=0.7\textwidth]{exp_intravisto_4D_heatmap-first.png}%
    }%
    \quad
    \subfloat[Injection at layer $25$.\label{fig:exp_intravisto_4_D2}]{%
        \includegraphics[width=0.7\textwidth]{exp_intravisto_4D_heatmap-prob.png}%
    }%
    \caption{InTraVisTo heatmap visualizations given the prompt \emph{``What is the capital of Italy?''} to Mistral with \emph{`Paris'} injected at different layers for the $16$th token.}
    \label{fig:exp_intravisto_4_D}
\end{figure}

\begin{figure}[tp!]
    \centering
    \includegraphics[width=0.9\textwidth]{exp_intravisto_4D_output.png}
    \caption{InTraVisTo input and output text given the prompt \emph{``What is the capital of Italy?''} to Mistral with \emph{`Paris'} injected at layer $25$ for the $16$th token.}
    \label{fig:exp_intravisto_4_D3}
\end{figure}

This self-correcting behavior is mildly surprising, but it is still interesting to observe the internal recovery process from the perspective of the model to understand where and how it occurs.
By directly observing the progression of tokens representing internal state of the model for the generation instance right after our injection (position $18$) in~\cref{fig:exp_intravisto_4_E1}, we can notice that the state loses confidence in the \texttt{\_Paris} token in a relatively slow way.
After starting the state morphing process from input to output, inside the middle layers, the model immediately enters a state space dominated by tokens such as \texttt{nah}, \texttt{wait}, \texttt{\_mistaken}, \texttt{\_joke}, \texttt{...} and \texttt{?!}, to finally settle on \texttt{?} with an exceptionally low confidence and surprisingly late ($47.75\%$ at layer $31$). % chktex 11
This behavior reflects the uncertainty of the model and its difficulty in integrating discordant information.
Furthermore, in~\cref{fig:exp_intravisto_4_E2} it is possible to notice a discrete amount of flow being gathered by the token referring to \emph{`Italy'} in the question through the attention component during the early stages of the model's recovery from the injection, suggesting that some form of comparison or self-check may be occurring in the earlier layers.

\begin{figure}[t!]
    \centering
    \subfloat[Heatmap visualization underlining residual states that express contradictory behaviors.\label{fig:exp_intravisto_4_E1}]{%
        \hspace{0.05\textwidth}%
        \includegraphics[width=0.2\textwidth]{exp_intravisto_4E_wrong.png}%
        \hspace{0.05\textwidth}%
    }%
    \begingroup%
    \captionsetup{width=0.63\textwidth}% 0.63 = 0.7 * 0.9
    \subfloat[Sankey diagram visualization computing flow starting from the $17$th token at layer $9$, highlighting the flow of information towards states containing \emph{`Italy'}.\label{fig:exp_intravisto_4_E2}]{%
        \includegraphics[width=0.7\textwidth]{exp_intravisto_4E_sankey.png}%
    }%
    \endgroup%
    \caption[InTraVisTo visualizations given the prompt \emph{``What is the capital of Italy?''} to Mistral with injections.]{InTraVisTo visualizations given the prompt \emph{``What is the capital of Italy?''} to Mistral with \emph{`Paris'} injected at layer $25$ for the $16$th token.}
    \label{fig:exp_intravisto_4_E}
\end{figure}

We propose another complete analytical overview starting from a query, this time of numerical nature: \emph{``Write numbers in reverse order. Number: 13843234 Reverse:''}.
By directly observing the output of the model in~\cref{fig:exp_intravisto_4_F1}, we can see that the answer is not correct since the model returned $43234831$ instead of $32341384$.
However, a closer examination of the obtained result reveals that the model has indeed tried to reverse the number by splitting it in half and recomposing the two halves in reverse order.
This behavior is confirmed by observing the token generation process using the heatmap illustrated in~\cref{fig:exp_intravisto_4_F2}.

The heatmap clearly indicates that there exists a clear step in the magnitude of token probabilities when passing from intermediate layers to final layers, a step that promptly tends to happen earlier in layers as the next token becomes more obvious.
For example, we can see that the sequence of digits that are reversed correctly by the model such as $2$, $3$, $4$ at positions $21$, $22$, $23$ and $3$, $8$, $4$ at positions $25$, $26$, $27$ appear at much earlier layers and quickly gain a consistent probability mass.
Whereas, tokens that determine the start of incorrect sections of digits such as $3$ and $1$ at positions $20$ and $24$ appear to only emerge in later layers with fluctuating probability values.
Notably, for the digit $1$ at position $24$, we can even see the model starting to predict an $8$ from the intermediate layers (which would be the correct choice), only to switch to a $1$ at layer $29$.

\begin{figure}[t!]
    \centering
    \subfloat[Input and output text focusing on the incorrect answer returned.\label{fig:exp_intravisto_4_F1}]{%
        \includegraphics[width=0.9\textwidth]{exp_intravisto_4F_output.png}%
    }%
    \quad
    \subfloat[Heatmap visualization highlighting the correlation between correct digits and model certainty in the last layers.\label{fig:exp_intravisto_4_F2}]{%
        \includegraphics[width=0.7\textwidth]{exp_intravisto_4F_heatmap.png}%
    }%
    \caption{InTraVisTo visualizations given the prompt \emph{``Write numbers in reverse order. Number: 13843234 Reverse:''} to Llama 2.}
    \label{fig:exp_intravisto_4_F}
\end{figure}

We also replicate the same experiment using Mistral, yielding a slightly more correct result.
In fact, Mistral almost entirely produces the reverse number, with the exception of two digits, $3$ and $8$, which appear inverted, as depicted in~\cref{fig:exp_intravisto_4_G1}.
On the other hand, if we take into consideration the heatmap shown in~\cref{fig:exp_intravisto_4_G2}, it is possible to observe all the previous probabilistic token patterns that were shown by Llama 2 in~\cref{fig:exp_intravisto_4_F}: in this scenario, only the first incorrect digit ($8$) displays the signs of low model confidence for its prediction.

Furthermore, as illustrated in~\cref{fig:exp_intravisto_4_G3}, it is also possible to localize which model components contribute to swaying the model residual into erroneous token representations.
We observe that tokens decoded from attention embeddings provide limited additional information and those decoded along residuals are generally similar to the final prediction, while tokens decoded from the FFNN layers occasionally show meaningful predictions.
In particular, an $8$ can be observed inside the FFNN of the $29^{th}$ layer, which corresponds to the actual correct prediction.
By comparing the behavior of FFNNs in the last layers, it is possible to notice that the correct prediction always emerges between layers $27$ and $30$, with a decreasing amount of probability as the generation proceeds.
Coincidentally, the model also seems to adjust its residual representation according to the correct token predicted by the FFNN if it is still uncertain.
According to these considerations, it would seem that the model struggles to switch its prediction into the correct computation either due to a lack of confidence in the correct prediction or to an excessive amount of confidence into the wrong prediction.

\begin{figure}[tp!]
    \centering
    \includegraphics[width=0.9\textwidth]{exp_intravisto_4G_output.png}
    \caption{InTraVisTo input and output text given the prompt \emph{``Write numbers in reverse order. Number: 13843234 Reverse:''} to Mistral.}
    \label{fig:exp_intravisto_4_G1}
\end{figure}

\begin{figure}[tp!]
    \centering
    \subfloat[Heatmap visualization highlighting the correlation between correct digits and model certainty in the last layers.\label{fig:exp_intravisto_4_G2}]{%
        \includegraphics[width=0.55\textwidth]{exp_intravisto_4G_heatmap.png}%
    }%
    \quad
    \subfloat[Sankey diagram visualization computing flow starting from the $25$th token at layer $32$, underlining single model components contributing towards an incorrect digit representation.\label{fig:exp_intravisto_4_G3}]{%
        \includegraphics[width=0.8\textwidth]{exp_intravisto_4G_sankey.png}%
    }%
    \caption{InTraVisTo visualizations given the prompt \emph{``Write numbers in reverse order. Number: 13843234 Reverse:''} to Mistral.}
    \label{fig:exp_intravisto_4_G}
\end{figure}

By performing an embedding injection earlier in the layer stack, and forcing the model's residual stream to include the correct token with higher confidence, the model is able to output the right digit ($8$).
The output following this procedure is \emph{``432348313''}.
At first, it appears that the problem concerning the two swapped digits is solved; however, the model generates an extra $3$ right at the end of the reversed number, instead of a newline character as it did before the injection.
We can confirm that $3$ is the only additional digit appended to the result by increasing the maximum count of generated tokens and observing that the next token would be, in fact, the newline character \emph{``\textbackslash{}n''}.

By analyzing the Sankey diagram in~\cref{fig:exp_intravisto_4_H1}, we can almost exactly trace back the introduction of the extra $3$ to the self-attention contribution of layer $31$, since until that point the decoded residual reads \emph{``\textbackslash{}n''}.
However, after the contribution of the implicated self-attention block we can see the residual shifting towards an embedding that represents the number $3$.
Moreover, if we inspect the attention traces that hold the most importance for the self-attention contribution of interest, we can observe that indeed they point to the question token containing a $3$, and the self-attention state itself can be decoded as the token $3$.
Once again, by utilizing a functionality of InTraVisTo and removing the contribution of that specific self-attention block, the model is able to avoid erroneously changing the main content of its residual stream.
Consequently, by ending the number with a newline character, the model is able to give the correct answer to the original query, as demonstrated in~\cref{fig:exp_intravisto_4_H2}.

\begin{figure}[t!]
    \centering
    \includegraphics[width=0.8\textwidth]{exp_intravisto_4H_sankey.png}
    \caption[InTraVisTo Sankey diagram visualization given the prompt \emph{``Write numbers in reverse order. Number: 13843234 Reverse:''} to Mistral with injection.]{InTraVisTo Sankey diagram visualization given the prompt \emph{``Write numbers in reverse order. Number: 13843234 Reverse:''} to Mistral with \emph{`8'} injected at layer $29$ for the $25$th token.}
    \label{fig:exp_intravisto_4_H1}
\end{figure}
\begin{figure}[t!]
    \centering
    \includegraphics[width=0.8\textwidth]{exp_intravisto_4H_sankey-abl.png}
    \caption[InTraVisTo Sankey diagram visualization given the prompt \emph{``Write numbers in reverse order. Number: 13843234 Reverse:''} to Mistral with injection and ablation.]{InTraVisTo Sankey diagram visualization given the prompt \emph{``Write numbers in reverse order. Number: 13843234 Reverse:''} to Mistral with \emph{`8'} injected at layer $29$ for the $25$th token and feedforward contribution removed at layer $32$ for the $28$th token.}
    \label{fig:exp_intravisto_4_H2}
\end{figure}

\subsection{Discussion}

InTraVisTo is fundamentally a tool designed to visualize internal states and the information flow within Transformer-based neural networks underlying modern LLMs.
It offers a wide array of functionalities and visualizations to NLP practitioners, which they can utilize to understand internal reasoning steps carried out by LLMs and, possibly, track down the causes of generation errors such as hallucinations.
As an interactive tool, InTraVisTo is equipped with additional tools to modify and manipulate internal representations in a user-friendly way, in order to enable small-scale causal investigations.
Moreover, InTraVisTo's implementation supports concurrent use by multiple users with the simultaneouse loading of distinct LLMs at the same time, making it a flexible application capable of being deployed reproducibly for NLP researchers and practitioners.

As extensively demonstrated in the proposed examples and experiments, InTraVisTo can be methodically used  to extract insights from the generation process of language models.
Our results in~\cref{ssec:exp_intravisto_exp4} illustrate how specific models encode numerical tokens and how these representations affect the contents of states belonging the residual flow.
We further employ the injection mechanism to investigate how anomalies in the generation process impact model outputs, uncovering common patterns and speculating on the probabilistic significance of internal states with respect to the generated tokens.
By proposing a technique for performing experiments using InTraVisTo in~\cref{sssec:exp_intravisto_exp4_expset}, we also aim to establish a workflow that acts as a guideline for users, thereby streamlining the research process and enhancing the generalizability of findings.

\section{Embedding Analysis}\label{sec:exp_emb}

Following the path laid out in~\cref{sec:rq_embeddings,sec:method_embeddings}, the experiments performed in this section will gravitate around the possibility of identifying linear properties modeling semantic, linguistic and factual relationships within the input and output embedding spaces of LLMs.

To explore this inquiry, we have set up various experiments aimed at replicating some of the well-established embeddings properties~\cite{mikolov2013} in recent state-of-the-art architectures.
Additionally, we expand upon these ideas and provide further explanations for the behaviors emerged from the performed experiments.

\subsection{Dataset}

The dataset used for all the experiments under this research question was created by integrating the original analogy dataset employed by word2vec~\cite{mikolov2013} and the \emph{BATS (Bigger Analogy Test Set)}~\cite{drozd2016}.
The word2vec dataset (also known as the Google analogy dataset) is structured in two main files: \emph{question-words} and \emph{question-phrases}.
The former contains $14$ categories for a total of $19\,544$ analogies spacing from linguistic relations to semantic relations, while the latter only features $5$ classes and a total of $3\,218$ analogies regarding common-knowledge relations involving proper nouns.
BATS, on the other hand, features a more complex hierarchy since it is meant to be an improvement over the Google dataset as indicated in~\cref{ssec:method_embeddings_analogies}.
It covers $4$ main relation types (inflectional morphology, derivational morphology, lexicographic semantics and encyclopedic semantics), each one being subdivided into $10$ further sub-categories containing $50$ unique word pairs each.
In addition, BATS provides multiple correct answers for word pairs when applicable.

Due to the different format in which the two datasets have been released (BATS containing word pairs, while the Google analogy dataset being comprised of already constructed analogies), considerable effort has been invested into collecting, standardizing and unifying analogies.
To this end, BATS word pairs have been combined into a total of $(50 \times 49) \times 10 \times 4 = 98\,000$ unique analogies to be combined with the $19\,544 + 3\,218 = 22\,762$ analogies provided by the Google dataset.
After the removal of duplicate entries, the final comprehensive total of unique analogies amounts to $116\,639$.

A peculiarity of the BATS dataset is the presence of multiple correct answers for a subset of word pairs.
These multiple correct answers include synonyms, alternative spellings or terms that are equivalent to the original answer.
Given the computational constraints of the experiments, only the first correct answer for any given analogy was taken into consideration.
Nonetheless, small empirical tests on a restricted set of models and a fraction of the dataset were conducted using multiple answers in order to assess the impact of the decision.

Additionally, a marginal set of entries belonging to the chosen datasets featured entities composed of multiple words separated by the underscore character `\_'.
To improve the parsing process for the models' tokenizers, these occurrences were changed to use the whitespace character (` ') instead.

\subsection{Experiment: Comparing Language Models \texorpdfstring{\linebreak}{} Inside the Embedding Space}\label{ssec:exp_emb_exp1}

As a starting point, we directly compare the performance of the input embeddings belonging to various state-of-the art and older models over the defined analogy task.
The comparison between the embeddings of old and new models is crucial, since it allows us to make assessments over the effectiveness of newer architectural paradigms in creating meaningful embedding spaces.
We expect to observe similar results for architectures that are closely related to one another, while also accounting for differences in vocabulary and vector sizes of embeddings.

Interestingly, \citet{drozd2016} finds that scaling the vector size of embeddings has mixed effects in terms of performance when evaluating analogies with our chosen metrics.
On the other hand, an increase in vocabulary size should always constitute an improvement over analogy resolution.
However, as \citet{elhage2022} highlights, feature sparsity is the main cause for superposition, which can cause positive interference and negative biases, hindering the expressiveness of linear operations in the embedding space.
Therefore, we also question if models with large vocabularies that are not backed by an appropriate embedding size may be at a disadvantage for the task at hand.

In essence, this experiment aims to verify the validity of the proposed inquiries, while taking into consideration possible limitations in embedding expressiveness for novel architectures.
Consequently, we expect to observe a clear distinction between the results achieved by newer models against older architectures, not necessarily due to differences in embedding quality but also to dimensional dissimilarities.

\subsubsection{Experimental Setup}\label{sssec:exp_emb_exp1_expset}

At its core, this class of experiments evaluates the resolution of analogies between sets of four terms utilizing the embedding layer of a model to encode words and compute distances.

Once the dataset and model are loaded, each batch of word analogies is processed, treating every single analogy as an independent computation.
All words that compose the input section of the analogy are appropriately encoded following~\cref{eq:method_embeddings_multitok-in}.
The selection of input words for each batch of analogies is determined by a hyperparameter containing the analogy layout (e.g.\ obtaining $w_1 - w_2 + w_4 = w_3$ from $w_1 : w_2 = w_3 : w_4$).
The provided layout also defines the arithmetic operations to be performed on the encoded words in order to obtain the encoded output term.

Afterwards, by following~\cref{eq:method_embeddings_analogy}, we perform a search on the model's embedding space using a hyperparameter-defined distance metric as shown in~\cref{eq:method_embeddings_distance}, obtaining the $k$ closest elements to the result of our embedding arithmetic.
Finally, in order to extrapolate a concrete result, we compare the set of computed tokens with the set of tokens obtained through the application of~\cref{eq:method_embeddings_multitok-out} to the output word defined by the layout.
The comparison is performed through the use of the appropriate metrics formalized in~\cref{eq:method_embeddings_topk-accuracy}.

% TODO
%~\cref{eq:method_embeddings_topk-accuracy,eq:method_embeddings_rankscore}

The complete set of hyperparameters and their possible values is as follows:
\begin{itemize}
    \item $\gbm{k}$ with \emph{positive integer values greater than $0$}: represents the maximum numbers of tokens considered when computing the closest tokens to the embedding returned by the analogy computation.
    \item \textbf{Distance metric} with values \emph{cosine}, \emph{L2}: spatial distance metric to measure element closeness inside the embedding space.
    \item \textbf{Embedding strategy} with values \emph{first\_only}: strategy used to handle input words composed of multiple tokens.
As it will be clarified later, this experiment only considers analogies where all words can be encoded using single tokens, therefore there is no need to define specific embedding strategies.
    \item \textbf{Multitoken solution strategy} with values \emph{first\_only}, \emph{subdivide}: similar concept to the embedding strategy, but handles output multi-token words by either considering the first token or every token.
Even with the assumption of single-token analogies, this hyperparameter is still relevant due to the addition of capitalized or non-capitalized alternatives to the output word, which may not result in a single-token word as their counterpart.
    \item \textbf{Pre-normalize embeddings} with \emph{boolean values}: flag that controls the use of normalized embeddings for all computations and comparisons, referencing~\cref{eq:method_embeddings_normalization}.
    \item \textbf{Layout} with values $w_1 - w_2 + w_4 = w_3$, $w_2 - w_1 + w_3 = w_4$, $w_1 + \Delta(w_4 - w_3) = w_2$: analogy resolution templates considering both classic analogies, reverse analogies and offset analogies.
Offset analogies follow a slightly different resolution process, illustrated in~\cref{eq:method_embeddings_delta-analogy-function}.
\end{itemize}

As mentioned before, due to the fact that models of dissimilar nature and with different tokenization strategies are being compared, a filtering operation over the dataset entries is applied.
This operation takes place after the dataset preprocessing pipeline and removes all analogies which contain even a single word that cannot be encoded into a single token by the model's tokenizer.

Additionally, it is possible to reduce the entire dataset to a common set of single-token analogies for all models by considering the tokenizers of multiple models at the same time.
Although, by doing this, the total amount of entries is drastically reduced, hindering the reliability of experiments.
Therefore, when applying single-token filtering, we compute a reduced dataset for each model separately.
This approach comes at the cost of having to take into account the dataset support for each model when evaluating results obtained from the experiments.

\subsubsection{Models}

Experiments are performed using the input embeddings of BERT large uncased~\cite{devlin2019}, GPT-2~\cite{radford2019}, Gemma 2 2B~\cite{rivi2024}, Llama 2 7B~\cite{touvron2023}, Llama 3 7B~\cite{dubey2024}, Mistral 7B v0.3~\cite{jiang2023} and Phi 3.5 mini instruct~\cite{abdin2024}.
Additionally, the word embeddings generated by word2vec~\cite{mikolov2013} and GloVe~\cite{pennington2014} are also included to be evaluated as a baseline.

\begin{table}[t!]
    \centering
    \begin{tabular}{| c | c c c c |}
        \rowcolorhang{bluepoli!40}
        \hline
            \textbf{Model} & \makecell{\textbf{Parameter}\\\textbf{Count}} & \makecell{\textbf{Vocabulary}\\\textbf{Size}} & \makecell{\textbf{Embedding}\\\textbf{Size}} & \makecell{\textbf{Tied}\\\textbf{Embeddings}} \\
		\hline \hline
            \textbf{Word2vec} & - & $3$M & $300$ & Yes \\[2px]
            \textbf{GloVe} & - & $400$K & $300$ & Yes \\[2px]
            \textbf{BERT large uncased} & $336$M & $30.5$K & $1\,024$ & Yes \\[2px]
            \textbf{GPT-2} & $124$M & $50.3$K & $768$ & Yes \\[2px]
            \textbf{Gemma 2} & $9$B & $256$K & $3\,584$ & Yes \\[2px]
            \textbf{Llama 2} & $7$B & $32$K & $4\,096$ & No \\[2px]
            \textbf{Llama 3} & $8$B & $128.3$K & $4\,096$ & No \\[2px]
            \textbf{Mistral v0.3} & $7$B & $32.8$K & $4\,096$ & No \\[2px]
            \textbf{Phi 3.5 mini} & $3.8$B & $32.1$K & $3\,072$ & No \\[2px]
        \hline
    \end{tabular}
    \caption{Numerical overview on model architectures.}
    \label{table:exp_emb_models}
\end{table}

\Cref{table:exp_emb_models} displays a variety of different models spanning over multiple architectures.
We observe word2vec and GloVe, older approaches solely oriented towards the creation of word embeddings, make use of entire word tokenization resulting in exceptionally large vocabularies and generally small hidden sizes.
We can also note GPT-2 and BERT, which are structurally closer the original Transformer architecture introduced by \citet{vaswani2017}, with modest vocabulary and hidden sizes.
Regarding tokenization, BERT utilizes the \emph{WordPiece} algorithm while GPT-2 employs \emph{Byte Pair Encoding (BPE)}, both being sub-word tokenization algorithms based on merge rules as seen in~\cref{ssec:background_transf_structure}.
WordPiece takes a more conservative approachby treating words without meaningful support as unknown and representing them with a special token (e.g.\ \texttt{[UNK]}), whereas BPE decomposes uncommon sequences to a series of individual characters.
On the other hand, Gemma, Llama 2, Llama 3, Mistral and Phi 3.5 are all different variations of the modern decoder-only architecture specified in~\cref{ssec:background_transf_current} using pre-normalization, Gated Linear Units and employing various versions of BPE tokenizers (Gemma, Llama 2, Mistral and Phi 3.5 use \emph{SentencePiece}\footnote{\rlap{\url{https://github.com/google/sentencepiece}}}, while Llama 3 uses \emph{tiktoken}\footnotemark).
Most of these modern models have disjointed embeddings, except for Gemma, which still ties the embedding and unembedding matrices, allowing for a greater overall vocabulary size.

\footnotetext{\rlap{\url{https://github.com/openai/tiktoken}}}

The evaluation of embeddings sourced from such a diverse set of models cannot be fair in nature.
More specifically, no effort has been made to align the datasets on which these architectures were trained, opting for comparing their most popular iterations instead.
The chosen approach has clear limitations, but it is not within our scope to evaluate these models with the intention of finding the best-performing model.
Rather, we are interested in assessing the presence of high-level trends which justifies the employment of an approach of less granular nature.

\subsubsection{Results}\label{sssec:exp_emb_exp1_results}

By observing the preliminary set of results in~\cref{fig:exp_emb_1_A1} it is apparent that Llama 2 and Mistral present the best overall performance against other models.
However, the performance gap against Gemma, Phi 3.5, and surprisingly, GPT-2 and BERT is minimal as shown in~\cref{fig:exp_emb_1_A2}.
The fact that Mistral and Llama 2 behave similarly is not unexpected, given the fact that they share most of their architecture.
However, despite the fact that Llama 2 and Llama 3 have also similar architectures, Llama 3 demonstrates an exceptionally low performance on the experiment.
Lastly, word2vec and GloVe, the oldest models, share the worst performance overall as it is possible to observe in~\cref{fig:exp_emb_1_A3}.

\begin{figure}[t!]
    \centering
    \subfloat[Accuracy for Llama 2, Llama 3 and Mistral.\label{fig:exp_emb_1_A1}]{%
        \includegraphics[width=0.5\textwidth]{exp_emb_1A_topk_l2-m-l3.pdf}%
    }%
    \subfloat[Accuracy for Gemma, Phi 3.5 and BERT.\label{fig:exp_emb_1_A2}]{%
        \includegraphics[width=0.5\textwidth]{exp_emb_1A_topk_g2-p3-b.pdf}%
    }%
    \quad
    \subfloat[Accuracy for word2vec and GloVe.\label{fig:exp_emb_1_A3}]{%
        \includegraphics[width=0.5\textwidth]{exp_emb_1A_topk_w2v-gv.pdf}%
    }%
    \caption{Top-$k$ accuracy for the word analogies task.}
    \label{fig:exp_emb_1_A}
\end{figure}

If we observe the accuracy trends of various models as $k$ increases, we can notice that the performance of some models grows in an abnormal way.
For instance, while Llama 3, Mistral, Phi and other newer models demonstrate an exceptional growth rate for increasing values of $k$, this behavior is not found for some older models such as GloVe and word2vec.
On the other hand, most models seem to present a plateau in performance around $k = 30$ except for Llama 3, which accuracy appears to be steadily increasing even at $k = 50$.
This patterns can be interpreted according to the vocabulary dimension the analyzed models, excluding older ones.
In fact, it is possible to observe a direct correlation between model vocabulary size and jump in performance between $k$ values from $1$ to $50$, which can be explained by virtue of exact tokens being more difficult to find in a more ``populated'' embedding space.
As directly shown in~\cref{fig:exp_emb_1_B}, some extreme examples are Llama 3 and Gemma with the largest vocabularies resulting in the widest accuracy jumps, and BERT with the smallest vocabulary inducing the narrowest accuracy jump.

\begin{figure}[t!]
    \centering
    \includegraphics[width=0.5\textwidth]{exp_emb_1B_topk_l3-g2-b.pdf}
    \caption[Top-$k$ accuracy for the word analogies task highlighting the accuracy differential between $k$ values for various models.]{Top-$k$ accuracy for the word analogies task highlighting the accuracy differential between $k$ values for Gemma, Llama 3 and BERT.}
    \label{fig:exp_emb_1_B}
\end{figure}

As previously noted, Llama 3 constitutes a performance outlier for newer models, since its accuracy does not hold up to other similar models.
This surprising result cannot be entirely attributed to differences in vocabulary size or embedding dimensions, since Gemma shows results comparable with the remaining models while having a vocabulary that is two times bigger and an embedding size slightly smaller than Llama 3.
We suggest that this is a direct consequence of the explicit multilingual training performed on Llama 3, which resulted in the creation of a token vocabulary that does not only reflect English language statistics, but also includes a significant portion of tokens from various other languages.
This property negatively impacts the embedding space expressiveness in two main ways.
First, tokens representing concepts in other languages are present in the vocabulary, resulting in a noisier embedding space and possibly fragmenting longer English words that contain them as sub-tokens.
Then, tokens that are shared between languages but have different senses or are part of words with different definitions determine the presence of embeddings that are in direct competition to give additional separate meanings to the same tokens.
% TODO: mention gemma

Another important fact that must be taken into consideration is the support of the resulting dataset for each model.
As anticipated in~\cref{sssec:exp_emb_exp1_expset}, models have been evaluated on different subsets of the overall dataset in order to extract analogies that could be expressed solely using single-token words.
In~\cref{fig:exp_emb_1_C} we can observe a breakdown by dataset categories of model performance represented by colored bars, with the additional information provided by dots constituting an indication for the percentage of dataset considered by the model for each category.

\begin{figure}[t!]
    \centering
    \includegraphics[width=\textwidth]{exp_emb_1C_cat_w2v-gv-l2-l3.pdf}
    \caption[Dataset categories breakdown for accuracy in the word analogies task for various models.]{Dataset categories breakdown for accuracy in the word analogies task for word2vec, GloVe, Llama 2 and Llama 3.}
    \label{fig:exp_emb_1_C}
\end{figure}

By examining~\cref{fig:exp_emb_1_C} it is possible to discern the reason for the underperformance of word2vec and GloVe with respect to newer models, as we can see that these older models offer a near complete support for all the dataset categories.
This implies that their vocabulary contains a great number of tokens encompassing single words, which drastically expands the number of analogies considered for the experiment, including analogies that the model may not readily solve.
More broadly, we also observe the fact that some certain dataset categories never seem to have support for any model.
These categories primarily represent the additional portion of the original Google dataset called \emph{question-phrases}, which models analogies between entities that are described by multiple words and intuitively do not fit the single-token constraint chosen for this experiment.
Another general pattern observable in~\cref{fig:exp_emb_1_C,fig:exp_emb_1_D1} concerns the fact that most models show better performance on the categories belonging to the original Google dataset.
This behavior can be easily explained by both the simplistic nature of the dataset and the fact that, being a well-known dataset, most models have seen it during training.
Lastly,~\cref{fig:exp_emb_1_C} shows an exceptionally low support across all categories for Llama 3, reinforcing our previous hypothesis regarding a fragmented vocabulary resulting from explicit multilingual training.

Finally, we discuss the effect of hyperparameters on this experiment.
The hyperparameter that causes the most visible effects is the analogy layout, representing the methodology used to combine words present in analogies.
In~\cref{fig:exp_emb_1_D2} we can notice that analogies of the type $w_1 - w_2 + w_4 = w_3$, seem to always return worst results than default ones: $w_2 - w_1 + w_3 = w_4$; this is likely the result of most analogies being created with a specific directionality in mind.
Nonetheless, by observing~\cref{fig:exp_emb_1_D1}, we can clearly see that this effect is mostly confined to categories that belong to the BATS dataset, as the performance for the Google dataset appears to be largely unaffected by hyperparameter combinations.
On the other hand, the pre-normalization of embeddings does not seem to have any major effect on any of the models, besides a slight generalized increase in performance.

\begin{figure}[t!]
    \centering
    \subfloat[Dataset categories breakdown for the word analogies task.\label{fig:exp_emb_1_D1}]{%
        \includegraphics[width=1\textwidth]{exp_emb_1D_cat_l2-l3-e3e4.pdf}%
    }%
    \quad
    \subfloat[Top-$k$ accuracy for the word analogies task.\label{fig:exp_emb_1_D2}]{%
        \includegraphics[width=0.5\textwidth]{exp_emb_1D_topk_l2-l3-e3e4.pdf}%
    }%
    \caption{Word analogies task for Llama 2 and Llama 3 considering classic and reverse analogy layouts.}
    \label{fig:exp_emb_1_D}
\end{figure}

Examining~\cref{fig:exp_emb_1_E} reveals that offset analogies offer some interesting insight into the behavior of models.
From a general perspective, most models experience an overall slight decrease in performance when resolving analogies using the offset formulation.
Additionally, the growth rate according to $k$ appears to less pronounced, implying that meaningful predictions are concentrated around low values of $k$, and that models are less likely to come up with the correct answer if it is not immediately identified.
This behavior can be ascribed to the analogy contribution of the offset embedding term, which likely involves a limited number of dimensions that transport the resulting embedding toward a section of the space containing terms not directly related to the analogy.
A plausible explanation for the aforementioned behavior is that the averaging operation referenced in~\cref{eq:method_embeddings_delta-analogy-function} may not be sufficient to eliminate noise or inherent correlations from a set of pairwise word comparisons sourced from the same set of analogies.
One notable case for offset analogies is Llama 3, which appears to differ significantly from its standard counterpart.
For instance, in~\cref{fig:exp_emb_1_E2} we can observe that Llama 3's baseline in the offset analogies case remains mostly unaltered and follows the performance pattern of most models.
Conversely, the actual model's performance sees a significant increase for low values of $k$, and a slight decrease for high values of $k$, which matches the previously identified flattening pattern.
However, the extreme improvement over the baseline suggests that most of the embedding contribution that leads to correct answers is given by the offset term.
Consequently, given the flat trends shown by most models, it is plausible to speculate that the overall contribution is largely dependent on the offset term.

\begin{figure}[t!]
    \centering
    \subfloat[Accuracy for BERT, word2vec and Mistral.\label{fig:exp_emb_1_E1}]{%
        \includegraphics[width=0.5\textwidth]{exp_emb_1E_topk_w2v-bd-m-dnd.pdf}%
    }%
    \subfloat[Accuracy for Llama 2 and Llama 3.\label{fig:exp_emb_1_E2}]{%
        \includegraphics[width=0.5\textwidth]{exp_emb_1E_topk_l2-l3-dnd.pdf}%
    }%
    \caption{Top-$k$ accuracy for the word analogies task considering classic and offset analogy layouts.}
    \label{fig:exp_emb_1_E}
\end{figure}

\subsection{Experiment: Analyzing Input and Output \texorpdfstring{\linebreak}{} Embeddings of LLMs}\label{ssec:exp_emb_exp2}

For this second experiment we wish to delve deeper into the actual capabilities of the embeddings belonging to recent decoder-only LLMs.
To this end, we will also take into consideration the unembedding layer of said models.
As mentioned in~\cref{ssec:background_transf_structure}, decoder-only Transformers feature a reversed embedding matrix inside their language modeling head.
This matrix is used to translate the last hidden state produced by the Transformer stack into the index for the next predicted token.
Since it shares a common structure with the actual embedding matrix, it should be possible to run the same analogy experiments and obtain meaningful, albeit different, results.

However, not all model architectures provide different embedding and unembedding matrices as already hinted in~\cref{ssec:background_transf_structure}.
Consequently, in this experiment we will also observe the differences between these architectural choices through the lens of analogies.

Additionally, inspired by the decoding approach introduced for the first research question (\cref{sec:rq_intravisto}), we are also going to take into consideration interpolated embeddings.
Embedding interpolation, defined in~\cref{ssec:method_intravisto_decoding} and formalized through~\cref{eq:method_intravisto_linear-interp,eq:method_intravisto_quadratic-interp}, is a novel technique that exploits linear properties of embedding spaces and hidden representations~\cite{park2023, mikolov2013, drozd2016} to generate intermediate unembedding matrices with the purpose of decoding said intermediate states.
Logically, since we are performing a linear interpolation operation, we expect to observe a trend in the analogy resolution accuracy of the interpolated embeddings that goes from the input embedding to the output one according to the interpolation percentage.
However, it could be possible that the actual rate of change for the interpolated performance might not be constant between the two end points, undermining the linearity assumption of linear interpolation, and opening up different approaches to interpolation-based decoding as shown in~\cref{eq:method_intravisto_quadratic-interp,eq:method_intravisto_max-p}.

\subsubsection{Experimental Setup}

The experimental setup for this second experiment presents only few differences with respect to the previous one (\cref{ssec:exp_emb_exp1}).
As a starting point, the core algorithm used to perform analogy resolution defined in~\cref{sssec:exp_emb_exp1_expset} remains unchanged, as the fundamental task is identical between the two experiments.
On the other hand, one of the main differences is the absence of any kind of filtering for the dataset, ensuring that all models share the same amount of test cases.
This change also imposes certain restrictions on the selection of models, which will be discussed in a dedicated section (\cref{sssec:exp_emb_exp2_models}).

Due to the fact that models may encounter multi-token words, the \emph{Embedding strategy} hyperparameter, identified within the experimental setup of the previous experiment, includes additional values to handle the encoding of multiple tokens converging inside a single word.
These new values are \emph{average} and \emph{sum}, which correspond to taking respectively the mean and the sum of the embeddings belonging to the tokens that compose the multi-token word in question.
As previously mentioned, these techniques are formalized in~\cref{eq:method_embeddings_multitok-in}.

\subsubsection{Models}\label{sssec:exp_emb_exp2_models}

For the experiment at hand we select a smaller set of models to compare due to the fact that we wish to observe the differences between input and output embeddings.
All changes to the models will be made referencing~\cref{table:exp_emb_models}, defined for~\cref{ssec:exp_emb_exp1}.

As a first step, we are going to discard models without proper sub-word tokenization such as word2vec, GloVe and BERT\@.
This allows us to also perform tests on the whole dataset, through the implementation of proper encoding strategies as defined in~\cref{eq:method_embeddings_multitok-in,eq:method_embeddings_multitok-out}.
In addition, we are not going to focus on models without distinct input and output embeddings such as GPT-2 and Gemma, therefore only GPT-2 will be included in order to provide a baseline evaluation.

\subsubsection{Results}

A preliminary examination of the results in~\cref{fig:exp_emb_2_A} suggests that there are some slight differences in performance between input and output embeddings across all models.
More specifically, our analysis reveals that for increasing values of $k$, output embeddings seem to consistently outperform input embeddings (contextually to overall model performance), while for smaller $k$ values input embeddings demonstrate greater accuracy.
This phenomenon suggests that input embeddings provide more reliable estimates for analogies results, but their vectors are laying in sparser latent spaces, thus making it less likely to accidentally include the correct vector out of a wrong prediction by extending the range given by $k$.
In contrast, the latent spaces of output embeddings are more compressed, allowing them to benefit more from larger $k$ values.
Finally, we show that input and output embedding pairs from to the same model tend to exhibit similar trends and are generally more prone to schieve similar accuracies.
This is outcome is not completely trivial, as despite sharing the same dimensionality and potentially some training datasets (depending on the training setup and initialization), they still operate with completely different weight matrices and fulfill distinct functions within the models.

\begin{figure}[t!]
    \centering
    \subfloat[Accuracy for GPT-2, Mistral and Phi 3.5.\label{fig:exp_emb_2_A1}]{%
        \includegraphics[width=0.5\textwidth]{exp_emb_2A_topk_gpt-m-p3-io.pdf}%
    }%
    \subfloat[Accuracy for GPT-2, Llama 2 and Llama 3.\label{fig:exp_emb_2_A2}]{%
        \includegraphics[width=0.5\textwidth]{exp_emb_2A_topk_gpt-l2-l3-io.pdf}%
    }%
    \caption{Top-$k$ accuracy for the word analogies task considering input and output embeddings on the complete dataset.}
    \label{fig:exp_emb_2_A}
\end{figure}

Comparing Llama 3's performance against the previous experiment reveals a remarkable increase in accuracy for lower values of $k$, which is now comparable to that of the other analyzed models, as shown in~\cref{fig:exp_emb_2_A2}.
This result is in clear contrast with the trends displayed by other models between the two experiments, as they tend to have a slightly worse performance when considering the whole dataset.
We can still speculate that the reason for this surprising behavior resides in the multilingual nature of Llama 3, and how it affects its tokenization process.
In fact, it is very possible that a part of the analogies easily solvable by the model were omitted from the dataset of the first experiment due to uncommon tokenization patterns.
Another perspective which corroborates this theory is the fact that the Llama 3's baseline also grows in accordance with its recorded performance, implying that the positive impact of the new dataset is not necessarily tied with the analogy resolution capabilities of the model itself.
Nonetheless, Llama 3's accuracy remains the lowest among all considered models. 

Another interesting finidng that can be garnered from the performed experiments concerns the effects of interpolation on the semantic expressiveness of embeddings.
As it is possible to observe in~\cref{fig:exp_emb_2_B1}, we can say that our expectations on the general layout of the results for interpolation tests on Mistral are indeed met, as the accuracy scores consistently fall between the extremes that produced the interpolated variations.
In addition, it is also possible to spot an interesting diverging pattern in how the interpolated traces are placed along the graph.
In Mistral's case, the performance of the interpolation at layers $7$ and $25$ is much closer to that of their corresponding input and output references models, whereas the interpolation at layer $15$ deviates more significantly.
This is surprising given that linear operations are performed and the chosen interpolation layers are all equally spaced, suggesting that the representation shift happening inside the model might not follow a linear arte through the layers.
However, this pattern is evident only for Mistral in~\cref{fig:exp_emb_2_B1}; for Llama 2 the same patterns are apparent only for small values of $k$ as shown in~\cref{fig:exp_emb_2_B2}.
This behavior is likely due to the minimal performance difference present between Llama 2's input and output embeddings, and noise, which even allows some interpolated models to surpass the two original embeddings for high values of $k$.

\begin{figure}[t!]
    \centering
    \subfloat[Accuracy for Mistral.\label{fig:exp_emb_2_B1}]{%
        \includegraphics[width=0.5\textwidth]{exp_emb_2B_topk_m-7-15-8-io.pdf}%
    }%
    \subfloat[Accuracy for Llama 2.\label{fig:exp_emb_2_B2}]{%
        \includegraphics[width=0.5\textwidth]{exp_emb_2B_topk_l2-7-15-8-io.pdf}%
    }%
    \caption[Top-$k$ accuracy for the word analogies task considering input, output and interpolated embeddings on the complete dataset.]{Top-$k$ accuracy for the word analogies task considering input, output and interpolated embeddings at layers $7$, $15$ and $23$ on the complete dataset.}
    \label{fig:exp_emb_2_B}
\end{figure}

Additionally, we can appreciate some particular model behaviors regarding the choice of hyperparameters, especially those defining the multi-token embedding strategy.
As shown in~\cref{fig:exp_emb_2_C}, for the classic analogy resolution pattern ($w_2 - w_1 + w_3 = w_4$) averaging the embeddings appears to be the overall best way to encode multi-token elements, followed by considering only the embedding of the first token and directly summing the embeddings.
This result reconfirms the findings reported by~\citet{drozd2016} for newer model paradigms and aligns with the hypothesis of a linear embedding space.

\begin{figure}[t!]
    \centering
    \includegraphics[width=0.5\textwidth]{exp_emb_2C_topk_l2-p3-afs.pdf}
    \caption[Top-$k$ accuracy for the word analogies task considering output embeddings with different multi-token encoding strategies on the complete dataset.]{Top-$k$ accuracy for the word analogies task considering output embeddings with \emph{average}, \emph{first\_only} and \emph{sum} multi-token encoding strategies on the complete dataset.}
    \label{fig:exp_emb_2_C}
\end{figure}

On the other hand, as observed in~\cref{fig:exp_emb_2_D2}, the expected drop in performance for reverse analogies ($w_1 - w_2 + w_4 = w_3$) is much more pronounced when analyzing the complete dataset rather than when filtering for single-token analogies.
This behavior is noticeable to the point that the performance of most models appears to be comparable to or below the baseline for some specific sets of hyperparameters which include reverse analogies in the specified format.
Moreover, by observing~\cref{fig:exp_emb_2_D1} we can notice that the drop in performance appears to be more widespread across all dataset categories, rather than being mostly confined to the BATS analogies as found in the previous experiment.
Newer models seem more susceptible to these types of hyperparameter variations, whereas GPT-2 gives the impression of being only marginally affected by them.

On the topic of pre-normalization, the effects seem to be negligible and similar to what was already stated in~\cref{sssec:exp_emb_exp1_results}.
Interestingly, pre-normalization appears to have slightly better results when considering the summation of embeddings as a strategy to resolve multi-token analogies, as shown in~\cref{fig:exp_emb_2_D3}.
This behavior can be intuitively explained by considering the fact that directly summing dimensions without dividing by the variable number of elements (as done the averaging case instead) produces inflated values, which are somewhat mitigated by the pre-normalization operation.

\begin{figure}[t!]
    \centering
    \subfloat[Dataset categories breakdown for the word analogies task considering input and output embeddings with classic and reverse analogy layouts.\label{fig:exp_emb_2_D1}]{%
        \includegraphics[width=\textwidth]{exp_emb_2D_cat_m-l3-i-e3e4.pdf}%
    }%
    \quad
    \begingroup
    \captionsetup{width=0.9\textwidth/2}
    \subfloat[Top-$k$ accuracy for the word analogies task considering input and output embeddings with classic and reverse analogy layouts.\label{fig:exp_emb_2_D2}]{%
        \includegraphics[width=0.5\textwidth]{exp_emb_2D_topk_m-l3-i-e3e4.pdf}%
    }%
    \subfloat[Top-$k$ accuracy for the word analogies task considering input and output embeddings with and without pre-normalization.\label{fig:exp_emb_2_D3}]{%
        \includegraphics[width=0.5\textwidth]{exp_emb_2D_topk_m-l3-io-norm.pdf}%
    }%
    \endgroup
    \caption{Word analogies task for Mistral and Llama 3 on the complete dataset.}
    \label{fig:exp_emb_2_D}
\end{figure}

Regarding offset analogies, observations from~\cref{fig:exp_emb_2_E1,fig:exp_emb_2_E2} unveil a trend that significantly differs from the one identified for the single-token analogies in~\cref{sssec:exp_emb_exp1_results}.
Even if the overall curve flattening tendency appears to carry over between experiments, in the current case we also observe an overall performance drop across all values of $k$.
Conversely, for some models the results of the input and output embeddings seem to converge, in contrast to those in the standard analogy experiment, as shown in~\cref{fig:exp_emb_2_E3}.
The reason behind this set of discrepancies between experiments is likely tied to the double averaging computation performed on the tokens constituting the analogy components: first, to optimally aggregate multi-token words into a single embedded representation as~\cref{eq:method_embeddings_multitok-in}, and second, as prescribed by the offset analogies experiment referenced in~\cref{eq:method_embeddings_delta-analogy-function}.
This conjecture is further reinforced by the fact that the performance gain over the baselines is less pronounced in this instance of the offset analogies experiment, which considers the entire dataset rather than only single-token analogies, as observable in~\cref{fig:exp_emb_2_E1,fig:exp_emb_2_E2}.

\begin{figure}[t!]
    \centering
    \subfloat[Accuracy for input embeddings.\label{fig:exp_emb_2_E1}]{%
        \includegraphics[width=0.5\textwidth]{exp_emb_2E_topk_gpt-l2-l3-m-p3-i.pdf}%
    }%
    \subfloat[Accuracy for output embeddings.\label{fig:exp_emb_2_E2}]{%
        \includegraphics[width=0.5\textwidth]{exp_emb_2E_topk_gpt-l2-l3-m-p3-o.pdf}%
    }%
    \caption[Top-$k$ accuracy for the word analogies task considering offset analogy layouts for various models on the complete dataset.]{Top-$k$ accuracy for the word analogies task considering offset analogy layouts for Llama 2, Llama 3, Mistral, Phi 3.5 and GPT-2 baseline on the complete dataset.}
    \label{fig:exp_emb_2_E}
\end{figure}
\begin{figure}[t!]
    \centering
    \includegraphics[width=0.5\textwidth]{exp_emb_2E_topk_p3-io-dnd.pdf}
    \caption{Top-$k$ accuracy for the word analogies task considering input and output embeddings with classic and offset analogy layouts on the complete dataset.}
    \label{fig:exp_emb_2_E3}
\end{figure}

\subsection{Discussion}\label{ssec:exp_emb_discussion}

In summary, we can affirm that LLMs are able to retain a surprising amount of semantic relationships inside their embeddings, despite the fact that sub-word tokenization actively works against the accumulation of meaning for token representations.
Their performance on the selected tasks is comparable, if not greater than less recent models used as baselines.
In particular, by comparing~\cref{table:exp_emb_models} and other experimental results presented in~\cref{fig:exp_emb_1_A,fig:exp_emb_1_B}, we can almost observe a slight correlation between vocabulary size and overall performance on tasks where models featuring smaller vocabularies seem to outperform those with larger ones.
From this, we can observe that very large, recent models such as Llama 3 and Gemma may be trending towards a direction of semantic impoverishment of the embedding space, whereas slightly older LLMs like Llama 2 and Mistral, as well as models designed for  compactness and portability, such as Phi 3.5, still preserve many of the original embedding space properties.

Interestingly, we observe that the only LLM that clearly underperformed on the proposed tasks was the only one that had undergone explicit multilingual training.
As previously stated, this finding suggests that multiple languages compete in giving different meaning to the same tokens, resulting in worse overall performance for the given tasks.
We must acknowledge the fact that these results do not constitute the focal point of our analysis, and most definitely require further experimentation to provide meaningful answers by expanding upon the provided conjectures.
Nonetheless, our observations remain significant for the defined research question, as they indicate that an increase in scale does not directly enhance the performance on the proposed tasks, nor does it necessarily improve the expressiveness of the embeddings.

Regarding input and output embeddings, we can conclude that input embeddings are generally better suited for resolving analogies and, by extension, inherently contain a greater amount of semantic information with respect to output embeddings.
However, we also notice that Llama 3 displays an inverse pattern regarding this regard.
We speculate that because output embeddings display a minor amount of semantic content, in Llama 3's case they also retain a smaller portion of the multilingual information, which we identify as the primary reason for Llama 3's underperformance.
Moreover, the model's vocabulary remains identical for both  input and output embeddings, so issues inherently associated with the tokenization process itself persist.

With the hyperparameter search performed on multiple experiments, we unveil that indeed averaging embeddings appears to be a relatively robust way to aggregate multiple embeddings into a single representation.
In fact, these outcomes are fundamental in the development of the embedding interpretation logic of InTraVisTo presented in~\cref{ssec:method_intravisto_decoding} for the first research question, as they rely on the same embedding space linearity assumptions as interpolation while also providing an intuitive way to group multiple token representations for the purpose of injection, as detailed in~\cref{ssec:method_intravisto_injection} and formalized through~\cref{eq:method_intravisto_emb-avg}.

Another important finding from the current experiments, which also finds application in InTraVisTo, concerns the interpolation of embedding spaces.
In fact, from the experimental results emerges the soundness of our proposed approach from a semantic perspective, indicating that the resulting interpolated embedding spaces retain at least the same amount of information as the original embeddings.
Additionally, the observed nonlinearities between the interpolation layer and performance offset relative to the original embedding spaces in related experiments have strongly motivated the creation of alternative interpolation methods, such as~\cref{eq:method_intravisto_quadratic-interp,eq:method_intravisto_max-p}

\section{First Order Prediction}\label{sec:exp_fom}

Building on the foundation established in~\cref{sec:rq_fom}, the experiments in this section are focused on exploring the implications of constructing a First Order Model (FOM).
A FOM is derived by removing all intermediate architectural components from an LLM, retaining only the input and output embedding layers along with the residual connections joining them.
As described in~\cref{sec:method_fom}, this transformation can be seen as the creation of a Markov whose transition matrix is given by the product of the input and output embedding matrices.

Most of the experiments in this section are geared towards understanding whether the FOM accurately represents a bigram Markov model over the original model's vocabulary.
Although past work has theorized and partially demonstrated this hypothesis for a restricted set of models~\cite{elhage2021}, we suspect that FOMs derived from different LLMs may exhibit varying degrees of Markovian behavior, and we aim to investigate this variation in this section.
Another perspective taken in consideration, is the possibility of the FOM transition matrix approximating an identity matrix, thus suggesting that input and output embeddings tend to converge towards equality during training, effectively modeling a form of natural weight tying~\cite{inan2017,press2017}.

Following on the normalization intuition that proved effective for InTraVisTo, as explored in~\cref{sssec:method_intravisto_decoding_norm}, we also explore an alternative of the basic FOM obtained from directly joining input and output embeddings.
In fact, this alternative was introduced with the intention of addressing one particular issue observed in certain experimental instances, which exhibited similarities to challenges previously encountered and resolved in relation to the research question referenced in~\cref{sec:rq_intravisto}.
The main issue can be described as an excessive flattening of the vocabulary distribution for decoded states, specifically referring to the probability distribution for the next token in the FOM transition matrix.
To mirror the previously proposed solution, the alternative model (FOM with RMS) incorporates a single RMS normalization (\cref{eq:background_rmsnorm}) step between the standard FOM embeddings.
As we will establish in this section, the proposed solution does not yield satisfactory results compared to its previous iteration.
Moreover, the performance of FOM models with RMS normalization falls below to that of their standard counterparts across the majority of tasks.

\subsection{Dataset}\label{ssec:exp_fom_dataset}

The dataset employed for this set of experiments is \emph{WikiText 103}~\cite{merity2017} (hereafter referred to as \emph{WikiText}), which consists of $1.81M$ rows of full articles sourced from Wikipedia.
This dataset is used to train the bigram Markov model and to conduct additional tests on various models by using a separate validation and test split.

In addition to \emph{WikiText}, we also utilize a set of $10\,000$ sentences randomly extracted from the training split of \emph{OpenWebText}~\cite{gokaslan2019}.
This extraction was performed in order to provide a less biased evaluation of models that were directly trained on \emph{WikiText}.

\subsubsection{Models}

Similarly to the models taken in consideration for~\cref{ssec:exp_emb_exp2}, the primary selected Transformer architectures feature distinct input and output embeddings.
From these models, we choose to analyze Llama 2 7B~\cite{touvron2023}, Mistral 7B v0.3~\cite{jiang2023} and Phi 3.5 mini instruct~\cite{abdin2024}.
In addition, Llama 3 8B~\cite{dubey2024} was initially considered.
However, its large vocabulary size ($128K$ tokens) resulted in a combinatorial explosion for the computation of its transition matrix, rendering it a computationally infeasible choice and leading to its exclusion.

The bigram Markov model is trained utilizing the training split of \emph{WikiText} for a total of $1.81M$ rows.
Its vocabulary is derived from that of the Transformer model being evaluated, ensuring 1-to-1 token comparisons for the transition matrices.
This is achieved by parsing the Markov model's training text and computing the frequencies using the tokenizer of the corresponding Transformer model.
The resulting token counts used for training Markov models are generally comparable across all selected reference models, amounting to approximately $140M$ for Llama 2, $135M$ for Mistral and $137M$ for Phi 3.5.

\subsection{Experiment: Direct Comparison of First \texorpdfstring{\linebreak}{} Order Models}\label{ssec:exp_fom_exp1}

The first experiment explores the main inquiry by means of direct matrix comparison, involving the computation of immediate metrics starting from the FOM and Markov model transition matrices.
This rather simplistic approach still provides surprising experimental results despite not being conventionally used for the comparison of probability distributions.

To obtain a more reliable estimate, we also use set similarity metrics as an alternative means of evaluating similarity between approaches.
As will be discussed in later sections, these metrics compare sets of top-$k$ model predictions across all vocabulary terms.
While this method does not account for the nuances of probability distributions, it still provides valuable insight into the comparison of greedy estimates between model predictions.

\subsubsection{Experimental Setup}\label{sssec:exp_fom_exp1_expset}

As previously introduced, the FOM transition matrix is computed by merging the input and output embeddings of the inspected models, following~\cref{eq:method_fom_fom-matrix}.
Conversely, the Markov model equivalent is derived by computing bigram statistics of tokens over the training dataset referenced in~\cref{ssec:exp_fom_dataset}.
To ensure comparability, we directly use the LLM's tokenizer to encode the text corpus used for training the Markov model, ensuring that both models share the same vocabulary.
As discussed in~\cref{sec:exp_fom} we also define an auxiliary class of FOMs that incorporates RMS normalization in the transition matrix formulation.
The specific approach is formalized in~\cref{eq:method_fom_fom-matrix-rms}, where the computation is divided between input and output embeddings.

In addition, an important detail regarding the construction of the FOM transition matrix that needs to be noted.
When the model makes a prediction, the structure of the unembedding matrix requires a softmax operation to produce a meaningful probability distribution, as already specified in~\cref{ssec:background_transf_structure}.
Consequently, in most scenarios, the FOM transition matrix cannot be directly compared against other transition matrices without first applying the specified conversion operation.
As explained in~\cref{ssec:method_fom_matrix}, the conversion from a logit transition matrix to an actual transition matrix is not performed in experiments that operate on greedy aggregations of most likely results.

Once the two main models are loaded as transition matrices, we compare them with each other and against the identity matrix.
As mentioned before, the identity matrix represents the transition matrix of a FOM derived from a Transformer-based language model with distinct input and output embeddings.
The comparison is conducted by quantifying the norm of the difference between two matrices as specified in~\cref{eq:method_fom_fom-i-comp,eq:method_fom_fom-markov-comp}.

On the other hand, we also perform comparisons by considering the transition matrices from a predictive token-by-token point of view as explored in~\cref{ssec:method_fom_pred}, where each row is represented as a set containing the column indexes corresponding to the top-$k$ values.
As a preliminary step, we compute the top-$k$ accuracy according to~\cref{eq:method_fom_topk} using different choices of $k_1$ and $k_2$ for various combinations of models.
Furthermore, we also consider some additional set-based metrics including the overlap coefficient and the Jaccard index.
These set metrics are used to compute and compare similarities between sets of distributions (transition matrices), as defined in~\cref{eq:method_fom_overlap}.
As pointed out before, in experiments involving set-based metrics and predictive comparisons, the distinction between transition matrices modeling probabilities and logit matrices is irrelevant.
This is because both the \emph{softmax} and \emph{logarithm} operators are monotonic, preserving the relative ordering of elements.

\subsubsection{Results}\label{sssec:exp_fom_exp1_results}

We begin our analysis by examining~\cref{fig:exp_fom_1_A2,fig:exp_fom_1_A1}, which depicts Llama 2's performance on the self-regression and bigram-regression tasks respectively, using the top-$k$ accuracy metric.
It is possible to notice how, for any given value of $k$, the model's accuracy in predicting the identity matrix appears to be lower than any of the reported values for the bigram Markov model accuracy comparison.
This trend is consistent across most other analyzed models, with the exception of Mistral.
In Mistral's case, shown in~\cref{fig:exp_fom_1_A4,fig:exp_fom_1_A3}, the predictions generated by the FOM appear to align more closely with the model's inputs rather than with the most probable outputs of the bigram Markov model ($k_2 = 1$).
This series of observations seems to persist for Mistral, even when considering all the more lenient variations that compute accuracy by comparing the top-$k_1$ predictions of the FOM with the top-$k_2$ predictions of the Markov model.
On the other hand,~\cref{fig:exp_fom_1_A5,fig:exp_fom_1_A4}, which depict Phi 3.5's performance, reveals that its general is similar to that of Llama 2 across both top-$k$ tasks.
The only notable deviation in Phi 3.5's behavior is its exceptionally low accuracy in predicting the identity matrix, as highlighted in~\cref{fig:exp_fom_1_A5}.

\begin{figure}[tp!]
    \centering
    \begingroup
    \captionsetup{width=0.9\textwidth/2}
    \subfloat[Accuracy between FOM and Markov model for Llama 2.\label{fig:exp_fom_1_A1}]{%
        \includegraphics[width=0.49\textwidth]{exp_fom_1A_l2-topk-markov.pdf}%
    }%
    \subfloat[Accuracy between FOM and Identity matrix for Llama 2.\label{fig:exp_fom_1_A2}]{%
        \includegraphics[width=0.49\textwidth]{exp_fom_1A_l2-topk-id.pdf}%
    }%
    \quad
    \subfloat[Accuracy between FOM and Markov model for Mistral.\label{fig:exp_fom_1_A3}]{%
        \includegraphics[width=0.49\textwidth]{exp_fom_1A_m-topk-markov.pdf}%
    }%
    \subfloat[Accuracy between FOM and identity matrix for Mistral.\label{fig:exp_fom_1_A4}]{%
        \includegraphics[width=0.49\textwidth]{exp_fom_1A_m-topk-id.pdf}%
    }%
    \quad
    \subfloat[Accuracy between FOM and Markov model for Phi 3.5.\label{fig:exp_fom_1_A5}]{%
        \includegraphics[width=0.49\textwidth]{exp_fom_1A_p-topk-markov.pdf}%
    }%
    \subfloat[Accuracy between FOM and identity matrix for Phi 3.5.\label{fig:exp_fom_1_A6}]{%
        \includegraphics[width=0.49\textwidth]{exp_fom_1A_p-topk-id.pdf}%
    }%
    \endgroup
    \caption{Top-$k$ accuracy for the next-token prediction task.}
    \label{fig:exp_fom_1_A}
\end{figure}

The previous observations seem to suggest that, for certain models, embeddings may exhibit a closer relationship to their inverse rather than serving as representations for predicting the next token based on the previous one.
However, this interpretation appears to conflict the recorded matrix distances shown in~\cref{table:exp_fom_distance}.
The matrix distance metric indicates that the FOM based on Mistral more accurately represents its corresponding Markov model than any other analyzed model, as it exhibits the lowest FOM/Markov distance and one of the highest FOM/identity distances.
Nonetheless, direct comparative statements based on the matrix distance should be approached with caution as this metric is neither widely used nor inherently reliable.

\begin{table}[t!]
    \centering
    \begin{tabular}{| c | c c c |}
        \rowcolorhang{bluepoli!40}
        \hline
        \textbf{Distance} & $\textbf{d(FOM, I)}$ & $\textbf{d(FOM, Markov)}$ & $\textbf{d(Markov, I)}$ \\
		\hline \hline
            \textbf{Llama 2} & $178.88$ & $6.37$ & $178.99$ \\[2px]
            \textbf{Llama 2 with RMS} & $180.33$ & $23.13$ & '' \\[2px]
            \textbf{Mistral} & $181.02$ & $5.95$ & $181.11$ \\[2px]
            \textbf{Mistral with RMS} & $180.99$ & $6.07$ & '' \\[2px]
            \textbf{Phi 3.5} & $179.06$ & $6.30$ & $179.16$ \\[2px]
            \textbf{Phi 3.5 with RMS} & $201.15$ & $91.53$ & '' \\[2px]
        \hline
    \end{tabular}
    \caption[Matrix distance metric between various models.]{Matrix distance metric between combinations of FOM, FOM with RMS, Markov model, and Identity matrix for Llama 2, Mistral, Phi 3.5.}
    \label{table:exp_fom_distance}
\end{table}

In fact, we can notice that all LLMs present the same trend regarding their matrix distance experiments, likely due to the noise present in their large matrices, which makes them unsuitable for direct comparison through a distance metric.
This finding confirms a potential bias towards lower distance estimates in the FOM/Markov case, possibly resulting from a lower proportion of null values present in the Markov transition matrix compared to the identity matrix.
However, slight differences between models emerge when considering FOM variants that incorporate RMS normalization.
In particular,~\cref{table:exp_fom_distance} shows that the FOMs with RMS derived from Llama 2 and Mistral exhibit a subtle bias toward the results previously observed in~\cref{fig:exp_fom_1_A}.
We speculate that this perceived trend arises because RMS normalization aligns internal representations in a way that makes probabilities appear more skewed, thus providing a better fit for their actual reference distribution.
Nonetheless, the obtained results still contain a vast amount of noise and cannot offer a reliable basis for directly comparing FOM behavior.
Furthermore, the FOM with RMS based on Phi 3.5 is a clear outlier, as its performance is significantly worse than its counterpart without RMS across other models and both considered comparisons.

On the other hand,~\cref{table:exp_fom_predictions} presents the most likely token predicted by models for a restricted set of randomly selected common input words.
This provides an approximate yet immediate view into the types of predictions generated by the extracted FOMs and their corresponding trained Markov models.
In this case, Markov model predictions are aggregated into a single entry, since they perfectly overlap for the chosen set of words.
As it is possible to observe, most next token predictions from both FOMs and Markov models can be contextualized to make sense within a natural language setting.
However, minor patterns emerge that differentiate how well FOMs extracted from different models approximate their respective Markov models.
For example, Llama 2 and Phi 3.5 frequently produce predictions aligned with Markov models.
Conversely, Mistral often repeats the input token across multiple entries, mimicking the identity transition matrix, and occasionally predicts tokens that do not appear meaningful when combined with the input token.
In addition, it is also possible to notice how FOM variants that include RMS normalization almost always return the same prediction as their counterpart without normalization.

\begin{table}
    \centering
    \resizebox{\columnwidth}{!}{\begin{tabular}{|c | c c c c c c c |}
        \rowcolorhang{bluepoli!40}
        \hline
            & \multicolumn{2}{c}{\textbf{Llama 2}} & \multicolumn{2}{c}{\textbf{Mistral}} & \multicolumn{2}{c}{\textbf{Phi 3.5}} & \textbf{Markov} \\[-0.1pt]
        \rowcolorhang{bluepoli!40}
            \multirow{-2}{*}{\textbf{Token}} & \textbf{no RMS} & \textbf{RMS} & \textbf{no RMS} & \textbf{RMS} & \textbf{no RMS} & \textbf{RMS} & \textbf{models} \\
		\hline \hline
    the     & same  & same  & ses    & klass  & entire     & entire     & \texttt{\textvisiblespace{}} \\ 
    my      & ri    & own   & own    & own    & own        & own        & life   \\ 
    in      & cis   & cent  & lc     & lc     & hib        & hib        & the    \\ 
    long    & temps & temps & long   & long   & ago        & ago        & @      \\ 
    smart   & phone & phone & phones & phones & phone      & phone      & phone  \\ 
    door    & way   & way   & door   & door   & confidence & confidence & .      \\ % chktex 26
    hear    & ings  & ings  & Pyx    & Pyx    & thy        & thy        & ings   \\ 
    \hline
    \end{tabular}}
    \caption[Visualization of the most likely token predicted given common input tokens for various models.]{Visualization of the most likely token predicted by FOM, FOM with RMS and Markov model given common input tokens for Llama 2, Mistral and Phi 3.5.}
    \label{table:exp_fom_predictions}
\end{table}

As previously mentioned, we introduce set metrics as a more reliable measure for comparing transition matrices.
Considering~\cref{fig:exp_fom_1_B1}, it is possible to notice that the two chosen set metrics appear to be roughly equivalent in assessing model comparisons.
Although the graph only includes Llama 2, set metrics exhibit the same general behavior across all analyzed models.
Substantial differences among metrics can be appreciated only for extremely low values of $k$.
We also examine the effects of RMS normalization on FOMs within the context of this experiment.
As it is possible to discern from~\cref{fig:exp_fom_1_B}, there does not appear to be any significant difference between the Jaccard index of a FOM and its counterpart with RMS for Llama 2.
This observation holds for other models and metrics as well, likely because, in this case, RMS primarily causes noticeable changes in the underlying probabily distribution while leaving the internal ordering of predictions mostly unaltered.

\begin{figure}[t!]
    \centering
    \includegraphics[width=0.5\textwidth]{exp_fom_1B_l2-opmetrics.pdf}
    \caption[Overlap coefficient and Jaccard index for Llama 2.]{Overlap coefficient and Jaccard index between FOM, Markov model and random baseline for Llama 2.}
    \label{fig:exp_fom_1_B1}
\end{figure}
\begin{figure}[t!]
    \subfloat[Overlap coefficient considering classic FOM.\label{fig:exp_fom_1_B2}]{%
        \includegraphics[width=0.5\textwidth]{exp_fom_1B_l2-jc.pdf}%
    }%
    \subfloat[Overlap coefficient considering FOM with RMS.\label{fig:exp_fom_1_B3}]{%
        \includegraphics[width=0.5\textwidth]{exp_fom_1B_l2rms-jc.pdf}%
    }%
    \caption[Jaccard index for Llama 2 considering FOM and FOM with RMS.]{Jaccard index between FOM, Markov model and random baseline for Llama 2.}
    \label{fig:exp_fom_1_B}
\end{figure}

On the other hand, comparative analysis reveals that while Llama 2 and Phi 3.5 seem to exhibit similar overall trends for their FOM/Markov equivalence, Mistral shows an entirely different profile.
As we can observe in~\cref{fig:exp_fom_1_C1,fig:exp_fom_1_C3}, disregarding raw performance, Llama 2 and Phi 3.5 present a high average overlap coefficient for low values of $k$, clearly outperforming both random baselines.
This coefficient slowly decreases until $k$ reaches a value slightly below $200$, after which it increases asymptotically, matching the pace of the random baselines.
In particular, the FOM/Markov overlap coefficient is not outperformed by the baselines for any value of $k$.
On the other hand, by observing~\cref{fig:exp_fom_1_C2} we can see that Mistral's average overlap coefficient is just slightly above the baselines for low values of $k$.
In addition, its growth rate nearly matches that of the baselines, meaning that the overlap coefficient associated to the FOM/Markov comparison is already at full capacity after the small positive jump for $k$ around $10$.

\begin{figure}[t!]
    \centering
    \subfloat[Overlap coefficient for Llama 2.\label{fig:exp_fom_1_C1}]{%
        \includegraphics[width=0.33\textwidth]{exp_fom_1C_l2-op.pdf}%
    }%
    \subfloat[Overlap coefficient for Mistral.\label{fig:exp_fom_1_C2}]{%
        \includegraphics[width=0.33\textwidth]{exp_fom_1C_m-op.pdf}%
    }%
    \subfloat[Overlap coefficient for Phi 3.5.\label{fig:exp_fom_1_C3}]{%
        \includegraphics[width=0.33\textwidth]{exp_fom_1C_p3-op.pdf}%
    }%
    \caption{Overlap coefficient between FOM, Markov model and random baseline.}
    \label{fig:exp_fom_1_C}
\end{figure}

The initial values of $k$ hold the greatest significance to the experiment as they are the primary next token predictions for the model, and after a certain value of $k$ every model regresses to the random baseline performance.
In particular, this means that there is a depth of $k$ for which the FOM is just modeling random noise.
By looking at Mistral's performance in~\cref{fig:exp_fom_1_C2}, since the overlap coefficient corresponding to the FOM/Markov comparison is parallel to that of the random baselines, we can infer that its FOM is actually modeling noise from the start, with the addition of some related terms that determine a slight improvement over the baseline nonetheless.
From this direct comparison we can clearly observe the patterns already found in previous experiments, identifying Mistral as the model which FOM exhibits the lowest affinity with a Markov model, while still approximating it to a lower degree of accuracy.
Whereas, the FOMs extracted from Llama 2 and Phi 3.5 can almost be considered faithful approximations of their respective Markov models.

\subsection{Experiment: Probabilistic Comparison of First Order Models}\label{ssec:exp_fom_exp2}

The second experiment expands upon the evaluation of transition matrices generated by concatenating input and output embeddings from Transformer models with respect to actual bigram Markov models, but via deeper means of analysis.
For this experiment, we shift our perspective on transition matrices from sets of elements generating binary predictions to actual probability distributions.
This approach enables a more precise evaluation by considering how models would concretely be utilized beyond these experimental scenarios.

In~\cref{ssec:method_fom_prob} we define the two primary mathematical tools used in this experiment: perplexity (\cref{eq:method_fom_perplexity}) and KL divergence (\cref{eq:method_fom_kldiv}).
These metrics allow us to determine the similarity or dissimilarity between two distributions over a vocabulary, either by directly comparing them using the KL divergence or by measuring their `surprise' over a text corpus using the perplexity metric.

\subsubsection{Experimental Setup}

The overall perplexity for a model over a dataset is determined by averaging the perplexity values computed according to~\cref{eq:method_fom_perplexity} over all sentences in the test dataset.
As already stated in the corresponding equation, the perplexity of each sentence is calculated by cumulating, over all tokens, the perplexity obtained by comparing the model's output logits with the one-hot encoded identifier of the subsequent token.

In this experiment, four perplexity scores are taken into consideration for each model variant.
The two main results used for central comparisons are the FOM and Markov model perplexity scores, while baseline comparisons are also made using results obtained from a uniform distribution over the model's vocabulary and the identity matrix model.

We also compute the Kullback-Leibler divergence (KL divergence) between pairs of relevant distributions in order to assess their similarity from a statistical standpoint.
The KL divergence is calculated according to~\cref{eq:method_fom_kldiv} and, as noted, is an asymmetric distance which measures how much a given distribution differs from another.
In practice, for each token, we compute point-wise contributions for every vocabulary item and then aggregate these contributions by summing them.
Subsequently, we average the computed KL divergences to obtain the average KL divergence over the entire vocabulary for each model.
Similarly to the perplexity computation, we evaluate the KL divergence for all combinations and reverse combinations of FOM, Markov model and identity matrix.

\subsubsection{Results}

Based on a preliminary analysis of the average perplexities presented in~\cref{table:exp_fom_wikitext}, we immediately note that all models exhibit similar trends in in the performance of their FOM relative to the uniform baseline and the computed Markov model counterpart.
In fact, we observe that the average perplexity from the FOM is consistently slightly lower than that of the uniform baseline while remaining considerably higher than the performance of the Markov model.
We hypothesize that this behavior is caused by a combination between the improper nature of the FOM and potential biases in the training set of the Markov model.
This last point is only marginally mitigated by the usage of a different test set, as~\cref{table:exp_fom_openwebtext} shows a significant increase in the perplexity for Markov models, although their performance remains markedly distinct from that of the FOMs.

\begin{table}[t!]
    \centering
    \begin{tabular}{| c | c c c c c |}
        \rowcolorhang{bluepoli!40}
        \hline
             & \multicolumn{2}{c}{\textbf{FOM}} & & & \\[-0.1pt]
        \rowcolorhang{bluepoli!40}
            \multirow{-2}{*}{\textbf{Perplexity}} & \textbf{no RMS} & \textbf{RMS} & \multirow{-2}{*}{\makecell{\textbf{Markov}\\\textbf{model}}} & \multirow{-2}{*}{\makecell{\textbf{Uniform}\\\textbf{probability}}} & \multirow{-2}{*}{\makecell{\textbf{Identity}\\\textbf{matrix}}} \\
		\hline \hline
            \textbf{Llama 2} & $31.94 \times 10^3$ & $189.06 \times 10^3$ & $205.38$ & $32.00 \times 10^3$ & $677.76 \times 10^9$ \\[2px]
            \textbf{Mistral} & $32.01 \times 10^3$ & $47.62 \times 10^3$ & $252.64$ & $32.77 \times 10^3$ & $675.52 \times 10^9$ \\[2px]
            \textbf{Phi 3.5} & $30.83 \times 10^3$ & $6.03 \times 10^9$ & $247.13$ & $32.06 \times 10^3$ & $676.79 \times 10^9$ \\[2px]
        \hline
    \end{tabular}
    \caption[Mean perplexity on WikiText for various models.]{Mean perplexity on WikiText for FOM, FOM with RMS, Markov model, Uniform probability and Identity matrix of Llama 2, Mistral and Phi 3.5.}
    \label{table:exp_fom_wikitext}
\end{table}

\begin{table}[t!]
    \centering
    \begin{tabular}{| c | c c c c c |}
        \hline
        \rowcolorhang{bluepoli!40}
             & \multicolumn{2}{c}{\textbf{FOM}} & & & \\[-0.1pt]
        \rowcolorhang{bluepoli!40}
            \multirow{-2}{*}{\textbf{Perplexity}} & \textbf{no RMS} & \textbf{RMS} & \multirow{-2}{*}{\makecell{\textbf{Markov}\\\textbf{model}}} & \multirow{-2}{*}{\makecell{\textbf{Uniform}\\\textbf{probability}}} & \multirow{-2}{*}{\makecell{\textbf{Identity}\\\textbf{matrix}}} \\
		\hline \hline
            \textbf{Llama 2} & $31.78 \times 10^3$ & $98.01 \times 10^3$ & $2.23 \times 10^3$ & $32.00 \times 10^3$ & $592.18 \times 10^9$ \\[2px]
            \textbf{Mistral} & $32.00 \times 10^3$ & $34.58 \times 10^3$ & $2.7 \times 10^3$ & $32.77 \times 10^3$ & $582.29 \times 10^9$ \\[2px]
            \textbf{Phi 3.5} & $29.74 \times 10^3$ & $152.72 \times 10^6$ & $2.22 \times 10^3$ & $32.06 \times 10^3$ & $591.73 \times 10^9$ \\[2px]
        \hline
    \end{tabular}
    \caption[Mean perplexity on OpenWebText for various models.]{Mean perplexity on OpenWebText for FOM, FOM with RMS, Markov model, Uniform probability and Identity matrix of Llama 2, Mistral and Phi 3.5.}
    \label{table:exp_fom_openwebtext}
\end{table}

Unsurprisingly, we observe the perplexity results of the identity matrix predictions being disproportionately higher than that of all other models, as it reflects a scenario where each token is predicted to repeat with complete certainty: a behavior that does not constitute a meaningful Markov model.
However, we still value this outcome since it provides a baseline for comparison for the FOM derived from each LLM, thereby confirming the findings from the previous experiment (\cref{ssec:exp_fom_exp1}) that FOMs are generally closer to being approximations of the actual Markov model rather than the identity matrix.
Nonetheless, we can speculate that the pattern unveiled in~\cref{sssec:exp_fom_exp1_results} appears to persist in~\cref{table:exp_fom_wikitext,table:exp_fom_openwebtext}, as the perplexity tied to the FOM based on Mistral is slightly higher than that for its Llama 2 and Phi 3.5 counterparts, possibly suggesting that it incorporates similarities to the uniform and identity matrix distributions.

One notable trend observable in~\cref{fig:exp_fom_2_A} is that when plotting perplexity over multiple sentences in the test dataset, a clear pattern of several horizontal lines emerges for the FOM version of most LLMs.
These lines vary in density and are located in the loose proximity of the perplexity level of the uniform model.
The number and position of the lines appears to be dependent on both the underlying LLM and the test dataset, although they generally tend to align with the uniform prediction.
The observed consistencies imply the existence of macro-groups of sentences for which a model consistently scores perplexity values that are marginally inferior or superior to random guessing.

\begin{figure}[t!]
    \centering
    \subfloat[Perplexity for Llama 2.\label{fig:exp_fom_2_A1}]{%
        \includegraphics[width=0.5\textwidth]{exp_fom_2A_l2-perp-owt.png}%
    }%
    \subfloat[Perplexity for Mistral.\label{fig:exp_fom_2_A2}]{%
        \includegraphics[width=0.5\textwidth]{exp_fom_2A_m-perp-owt.png}%
    }%
    \quad
    \subfloat[Perplexity for Phi 3.5.\label{fig:exp_fom_2_A3}]{%
        \includegraphics[width=0.5\textwidth]{exp_fom_2A_p3-perp-owt.png}%
    }%
    \caption{Perplexity on OpenWebText sentences for FOM, Markov model, Uniform probability and Identity matrix.}
    \label{fig:exp_fom_2_A}
\end{figure}

Examining the proposed perplexity graphs reveals some interesting differences between models.
For instance, the FOM based on Llama 2 (\cref{fig:exp_fom_2_A1}) clusters most sentences around the perplexity value corresponding to the uniform baseline, while a significant minority of sentences exhibits a consistently lower perplexity.
On the other hand, most sentences evaluated with the FOM derived from Mistral (\cref{fig:exp_fom_2_A2}) lie on a slightly lower perplexity value than the uniform baseline; however, two sizeable subsets of sentences perform considerably better and worse, respectively.
Lastly, the FOM based on Phi 3.5 (\cref{fig:exp_fom_2_A3}) reflects the same overall pattern to that of Mistral, but with notably lower perplexity values, such that only a small fraction of sentences have perplexity close to the uniform baseline.

In summary, we can affirm that the difference between FOM and uniform performance is more pronounced in the plotted results than numerical averages, as the plots more clearly reveal the outliers present in the datasets.

\todo[purple!30]{Explore some outlier sentences in depth}

Regarding the alternative FOM formulation that incorporates RMS normalization, we can clearly see a drastic and widespread drop in performance.
Even when comparing the gathered results from the previous experiment described in~\cref{ssec:exp_fom_exp1}, FOMs with RMS normalization exhibit severe underperformance, with their average perplexity appearing to be several orders of magnitude higher than both their standard FOM counterparts and the uniform baseline, as noted in~\cref{table:exp_fom_llama-kl,table:exp_fom_mistral-kl}.
One notable exception is the FOM with RMS tied to Mistral, which shows only a moderate drop in performance compared to FOMs obtained from other models.
Moreover,~\cref{fig:exp_fom_2_B} reveals an interesting phenomenon: the two perplexity groups observed in~\cref{fig:exp_fom_2_A2} appear to fragment into multiple groups with progressively increasing perplexity.
In general, it is possible that the degraded performance unveiled by the RMS normalization is indicative of actual model performance by eliminating the uniform correlation and skewing the probability distributions.
However, we consider this hypothesis unlikely, as the effect varies across models, in contrast to the findings shown in~\cref{sssec:exp_fom_exp1_results}.

\begin{figure}[t!]
    \centering
    \includegraphics[width=0.5\textwidth]{exp_fom_2B_m-rms-perp-owt.png}
    \caption{Perplexity on OpenWebText sentences for FOM with RMS, Markov model, Uniform probability and Identity matrix of Mistral.}
    \label{fig:exp_fom_2_B}
\end{figure}

In contrast to the perplexity results, outcomes obtained by computing KL divergence over the vocabulary can be considered more straightforward to interpret, as they reveal a clearer difference between the performance of models.
Interestingly, we notice that this series of outcomes does not necessarily align with previous observations; however, we can ascribe discrepancies in the results to the inherent similarity between FOM probabilities and the uniform distribution.
Following this idea, we also experience interesting behaviors when considering the FOM variant which includes RMS normalization due to its direct influence on the uniformity of FOM probabilities.

\begin{table}[t!]
    \centering
    \begin{tabular}{| >{\columncolor{bluepoli!40}}c || c c c c |}
        \hhline{-||----}
        \rowcolorhang{bluepoli!40}
            \textbf{Llama 2} $\gbm{\overline {D_{KL}}}$ & \textbf{FOM} & \makecell{\textbf{FOM}\\\textbf{with RMS}} & \Gape[0pt][1pt]{\makecell{\textbf{Markov}\\\textbf{model}}} & \Gape[0pt][1pt]{\makecell{\textbf{Identity}\\\textbf{matrix}}} \\
		\hhline{=::====}
        \textbf{FOM} & $-$ & $-$ & $0.202626$ & $10.373777$ \\[2px]
        \textbf{FOM with RMS} & $-$ & $-$ & $2.099694$ & $12.416668$ \\[2px]
        \textbf{Markov model} & $0.054688$ & $2.262181$ & $-$ & $10.388289$ \\[2px]
        \textbf{Identity matrix} & $17.260256$ & $19.609880$ & $17.458118$ & $-$ \\[2px]
        \hhline{-||----}
    \end{tabular}
    \caption[Mean KL divergence for Llama 2.]{Mean KL divergence between FOM, FOM with RMS, Markov model and Identity matrix for Llama 2.}
    \label{table:exp_fom_llama-kl}
\end{table}

\begin{table}[t!]
    \centering
    \begin{tabular}{| >{\columncolor{bluepoli!40}}c || c c c c |}
        \hhline{-||----}
        \rowcolorhang{bluepoli!40}
            \textbf{Mistral} $\gbm{\overline {D_{KL}}}$ & \textbf{FOM} & \makecell{\textbf{FOM}\\\textbf{with RMS}} & \Gape[0pt][1pt]{\makecell{\textbf{Markov}\\\textbf{model}}} & \Gape[0pt][1pt]{\makecell{\textbf{Identity}\\\textbf{matrix}}} \\
		\hhline{=::====}
        \textbf{FOM} & $-$ & $-$ & $0.163428$ & $10.375000$ \\[2px]
        \textbf{FOM with RMS} & $-$ & $-$ & $0.561153$ & $9.555813$ \\[2px]
        \textbf{Markov model} & $0.075116$ & $0.450965$ & $-$ & $10.409687$ \\[2px]
        \textbf{Identity matrix} & $17.659594$ & $17.644035$ & $17.417395$ & $-$ \\[2px]
        \hhline{-||----}
    \end{tabular}
    \caption[Mean KL divergence for Mistral.]{Mean KL divergence between FOM, FOM with RMS, Markov model and Identity matrix for Mistral.}
    \label{table:exp_fom_mistral-kl}
\end{table}

\begin{table}[t!]
    \centering
    \begin{tabular}{| >{\columncolor{bluepoli!40}}c || c c c c |}
        \hhline{-||----}
        \rowcolorhang{bluepoli!40}
            \textbf{Phi 3.5} $\gbm{\overline {D_{KL}}}$ & \textbf{FOM} & \makecell{\textbf{FOM}\\\textbf{with RMS}} & \Gape[0pt][1pt]{\makecell{\textbf{Markov}\\\textbf{model}}} & \Gape[0pt][1pt]{\makecell{\textbf{Identity}\\\textbf{matrix}}} \\
		\hhline{=::====}
        \textbf{FOM} & $-$ & $-$ & $0.201826$ & $10.410433$ \\[2px]
        \textbf{FOM with RMS} & $-$ & $-$ & $9.795715$ & $21.472530$ \\[2px]
        \textbf{Markov model} & $0.059503$ & $7.225024$ & $-$ & $10.390044$ \\[2px]
        \textbf{Identity matrix} & $17.257671$ & $24.772488$ & $17.455151$ & $-$ \\[2px]
        \hhline{-||----}
    \end{tabular}
    \caption[Mean KL divergence for Phi 3.5.]{Mean KL divergence between FOM, FOM with RMS, Markov model and Identity matrix for Phi 3.5.}
    \label{table:exp_fom_phi-kl}
\end{table}

The distribution dissimilarity between FOM and Markov model is substantially lower than that observed for any other pair of model distributions across all analyzed LLMs as shown in~\cref{table:exp_fom_llama-kl,table:exp_fom_mistral-kl,table:exp_fom_phi-kl}.
We can also notice that ${\overline {D_{KL}}}(\gbm{Q}_{Markov} || \gbm{Q}_{FOM})$ is consistently smaller than the reverse (${\overline {D_{KL}}}(\gbm{Q}_{FOM} || \gbm{Q}_{Markov})$) for all LLMs, implying that FOMs are a better approximation of Markov models than vice versa.
This result is likely tied to the difference in probability magnitude over the majority of the vocabulary terms.
Since the Markov model was trained by analyzing a restricted dataset with a set of terms smaller than the vocabulary, we believe that its distribution neglects tokens that were never seen during training, even despite the smoothing coefficient applied to token counts.
On the other hand, the FOM possesses a much more uniform-like baseline, as demonstrated in previous experiments, and therefore serves as a more appropriate reference distribution in the KL divergence computation.

Although the results are much more aligned between models with regard to previous experiments, it is still possible to partially observe the same properties tied to the FOM approximations of specific LLMs underlined in previous experiments.
For instance, in~\cref{table:exp_fom_mistral-kl} we can notice a slightly higher ${\overline {D_{KL}}}(\gbm{Q}_{Markov} || \gbm{Q}_{FOM})$ value and a slightly lower ${\overline {D_{KL}}}(\gbm{Q}_{FOM} || \gbm{Q}_{Markov})$ value for the Mistral FOM compared against other models.
Additionally, KL divergence values related to comparisons between identity matrix and FOM for the Mistral model seem to present the same behavior with respect to Llama 2 and Phi 3.5.

As far as FOMs that include RMS normalization are concerned, there are several meaningful differences brought to light by the performed experiment.
First, in~\cref{table:exp_fom_llama-kl} we can observe that the FOM based on Llama 2 performs slightly better than its alternate version including RMS\@.
This finding is unsurprising, as it conforms to the overall performance pattern established and already mentioned in previous experiments.
However, if we also consider~\cref{table:exp_fom_mistral-kl}, it is possible to notice that the performance drop tied to the FOM with RMS based on Mistral is actually much less noticeable than that of its Llama 2 counterpart.
Mistral's performance is amplified by the fact that the KL divergence between FOM with RMS and identity matrix is actually lower than the KL divergence considering the original FOM\@.
This underlines the apparent affinity that the FOM based on Mistral displays towards self-regression,
Conversely,~\cref{table:exp_fom_phi-kl} shows the substantial impact on KL divergence that RMS normalization has on the Phi 3.5 based FOM, resulting in a significant decrease in performance across all comparisons that include the FOM, in line with the findings from the previous experiment on perplexity.

\subsection{Discussion}\label{ssec:exp_fom_discussion}

Overall, the feasibility of using a FOM as a Markov model approximation appears promising, although it may vary depending on the specific choice of LLM, as some models seem to be more inclined than others.
Nevertheless, every analyzed FOM seems to exhibit a slight bias towards trying to approximate an actual bigram Markov model rather than the identity matrix.

This collection of results provides partial empirical confirmation of what was initially theorized by~\citet{elhage2021}.
Additionally, it provides solid evidence that output embeddings do not naturally tend to drift towards modeling input embeddings.
This implies that current Transformer-based LLMs might not benefit as much from weight tying as some older language models, reducing the quantifiable improvements to a marginal reduction in the number of parameters.

The results obtained in the current sectiont can be further contextualized within the InTraVisTo framework, which is centered around the concept of residual flow as previously mentioned in~\cref{ssec:exp_intravisto_exp2}.
The tendency of the FOM to act as a Markov model fits perfectly into the perspective of a residual flow that, starting from the current token, reaches a probability distribution over the vocabulary for the next token through a process of continuous refinement enacted by various modules comprising Transformer architectures.
If we strip the model of its modules, we are left with a direct communication channel that provides the simplest possible prediction given the available transformations.
By extending this interpretation and capitalizing on the fact that operations on the residual stream are performed additively, we can rationalize the purpose of InTraVisTo as a tool that can be utilized to extract partial interpretations of a state that is progressively drifting towards a final prediction.

Furthermore, we theorize the possibility of extracting unembedding weights from the transition matrix of a bigram Markov model.
These weights could see use as initialization values for the decoder weights of an LLM at training time.
This represents one of the interesting potential developments arising from the findings of the current section, which may be explored in future work.


\chapter{Conclusions and future developments}\label{ch:conclusions}
In this work, we have explored various aspect of transformer interpretability, focusing on the creation of a specialized tool that enables the visualization of internal states of transformer-based language models: \textbf{InTraVisTo}.
InTraVisTo was created with research in mind: its main purpose \todo{being of aiding} researchers and practitioners in the field of NLP to observe the internal reasoning steps carried out by LLMs in an immediate and interactive \todo{manner}.
Moreover, its easily configurable, highly optimized and multi-user oriented implementation allows for an overall user-friendly experience centered around the discovery of LLM architectures through the lens of their own embeddings.
InTraVisTo answers the first identified research question in \cref{sec:rq_intravisto} by providing an expanded and general view on the topic of vocabulary space decoding for LLMs.
The addition of the information flow visualization and the possibility of interacting with the generative process of the model from the inside offer a unique experience \todo{catered} to the preliminary exploration and investigation process performed by NLP researchers.

The development of InTraVisTo introduced a series of additional research questions that needed separate investigation in order to address \todo{important} issues and \todo{substantiate} methodologies employed in the final application.
First of these secondary inquiries is the second research question formulated in \cref{sec:rq_embeddings} and explored in \cref{sec:exp_emb}.
We examined the behavior of the embeddings belonging autoregressive LLMs, with the \todo{ultimate} goal of understanding whether these recent model retain linear properties in their embedding spaces.
Our analysis revealed that while these properties are still present in LLMs to some extent, their prominence varies across different architectures, model sizes and is \todo{especially} influenced by the tokenization process and overall training objectives.
Additionally, our comparison of embedding and unembedding spaces highlighted crucial differences in their functional roles, particularly in models where these two spaces are distinct where we observe that input embeddings contain more semantic information \todo{rather than the latter}.
As already mentioned in \cref{ssec:exp_emb_discussion}, the proposed study on embedding linearity significantly drove the implementation of the vocabulary decoding techniques for hidden representations in InTraVisTo.
In particular, both the concept of interpolation and the idea of performing interpolation with a nonlinear \todo{pace} stem results obtained while pursuing the second research question.

The third research question (\cref{sec:exp_fom}) addressed the possibility of obtaining first-order predictions by concatenating the input and output embedding spaces of LLMs.
Building upon previous findings, we investigated whether such concatenation effectively replicates the behavior of a Markov model following the theoretical groundwork laid by~\citet{elhage2021}.
Our results indicate that the first order models (FOMs) extracted from most LLMs are generally faithful approximations of the corresponding Markov models and exhibit a solid bias towards the next-token prediction task, although the amount of \todo{closeness} to the Markov model varies greatly across model architectures and weights.
Furthermore, our analysis investigates the compatibility of our results with the practice of weight tying, concluding that the benefit from setting the unembedding weights \todo{equal to the embedding ones} is limited to a reduction in the number of overall model parameters. 
The \todo{obtained} findings tie back into InTraVisTo's view of the model by identifying the residual flow as the main communication channel between embeddings, \todo{to which the} components progressively add information as the prediction progresses through the layers.

Future work could extend InTraVisTo by adding support for new decoding techniques, model architectures and injection approaches.
Additional developments on \todo{the} secondary points treated in this work could include further experimenting and investigation on the impact of multilingual training on the embedding space LLMs or \todo{further} considerations on the feasibility of interpolating embeddings.
As already \todo{mentioned} in \cref{ssec:exp_fom_discussion}, we also acknowledge the possibility of utilizing a Markov model to initialize the embedding and unembedding weights pair for model training, which could be an interesting future development.

In conclusion, this thesis confirms previous findings and brings novel perspectives in the NLP field, providing a valuable interactive visualization tool aimed at \todo{streamlining} the study of transformer interpretability: InTraVisTo.
By providing insight and innovative tools to the explainability field we aim to contribute to the broader goal of making LLMs more transparent and interpretable.


%-------------------------------------------------------------------------
%	BIBLIOGRAPHY
%-------------------------------------------------------------------------

\addtocontents{toc}{\vspace{2em}} % Add a gap in the Contents, for aesthetics
\bibliography{Thesis_bibliography} % The references information are stored in the file named "Thesis_bibliography.bib"

%-------------------------------------------------------------------------
%	APPENDICES
%-------------------------------------------------------------------------

\cleardoublepage
\addtocontents{toc}{\vspace{2em}} % Add a gap in the Contents, for aesthetics
\appendix
\chapter{Appendix A}
If you need to include an appendix to support the research in your thesis, you can place it at the end of the manuscript.
An appendix contains supplementary material (figures, tables, data, codes, mathematical proofs, surveys, \dots)
which supplement the main results contained in the previous chapters.

\chapter{Appendix B}
It may be necessary to include another appendix to better organize the presentation of supplementary material.


% LIST OF FIGURES
\listoffigures

% LIST OF TABLES
\listoftables

% LIST OF SYMBOLS
% Write out the List of Symbols in this page
\chapter*{List of Symbols} % You have to include a chapter for your list of symbols (
\begin{table}[H]
    \centering
    \begin{tabular}{lll}
        \textbf{Variable} & \textbf{Description} & \textbf{SI unit} \\
		\midrule \\[-9px]
        $\bm{u}$ & solid displacement & m \\[2px]
        $\bm{u}_f$ & fluid displacement & m \\[2px]
    \end{tabular}
\end{table}

% ACKNOWLEDGEMENTS
\chapter*{Acknowledgements}
Here you might want to acknowledge someone.

\cleardoublepage

\end{document}
